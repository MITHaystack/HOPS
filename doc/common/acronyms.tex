%
% \section{Acronyms, Commands and Glossary}
%
\begin{acronym}
% A---------------------------------------------------------------
\acro{A-list}{a one line description of baseline fringes used by \ac{HOPS}}
\acro{alist}{a program for creating a file of \ac{A-list} scans}
\acro{aedit}{a program for editing a file of \ac{A-list} scans}
\acro{AIPS}{Astronomical Image Processing System}
% C---------------------------------------------------------------
\acro{C}{The ``C'' programming language, created to make \ac{UNIX} portable}
\acro{C++}{The C++ programming language, an object-oriented
successor to \ac{C}}
\acro{CASA}{Common Astronomy Software Applications package}
\acro{CorAsc2}{Correlator to Ascii (2nd version)}
% D---------------------------------------------------------------
\acro{DFT}{Discrete Fourier Transform}
\acro{DiFX}{the ``distributed'' \ac{FX} correlator}
\acro{difx2mark4}{a program (part of \ac{DiFX}) to convert \ac{SWIN}
format correlation products into the ``Mark4'' data files used by \ac{HOPS}}
% E---------------------------------------------------------------
\acro{EHT}{the Event Horizon Telescope, which usually refers to the
observing array}
\acro{EHTC}{the Event Horizon Telescope Collaboration, which usually
refers to the organization that operates the \ac{EHT}}
% F---------------------------------------------------------------
\acro{FFT}{Fast Fourier Transform}
\acro{FFTW3}{Fastest Fourier Transform in the West, version 3}
\acro{Fortran}{a FORmula TRANslation language, in common use prior to \ac{C}}
\acro{FX}{a general term for correlation that does the cross-correlation
after first transforming to frequency space}
% F---------------------------------------------------------------
\acro{FITS}{Flexible Image Transport System, now referring to a
general digital data format}
\acro{FITS-IDI}{A dialect of \ac{FITS}
designed for the interchange of data for interferometry}
\acro{fourfit}{the main fringe-finding command in \ac{HOPS}}
\acro{fringex}{an \ac{HOPS} tool to explore the fringe} 
% G---------------------------------------------------------------
\acro{ghostscript}{Ghostscript, the GNU postscript emulator}
\acro{GHz}{one billion Hz}
\acro{GNU}{GNU is Not Unix (a software project launched by
Richard Stallman in the 80's)}
\acro{GNU/Linux}{a family of operating systems using Linus' kernel and
GNU's software packages}
\acro{GUI}{Graphical User Interface}
% H---------------------------------------------------------------
\acro{HDF5}{Hierarchical Data Format, version 5.}\footnote{Why would you
\emph{want} to use anything that took 5 versions to get right?}
\acro{HOPS}{Haystack Observatory Postprocessing System}
\acro{Hz}{A frequency unit named for Heinrich Hertz.
A frequency of one Hz is one oscillation per second.}
% J---------------------------------------------------------------
\acro{JIVE}{now just a name for an organization, it is still an
Institution for VLBI in Europe, just not a Joint one}
% L---------------------------------------------------------------
\acro{Linux}{a family of operating systems built around Linus Torval's version
of the UNIX kernel}
\acro{LSF}{Least Squares Fit}
% M---------------------------------------------------------------
\acro{MBD}{Multi-Band Delay}
\acro{Mk4}{The fourth in a series of \ac{VLBI} hardware correlators.  The
Mark4 replaced the Mark3 near the beginning of the millenium, and was
finally put to rest by \ac{DiFX} in the mid 2010's}
\acro{MS}{Measurement Set, a formal specification for data to be analyzed
with reference to a Measurement Equation}
\acro{MSRI}{Mid-scale Research Initiatives program at the \ac{NSF}}
% N---------------------------------------------------------------
\acro{NSF}{National Science Foundation}
\acro{ngEHT}{next-generation \ac{EHT}}
% P---------------------------------------------------------------
\acro{PGPLOT}{a ``pretty good'' plotting package developed and maintained
by Tim Pearson at Caltech.  He's retired now, so it is stuck at verion 5.2.2,
(released Feb 2001)}
\acro{Python}{a programming language named in honor of Monty Python's Flying
Circus}
% S---------------------------------------------------------------
\acro{SBD}{Single Band Delay}
\acro{SFXC}{\ac{JIVE}'s software FX-kind correlator}
\acro{SNR}{Signal to Noise Ratio}
\acro{SWIN}{the output format used by the \ac{DiFX} correlator}
\acro{MHO}{MIT Haystack Observatory Postprocessing System}
% U---------------------------------------------------------------
% V---------------------------------------------------------------
\acro{VLBI}{Very Long Baseline Interferometry}
\acro{UNIX}{the name of a family of operating systems
(born in the 70's at Bell Laboratories)}
% X---------------------------------------------------------------
\acro{XF}{a general term for correlation that does the cross-correlation
first, and then transforms the result to frequency space}
% Z---------------------------------------------------------------
\end{acronym} 
%
% eof
%
