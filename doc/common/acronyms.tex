%
% \section{Acronyms, Commands and Glossary}
%
% \acro{acronym}[short name]{full name / description}
% \ac{acronym} is the usual usage in text that defines (and gives short name)
% \acs{acronym} gives just the short name
% \acf{acronym} gives just the full name
% \acsu{acronym} gives the short name and marks it used
%
% the optional short name can include math as the acronym key cannot
%
% see https://ctan.org/pkg/acronym
%
\begin{acronym}
% A---------------------------------------------------------------
\acro{A-list}{a one line description of baseline fringes used by \ac{HOPS}}
\acro{alist}{a program for creating a file of \ac{A-list} scans}
\acro{aedit}{a program for editing a file of \ac{A-list} scans}
\acro{AIPS}{Astronomical Image Processing System}
\acro{ALMA}{Atacama Large Millimeter/Submillimeter Array}
\acro{AMP}{short for ``amplitude'' the correlation coefficient}
% C---------------------------------------------------------------
\acro{C}{The ``C'' programming language, created to make \ac{UNIX} portable}
\acro{C++}{The C++ programming language, an object-oriented
    successor to \ac{C}}
\acro{CASA}{Common Astronomy Software Applications package}
\acro{channel}{an ambigous term which refers either to a spectral channel,
    \ie~frequency point of an \acsu{FFT} or to a sub-band of a
    larger receiver band.}
\acro{cofit}{a \ac{HOPS} tool to assess atmospheric coherence in terms of
    \ac{SNR} and \ac{AMP} variation with integration interval}
\acro{CorAsc2}{Correlator to Ascii (2nd version)}
\acro{cover}{a coverage test exercises all logic branches of some code module}
% D---------------------------------------------------------------
\acro{DFT}{Discrete Fourier Transform}
\acro{DR}{Delay Rate, the fringe parameter concerning
    the change of delay with time}
\acro{DiFX}{the ``distributed'' \ac{FX} correlator}
\acro{difx2mark4}{a program (part of \ac{DiFX}) to convert \ac{SWIN}
    format correlation products into the ``Mark4'' (or \acs{Mk4})
    data files used by \ac{HOPS}}
% E---------------------------------------------------------------
\acro{EHT}{the Event Horizon Telescope, which usually refers to the
    observing array}
\acro{EHTC}{the Event Horizon Telescope Collaboration, which usually
refers to the organization that operates the \ac{EHT}}
% F---------------------------------------------------------------
\acro{FFT}{Fast Fourier Transform}
\acro{FFTW3}{Fastest Fourier Transform in the West, version 3}
\acro{Fortran}{a FORmula TRANslation language, in common use prior to \ac{C}}
\acro{FX}{a general term for correlation that does the cross-correlation
    after first transforming to frequency space}
\acro{FITS}{Flexible Image Transport System, now referring to a
    general digital data format}
\acro{FITS-IDI}{A dialect of \ac{FITS}
    designed for the interchange of data for interferometry}
\acro{fourfit}{the main fringe-finding command in \ac{HOPS}}
\acro{fringex}{an \ac{HOPS} tool to explore the fringe} 
\acro{fourmer}{a program that combines data from two sub-bands into
    a larger common band}
% G---------------------------------------------------------------
\acro{ghostscript}{Ghostscript, the GNU \acs{PostScript} emulator}
\acro{GHz}{one billion Hz}
\acro{GNU}{GNU is Not Unix (a software project launched by
    Richard Stallman in the 80's)}
\acro{GNU/Linux}{a family of operating systems using Linus' kernel and
    GNU's software packages}
\acro{GS}{short for \ac{ghostscript}}
\acro{GUI}{Graphical User Interface}
% H---------------------------------------------------------------
\acro{HDF5}{Hierarchical Data Format, version 5.}\footnote{Why would you
    \textit{want} to use anything that took 5 versions to get right?}
\acro{HOPS}{Haystack Observatory Postprocessing System}
\acro{Hz}{A frequency unit named for Heinrich Hertz.
    A frequency of one Hz is one oscillation per second.}
% I---------------------------------------------------------------
\acro{i/o}{short for input/output referring to the fact that programs are
    written to act on something and provide something}
\acro{IPP}{Intel Performance Primitives is a library of functionality
    optimized for use with the Intel processor family}
% J---------------------------------------------------------------
\acro{JIVE}{now just a name for an organization, it is still an
    Institution for VLBI in Europe, just not a Joint one}
% L---------------------------------------------------------------
\acro{Linux}{a family of operating systems built around Linus Torval's version
    of the UNIX kernel}
\acro{LSF}{Least Squares Fit}
% M---------------------------------------------------------------
\acro{MBD}{Multi-Band Delay, the delay parameter referring to the change of
    phase with frequency in a multi-channel (sub-band) system.}
\acro{Mk4}{The fourth in a series of \ac{VLBI} hardware correlators.  The
    Mark4 replaced the Mark3 near the beginning of the millenium, and was
    finally put to rest by \ac{DiFX} in the mid 2010's}
\acro{m4py}{a shallow \acs{Python} wrapper which provides access to
    \acs{Mk4} data files and types}
\acro{MS}{Measurement Set, a formal specification for data to be analyzed
    with reference to a Measurement Equation}
\acro{MSRI}{Mid-scale Research Initiatives program at the \ac{NSF}}
% N---------------------------------------------------------------
\acro{NSF}{National Science Foundation}
\acro{ngEHT}{next-generation \acs{EHT}}
% O---------------------------------------------------------------
\acro{ovex}{an ``observer'' dialect of \acs{VEX}}
\acro{OpenMPI}{Open MPI Project is an open source Message Passing Interface
    implementation}
% P---------------------------------------------------------------
\acro{PDF}{Portable Document Format (developed by Adobe) as a successor
    to \ac{PostScript}}
\acro{PGPLOT}{a ``pretty good'' plotting package developed and maintained
    by Tim Pearson at Caltech.  He's retired now, so it is stuck at verion
    5.2.2, (released Feb 2001)}
\acro{PostScript}{a printer page description language developed by Adobe.
    \ac{fourfit} plots are currently generated in \ac{PostScript} and
    often converted to \acs{PDF}}
\acro{PS}{short for \ac{PostScript}}
\acro{Python}{a programming language named in honor of Monty Python's Flying
    Circus}
% S---------------------------------------------------------------
\acro{SBD}{Single Band Delay, the delay parameter referring to the time
    offset between two signals being correlated}
\acro{search}{this is a tool that searches in delay/delay-rate space to
    allow visualization of a fringe peak and to aid in establishing the
    validity of more marginal-\acs{SNR} cases}
\acro{SFXC}{\acs{JIVE}'s software \ac{FX}-kind correlator}
\acro{SNR}{Signal to Noise Ratio}
\acro{SWIN}{the output format used by the \ac{DiFX} correlator}
\acro{MHO}{MIT Haystack Observatory Postprocessing System}
% U---------------------------------------------------------------
\acro{unit}{a unit test is a short test used to validate a small part of
    some larger code module}
\acro{UNIX}{the name of a family of operating systems
    (born in the 70's at Bell Laboratories)}
% V---------------------------------------------------------------
\acro{VEX}{\acs{VLBI} EXperiment (file), a means of fully describing
    a planned \acs{VLBI} experiment or observation}
\acro{VEX2XML}{a program that converts \acs{VEX} files into an easily
    parsed \acs{XML} represention}
\acro{VGOS}{\acs{VLBI} Global Observing System;
    was called \acs{VLBI}2010 until the mid 2010's}
\acro{VLBI}{Very Long Baseline Interferometry}
% W---------------------------------------------------------------
\acro{Whitneys}{correlation amplitudes are normally expressed between 0 and 1,
    but in our work they are usually small and in \ac{HOPS} traditionally
    multiplied by ten thousand, in which case, the unit of correlation amplitude
    is ``Whitneys'' after Alan Whitney who may be commended or blamed for the
    usage.}
% X---------------------------------------------------------------
\acro{XF}{a general term for correlation that does the cross-correlation
    first, and then transforms the result to frequency space}
\acro{XML}{eXtensible Markup Language}
% Z---------------------------------------------------------------
\acro{zero-pad}{the practice of extending time or frequency sequences with
    some number of zeroes which, for \ac{FFT}s has the effect of smoothing
    in the other domain}
\end{acronym} 
%
% eof
%
