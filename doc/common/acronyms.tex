%
% \section{Acronyms, Commands and Glossary}
%
\begin{acronym}
% A---------------------------------------------------------------
\acro{A-list}{a one line description of baseline fringes used by \ac{HOPS}}
\acro{aedit}{a program for editing a file of \ac{A-list} scans}
% C---------------------------------------------------------------
\acro{CorAsc2}{Correlator to Ascii (2nd version)}
% D---------------------------------------------------------------
\acro{DiFX}{the ``distributed'' \ac{FX} correlator}
\acro{difx2mark4}{a program (part of \ac{DiFX}) to convert \ac{SWIN}
format correlation products into the ``Mark4'' data files used by \ac{HOPS}}
% F---------------------------------------------------------------
\acro{FX}{a general term for correlation that does the cross-correlation
after first transforming to frequency space}
% F---------------------------------------------------------------
\acro{fourfit}{the main fringe-finding command in \ac{HOPS}}
\acro{fringex}{an \ac{HOPS} tool to explore the fringe} 
% G---------------------------------------------------------------
\acro{GUI}{Graphical User Interface}
% H---------------------------------------------------------------
\acro{HOPS}{Haystack Observatory Postprocessing System}
% M---------------------------------------------------------------
% S---------------------------------------------------------------
\acro{SWIN}{the output format used by the \ac{DiFX} correlator}
\acro{MHO}{MIT Haystack Observatory Postprocessing System}
% X---------------------------------------------------------------
\acro{XF}{a general term for correlation that does the cross-correlation
first, and then transforms the result to frequency space}
% Z---------------------------------------------------------------
\end{acronym} 
%
% eof
%
