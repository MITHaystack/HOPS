%
% the outline
%

\section{Re-Design Considerations}
\label{sec:consider}

\subsection{Outline of Discussion Topics}
\label{sec:outline}

This section contains an outline to help organize thought.
For clarity, the discussion is shifted to paragraphs of a subsequent
subsection.

\begin{outline}[enumerate]
\1 Software features and design (\ref{sec:commentary})
  \2 General Architecture (\ref{sec:genarch})
    \3 Language choice: C/C++ with python (\ref{sec:software-lang})
    \3 Build system and version control (\ref{sec:software-build})
    \3 Options for parallel processing (\ref{sec:software-parallel})
    \3 Interactivity vs. batch processing (\ref{sec:software-interaction})
    \3 External package dependencies (\ref{sec:software-externals})
  \2 Imports from Correlator Output (\ref{sec:corr-imports})
    \3 DiFX (Swin) output (\ref{sec:difx-corr})
    \3 File conversion: difx2mark4, difx2fits, etc. (\ref{sec:corr-import})
  \2 Exports to subsequent analyses (\ref{sec:exports})
    \3 Export to imaging (UVFITS) (\ref{sec:uvfits})
    \3 Export to geodetic reductions (CALCSOLVE) \ref{sec:calcsolve}
  \2 HOPS file specifications (\ref{sec:hopsfiles})
%   \3 Output file types Mark4, FITS, HDF5, MS, alist, etc.? (\ref{sec:ftypes})
    \3 Mark4 file types \ref{sec:mk4types}
    \3 python wrappers (mk4) \ref{sec:pywrap}
    \3 alist format \ref{sec:alist}
    \3 fourfit control file \ref{sec:control}
    \3 vex2xml and vex2.0 \ref{sec:vex2xml}
  \2 New Objects (\ref{sec:newobjects})
    \3 equivalent to above
  \2 Algorithm specification (\ref{sec:algospecs})
    \3 baseline-specific delay/delay-rate fitting (\ref{sec:fringing})
    \3 global-fringe fitting (\ref{sec:globalfringe})
    \3 spectral line fringing (\ref{sec:specline})
    \3 vgos ionospheric fitting (\ref{sec:ionosphere})
    \3 coherence fitting (\ref{sec:cofit})
    \3 weak fringe searching (\ref{sec:search})
    \3 pulsar folding/searching (\ref{sec:pulsar})
    \3 space-based problems (\ref{sec:space})
    \3 Polconvert (\ref{sec:polconvert})
  \2 Calibration specification (\ref{sec:calspecs})
    \3 phase calibrations (EHT v Geodesy) (\ref{sec:phasecal})
    \3 derivation of manual phase cals (\ref{sec:manphasecal})
    \3 application of a-priori phase-cal. (\ref{sec:pulsephasecal})
    \3 instrumental bandpass-correction (\ref{sec:bandpass})
    \3 atmospheric phase correction (\ref{sec:atmosphere})
    \3 polarization dependent corrections (\ref{sec:polarization})
    \3 ionospheric corrections (\ref{sec:ionoscalcorr})
    \3 source structure (\ref{sec:sourcestructcorr}
  \2 Infrastructure (\ref{sec:infra})
    \3 output messaging (\ref{sec:msg})
    \3 data utilities (\ref{sec:utils})
    \3 performance monitoring and profiling (\ref{sec:perform})
    \3 averaging (\ref{sec:average})
  \2 Data inspection and visualization (\ref{sec:inspect})
    \3 inspection (corAsc) and import from ascii (\ref{sec:ascii})
    \3 What do we do with the fourfit plot? (\ref{sec:fplot})
    \3 Interactive tools like aedit? (\ref{sec:aedit})
    \3 Alternate visualization options? (\ref{sec:alternatives})
  \2 New Libraries (\ref{sec:libes})
    \3 equivalent to above
  \2 New Programs and scripts (\ref{sec:progs})
    \3 equivalent to above
\1 Development schedule (\ref{sec:devsched})
  \2 Pre-requisites (\ref{sec:prereq})
  \2 Re-use of existing code? (\ref{sec:reuse})
  \2 Unit test coverage (\ref{sec:unitest})
  \2 Testing/validation (\ref{sec:vandv})
  \2 Other considerations (\ref{sec:genarch})
    \3 What can be worked on in parallel? (\ref{sec:parallel-work})
    \3 What must be done sequentially? (\ref{sec:sequential-work})
    \3 What dependencies exist between modules? (\ref{sec:dependencies})
    \3 What other resources may be used (\ref{sec:otherbodies})
    \3 What parts must be supported by the geodetic team (\ref{sec:geodesy})
    \3 Clarity Descope options (\ref{sec:descope})
\end{outline}

%\subsection{Inputs and Resources}
%\label{sec:money}

%\marginnote{\tiny{at this time it is still 4x1FTE, but we have 3 years to complete}}

%\subsection{Proposed Timeline}
%\label{sec:timeline}

%marginnote{\tiny{

%A design review at the end of 2020 (end of Q05) seems to be
%desirable--corresponding roughly to the end of Q03 in this list.  The overall
%ngEHT effort will need to start working on the MSRI-II proposal around Q13,
%so at that point, we should merely be completing the work necessary to be
%shovel-ready for whatever the ngEHT decides for its proposal.%}}

%The preliminary proposal timeline called for the following general
%schedule:
%\begin{itemize}[itemsep=-1ex,label={}]
% \item Q01  Obtain input for feature definitions/requests
% \item Q02  Define software requirements
% \item Q03  Selection of architecture and file system, begin porting algorithms
% \item Q07  Review progress
% \item Q10  Finish porting algorithms
% \item Q12  Verification of new software package agains old
% \item Q14  Validation on new wideband data
% \item Q16  Final release
%\end{itemize}
%The work planned for the first two quarters will almost certainly result
%in some modifications of the general schedule, so we do not plan to fix
%the schedule at this point.  We anticipate that the general framework for
%the new HOPS will remain (we have been discussing this for several years),
%but some of the priorities and features may very well require adjustment
%by the start of Q3.

%
% eof
%
