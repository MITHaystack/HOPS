%
% Introductory material
%

\section{Introduction}
\label{sec:intro}

\subsection{EHT MSRI Project}
\label{sec:msriproj}

The Event Horizon Telescope (EHT) has launched an
MSRI (Mid-Scale Research Initiative) project which looks to develop
the technologies needed for a second-generation Event Horizon
Telescope (EHT).  This project is looking at all parts of the existing
system and looking
to scale it up to a larger and more capable array.  A significant component
of the telescope is the software needed to properly operate, reduce and
analyze the data taken by the member telescopes.  It is unfortunately true
that the development and maintenance of such critical software is historically
a side effect of other programmatics---\textit{i.e.} it is easier to get
resources to build something than it is to obtain resources to properly
analyze the data from it.  In this case, however, the MSRI project is
unusual in that it specifically allocate resources at Haystack to address
this and move the current collection
of EHT-required software into the 21-st century.  This document is a start
on organizing the thoughts in concert with the accepted MSRI proposal.

The next generation EHT is looking to support $\sim$20 stations, with
wider bandwidth (128 Gbps has been mentioned---meaning four dual-polarization,
4 GHz bands), although support for greater bit depth
is potentially also of interest should recording media be available.
The EHT to date has had an annual cycle
of observing; but the ability to make more observations per year has been
mentioned.  Without a substantial increase in analytic support manpower,
all of this implies that processing with HOPS must be made, smarter, more
automatic, more robust and easier to use.

At the same time it is critically important to recognize that the existing
HOPS (Haystack Observatory Postprocessing System)
framework is required by the geodetic community for current operations
as well as planned development to their generation of geodetic stations
(\cite{niell2018}).  At the same time, the geodetic analyses struggle
from some of the same constraints the EHT is facing.  Thus the new design and
implementation plan must repect the geodetic needs---at the end of the
process, there are not likely to be resources to support divergent HOPS
packages.  In this document we out the basic plan for development of the
package and identify the work that will transpire under the MSRI program.

\subsection{History of HOPS}
\label{sec:histhops}

At the heart of the VLBI technique is the correlation of the raw
station data using either dedicated hardware or software to find
the correlated signal from the cosmic source.  The correlation is
manifest as an interference fringe that changes in an expected way
as the Earth rotates.  This is a simple, but (computationally)
expensive process that requires good, but nevertheless approximate,
models in order to obtain useful a useful fringe.  Thus some
post-correlation processing software is required to analyze the
fringes to obtain scientifically useful results.

The current Haystack Observatory Postprocessing system (HOPS) was
born from the efforts of Alan Rogers in the late 70's with a program
called FRNGE which was written in Fortran and designed to be efficient
on an HP-21MX (later renamed HP-1000) minicomputer.  With improvements
in hardware and software, a rewrite of the toolset was launched in
the early 90's by Colin Lonsdale, Roger Cappallo and Cris Niell as
driven by the needs of the geodetic community.  The basic algorithms
were adopted from FRNGE; but there was a complete rewrite of the code
into (K\&R) C and substantial revisions of the i/o, control and file
structures resulting in the framework of the current HOPS system.
This was followed by a substantial effort in the early-mid
00's to develop tools for optimizing SNR and deriving correction factors
for data with imperfect coherence, based on analysis of amplitude with
coherent averaging time.
While there is no definitive, published, HOPS reference in the literature,
the Mark 4 Correlator paper (\cite{whitney2004mark}) touches upon the basic
implementation available by this time.
Further evolution in the late 00's was provoked
by the re-emergence of software correlation
(DiFX, \cite{deller2007difx}, \cite{deller2011difx}),
and in the 10's by the
needs of EHT-scale mm-VLBI which brings us to HOPS in its current form.

Acknowledging its geodetic heritage, HOPS was optimized for precision
on per-baseline delay and delay-rate measurements which are the raw
material for the geodetic analysis programs.  Consequently, it is
somewhat light on support for some routine calibration processes found
in some other astronomical software packages (e.g. AIPS or CASA).
Nevertheless, it provides a good framework for the reduction and
analysis of mm-VLBI data, where the vagaries of atmospheric effects
require ever more specialized processing to harvest significant
astronomical results.

For the needs of the EHT Campaigns of 2017 it was decided to augment
the existing HOPS package with some python-based packages in order to
create a pipeline for the initial reduction of data.  (See
\cite{blackburn2019eht}, and \cite{eht1}, \cite{eht2}, and \cite{eht3}).
The EHT also looked at data reduction with
other packages.  There were initial surveys of options in 2015 (Leiden
workshop) and 2016 (Nijmegen) which led to a focus on HOPS, AIPS and CASA
as the three viable options to pursue for 2017.
The Calibration and Error Analysis working group
of the EHT was able to demonstrate consistent results between the three
packages; ultimately production processing via HOPS for the EHT was the
winning solution.  In this continued development of HOPS, we shall assume
that HOPS alone must be capable of the full analysis; but we should also
be mindful that options to move the data to AIPS or CASA must at some
level exist.

\subsection{The Basic Plan}
\label{sec:theplanstan}

So our charge from the MSRI project is to update the existing HOPS and
EHT Pipeline system and produce a better organized, more useable and
flexible system for the next decades.  A significant constraint is that
HOPS is currently the critical analysis package for geodetic use.  This
is especially true with the introduction of the VGOS system
(Reference\dots).  Since we will not ultimately have resources to support
multiple systems, it is a de facto requirement that the next-generation
HOPS system work seamlessly with all of its user community (\textit{i.e.}
EHT/astronomy and geodetic).  Given all the validation work that went
into the HOPS/EHT Pipeline for the 2017 data analysis, this is not a
real restriction.  The EHT will demand the same results from the old
as well as the new HOPS.

This document will evolve into a full development plan once in the
months ahead.

%
% eof
%
