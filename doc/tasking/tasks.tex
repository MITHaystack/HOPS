\documentclass[notitlepage,letterpaper,pdftex,12pt,final]{article}
%\documentclass[notitlepage,letterpaper,pdftex,12pt,draft]{article}
% article < report < book ; preable material follows
%
% This file is mostly boilerplate that you can copy and tweak.
% The true material is in included files.  ../common should hold
% things that more than one document might need.  The Makefiles
% should likewise be mostly boilerplate with minor tweaks.
%

% (un)comment to (see)hide full paths to files actually used
% however, you'll need to override the TEXOPT= setting to batchmode
% made in the Makefile.
\listfiles

\addtolength\textwidth{1.7in}
\addtolength\oddsidemargin{-0.95in}
\addtolength\evensidemargin{-0.95in}
\addtolength\marginparwidth{-0.95in}
\typeout{textwidth is \the\textwidth}

% useful to block out portions of text
\usepackage{ifthen}
% to allow defining our own colors
\usepackage[dvipsnames]{xcolor}
% makes hyperlinks work
\usepackage{hyperref}
\hypersetup{colorlinks=true,linkcolor=darkblue,citecolor=darkgreen}
% for figures; insert [draft] before {...} not to see imgs
\usepackage{graphicx}
\graphicspath{ {../common/uml-diagrams} }
% for small figures surrounded by text
\usepackage{wrapfig}
% for control over headers and footers
\usepackage{fancyhdr}
\pagestyle{fancy}
\fancyhf{}
% for handling acronyms
\usepackage[printonlyused]{acronym}

% for control of captions
\usepackage[font=small]{caption}

% for bibliographies if needed
\usepackage[utf8]{inputenc}
\usepackage[english]{babel}
%\usepackage[backend=bibtex,style=numeric,sorting=none]{biblatex}
\usepackage[backend=biber,style=numeric,sorting=none]{biblatex}
%\bibliographystyle{authortitle}
\addbibresource{hops.bib}
%\usepackage{natbib}
% Use the appendix package.
\usepackage{appendix}
% context-sensitive quotes
\usepackage{csquotes}
% underline for emphasis: \uline etc
% -- note this breaks italics on journal names which breaks line wrapping
% \usepackage{ulem}
% for generating outlines
\usepackage{outlines}
\usepackage{enumitem}

% for bold math
\usepackage{bm}
% for serious math which we may have in a few places (eventually)
\usepackage{amsmath}
% equations include section number
\numberwithin{equation}{section}

% for marginal notations concerning geodesy
\usepackage{marginnote}

% to go back to default after \ragged commmands
\usepackage{ragged2e}

% see common/shortcuts.tex for what is defined in this file
%
% a file of commands to save typing
%

% commonly used things
\newcommand{\eg}{\textit{e.g.}}
\newcommand{\EG}{\textit{E.g.}}
\newcommand{\ie}{\textit{i.e.}}
\newcommand{\IE}{\textit{I.e.}}
\newcommand{\etc}{\textit{\&c.}}
\newcommand{\Sec}{Section}
\newcommand{\Fig}{Figure}
\newcommand{\Tab}{Table}
\newcommand{\App}{Appendix}

% for a work in progress:
\newcommand{\FIX}[1][fixme]{{\color{red}#1}}
\newcommand{\TBC}{{\color{red}TBC}}
\newcommand{\TBD}{{\color{red}TBD}}
\newcommand{\TBR}{{\color{red}TBR}}
\newcommand{\FIXME}[1][]{{\color{red}FIXME -- #1}}

% specific to this project
\newcommand{\MHO}{MIT-HOPS}
\newcommand{\HOPS}{HOPS}

% standard colors
\definecolor{darkblue}{rgb}{0,0,.5}
\definecolor{darkgreen}{rgb}{0,.5,0}
\definecolor{darkred}{rgb}{.5,0,0}
\definecolor{violet}{rgb}{.4,0.,.7}

% for coping with geodetic things
\newboolean{geos}
\setboolean{geos}{true}% or false to hide geodetic marginal notations
%
\definecolor{boxfrcolor}{rgb}{.50,.20,.00}
\definecolor{boxbgcolor}{rgb}{0.9,.98,.98}
\definecolor{boxfgcolor}{rgb}{0.4,0.0,0.0}
% \geomargin{footnote comment}
% \geomargin[{\color{..}box note}]{footnote comment}
\newcommand{\geobox}[1]%
{{\parbox{10mm}{\small\textit{\fcolorbox{boxfrcolor}{boxbgcolor}{#1}}}}}
\newcommand{\geomargin}[2][{\color{boxfgcolor}geodesy}]%
{\ifthenelse{\boolean{geos}}{% if:
\marginnote{\protect\geobox{#1}}\footnote{#2}}{% else: show nothing
}}

% for itemizing requirements
\newcounter{req}
\newcommand{\reqid}[1]{\item[\refstepcounter{req}#1-\thereq]}

%
% eof
%

\setboolean{geos}{false}% or false to hide geodetic marginal notations

% control options
\newboolean{skipappendix}
\setboolean{skipappendix}{false}

% at some point this gets frozen
\newcommand{\recdate}{\today}

\begin{document}
\DeclareGraphicsExtensions{.png, .jpg, .pdf}
% best to put figures in subdirs
%\graphicspath{{figs/}{scans/}}

% for subsequent pages
\setlength\headheight{15pt}
\fancyhead[L]{HOPS}
\fancyhead[C]{}
\fancyhead[R]{Tasking}
\fancyfoot[R]{Page \thepage\ of\ \pageref{page:LastPage}}
\fancyfoot[L]{\recdate}

\title{ngEHT Tasking for Delivery of a new HOPS}

\author{%
\LARGE John Barrett, Geoff Crew, Dan Hoak and Violet Pfeiffer \\
\Large MIT Haystack Observatory}
\date{Version 1.0, \recdate}
\maketitle
\normalsize

\renewcommand\abstractname{\large Executive Summary}
\abstract{
This document generates detailed planning graphics to make it possible to
manage the work.  It is a replacement for other, more traditional management
tools. Really.  Realistically, some of the materials generated in this
document are meant to be copied out to form the stable Softare Development
Plan (\ie what is planned).  Thereafter it becomes a living document of state
should anyone want it.  Not clear why.
}

% \begingroup . . . \endgroup can be used to keep things together
% and/or an explicit page break and/or adjust spacing as needed so
% it looks presentable.  Uncomment the sections you need.
%\begingroup
%\renewcommand\contentsname{Contents}
%\renewcommand\listfigurename{Figures}
%\renewcommand\listtablename{Tables}
%\vspace{24pt}\hrule

\pagebreak
\tableofcontents

\pagebreak
%\vspace{24pt}\hrule
\listoffigures
\pagebreak

%\vspace{24pt}\hrule
\listoftables
\pagebreak
%\vspace{24pt}\hrule
%\endgroup

% it is sometimes cleaner to start sections on new pages in a longer
% document--then changes to each section don't repage everything.

% break document into appropriate portions

\pagebreak
%
%  original proposed schedule, for reference
%
\section{Overview}
\label{sec:overview}
% Describe work to be done?
% Assumptions and Constraints
% Deliverables
% Personnel
% Personnel roles and responsibilities
% Cost management
% Review process
\subsection{Work to be done}
For the complete list of work items to be done, refer to the HOPS 4 Requirements document.
\subsection{Assumptions and Constraints}
\begin{itemize}
Community VEX Standard remains at v1.5.1.
Due to the standard hardware our users have access to, memory leaks are not of utmost concern. 
Using the new automation tool will automate and streamline the code review process.
The requirements as they are currently stated will not be modified.
The data container architecture that has been implemented is modular and flexible enough to be adapted to new data types.
\end{itemize}

\subsection{Deliverables}
\begin{itemize}
HOPS 4 Requirements Document
HOPS 4 Software Specifications Document
HOPS 4 Software Development Plan
Coverage and Test Plan
Hops 4.0 and subsequent patches
\end{itemize}

\subsection{Personnel}
\begin{itemize}
% Maybe make this in to a table
John Barrett - Lead Developer
Geoff Crew - Consultant
Dan Hoak - Developer
Violet Pfeiffer - Developer
\end{itemize}

\subsection{Cost Management}
Each team member will spend 1/4 of their time working on the HOPS 4 redevelopment project.

\subsection{Review Process}
There will be a review with the ngEHT stakeholders and the HOPS team to review the deliverables.
%
% eof
%


\pagebreak
\input{taskage}

%\pagebreak
%\addtocounter{section}{1}
%\renewcommand{\refname}{\thesection. References}
%\addcontentsline{toc}{section}{\thesection. References}
%\bibstyle{plainurl}
%\printbibliography
%\label{sec:references}

% if skipappendix-is-true then (nothing) else typeset Appendices
\ifthenelse{\boolean{skipappendix}}{}{%
\appendix

%\pagebreak
\section{Acronyms, Commands, and Glossary}
%
% \section{Acronyms, Commands and Glossary}
%
% \acro{acronym}[short name]{full name / description}
% \ac{acronym} is the usual usage in text that defines (and gives short name)
% \acs{acronym} gives just the short name
% \acf{acronym} gives just the full name
% \acsu{acronym} gives the short name and marks it used
% \a..p{acronum} makes it plural
%
% the optional short name can include math as the acronym key cannot
% there are a zillion other options, see https://ctan.org/pkg/acronym
%
\begin{acronym}
% A---------------------------------------------------------------
\acro{A-list}{a one line description of baseline fringes used by \ac{HOPS}}
\acro{adump}{a program that dumps columns from \ac{A-list} scan data}
\acro{alist}{a program for creating a file of \ac{A-list} scan data}
\acro{aedit}{a program for editing a file of \ac{A-list} scan data}
\acro{average}{a program that calculates averages on \ac{A-list} scan data}
\acro{AIPS}{Astronomical Image Processing System}
\acro{ALMA}{Atacama Large Millimeter/Submillimeter Array}
\acro{AMP}{short for ``amplitude'' the correlation coefficient}
\acro{AP}{Acquision Period which refers to a period of time over which the
    correlator integrates the input (noisy) data to produce a usable output.
    Terms such as \acs{dump} or ``integration'' are also sometimes used,
    but both can be ambiguous.}
\acro{awk}{A programmable language for parsing line and field oriented input.
    The program was part of the original \acs{UNIX} product, and is named for
    its three authors,  Alfred Aho, Peter Weinberger, and Brian Kernighan}
% B---------------------------------------------------------------
\acro{bigendian}{refers to a computer hardware architecture where the
    most significant
    bits of a larger storage object (bytes, words\ldots) are serialized first.}
\acro{bit}{a 0 or 1}
\acro{byte}{a unit of storage corresponding to 8 bits}
% C---------------------------------------------------------------
\acro{C}{The ``C'' programming language, created to make \ac{UNIX} portable}
\acro{C++}{The C++ programming language, an object-oriented
    successor to \ac{C}}
\acro{C/C++}{Refers to code that may be either \acs{C}, \acs{C++} or a mix of
    the two ``dialects''.  The two compliers currently in use in the project,
    \acs{GCC} and \acs{Clang} manage both dialects.}
\acro{CASA}{Common Astronomy Software Applications}
\acro{channel}{an ambigous term which refers either to a spectral channel,
    \ie~frequency point of an \acsu{FFT} or to a sub-band of a
    larger receiver band.}
\acro{cofit}{a \ac{HOPS} tool to assess atmospheric coherence in terms of
    \ac{SNR} and \ac{AMP} variation with integration interval}
\acro{CorAsc2}{Correlator to Ascii (2nd version)}
\acro{cover}{a coverage test exercises all logic branches of some code module}
% D---------------------------------------------------------------
\acro{DFT}{Discrete Fourier Transform}
\acro{DR}{Delay Rate, the fringe parameter concerning
    the change of delay with time}
\acro{DiFX}{the ``distributed'' \ac{FX} correlator}
\acro{difx2mark4}{a program (part of \ac{DiFX}) to convert \ac{SWIN}
    format correlation products into the ``Mark4'' (or \acs{Mk4})
    data files used by \ac{HOPS}}
\acro{dump}{a term used with hardware correlators to refer to a time
    integration performed by hardware/firmware circuitry.  The dumped
    data may then be further integrated in software.}
% E---------------------------------------------------------------
\acro{EHT}{the Event Horizon Telescope}
\acro{EHTC}{the Event Horizon Telescope Collaboration, which usually
refers to the organization that operates the \ac{EHT}}
% F---------------------------------------------------------------
\acro{FFT}{Fast Fourier Transform}
\acro{FFTW3}{Fastest Fourier Transform in the West, version 3}
\acro{Fortran}{a FORmula TRANslation language, in common use prior to \ac{C}}
\acro{FX}{a general term for correlation that does the cross-correlation
    after first transforming to frequency space}
\acro{FITS}{Flexible Image Transport System, now referring to a
    general digital data format}
\acro{FITS-IDI}{A dialect of \ac{FITS}
    designed for the interchange of data for interferometry}
\acro{flag}{A term commonly used in radio astronomy to mark bad data
    for exclusion from further analysis.}
\acro{fourfit}{the main fringe-finding command in \ac{HOPS}}
\acro{fringex}{an \ac{HOPS} tool to explore the fringe} 
\acro{fourmer}{a program that combines data from two sub-bands into
    a larger common band}
% G---------------------------------------------------------------
\acro{ghostscript}{Ghostscript, the GNU \acs{PostScript} emulator}
\acro{Gbps}{refers to data recording rate, usually.  8 Gbps is 1 GB/s
    or one billion characters (of ASCII) per second.  Usually there
    are (packet) overheads in the actual recording so the write or
    playback speed may be somethings slightly or grossly different.
    The \ac{HOPS} era started with kbps worked through Mbps and ended
    with Gbps.  Tbps will probably be with us in another decade.}
\acro{GHz}{one billion Hz}
\acro{GNU}{GNU is Not Unix (a software project launched by
    Richard Stallman in the 80's)}
\acro{GNU/Linux}{a family of operating systems using Linus' kernel and
    GNU's software packages}
\acro{grep}{global regular expression parser, a name for a collection of
    tools that perform regular expression parsing of input data strings.}
\acro{GS}{short for \ac{ghostscript}}
\acro{GSL}{\acs{GNU} Science Library, a library of functionality for
    science applications.}
\acro{GUI}{Graphical User Interface}
% H---------------------------------------------------------------
\acro{HDF5}{Hierarchical Data Format, version 5.\protect\footnote{Why would you
    \textit{want} to use anything that took 5 versions to get right?}}
\acro{HOPS}{Haystack Observatory Postprocessing System}
\acro{Hz}{A frequency unit named for Heinrich Hertz.
    A frequency of one Hz is one oscillation per second.}
% I---------------------------------------------------------------
\acro{i/o}{short for input/output referring to the fact that programs are
    written to act on something and provide something}
\acro{IPP}{Intel Performance Primitives is a library of functionality
    optimized for use with the Intel processor family}
% J---------------------------------------------------------------
\acro{JIVE}{now just a name for an organization, it is still an
    Institution for VLBI in Europe, just not a Joint one}
% L---------------------------------------------------------------
\acro{Linux}{a family of operating systems built around Linus Torval's version
    of the UNIX kernel}
\acro{littleendian}{refers to a computer hardware architecture where the
    least significant
    bits of a larger storage object (bytes, words\ldots) are serialized first.}
\acro{LSF}{Least Squares Fit}
% M---------------------------------------------------------------
\acro{MBD}{Multi-Band Delay, the delay parameter referring to the change of
    phase with frequency in a multi-channel (sub-band) system.}
\acro{Mk4}{The fourth in a series of \ac{VLBI} hardware correlators.  The
    Mark4 replaced the Mark3 near the beginning of the millenium, and was
    finally put to rest by \ac{DiFX} in the mid 2010's}
\acro{m4py}{a shallow \acs{Python} wrapper which provides access to
    \acs{Mk4} data files and types}
\acro{MS}{Measurement Set, a formal specification for data to be analyzed
    with reference to a Measurement Equation}
\acro{MSRI}{Mid-scale Research Initiative}
% N---------------------------------------------------------------
\acro{NSF}{National Science Foundation}
\acro{ngEHT}{next-generation \acs{EHT}}
% O---------------------------------------------------------------
\acro{ovex}{an ``observer'' dialect of \acs{VEX}}
\acro{OpenMPI}{Open MPI Project is an open source Message Passing Interface
    implementation}
% P---------------------------------------------------------------
\acro{PDF}{Portable Document Format (developed by Adobe) as a successor
    to \ac{PostScript}}
\acro{PGPLOT}{a ``pretty good'' plotting package developed and maintained
    by Tim Pearson at Caltech.  He's retired now, so it is stuck at verion
    5.2.2, (released Feb 2001)}
\acro{PostScript}{a printer page description language developed by Adobe.
    \ac{fourfit} plots are currently generated in \ac{PostScript} and
    often converted to \acs{PDF}}
\acro{PERL}{Practical Extraction and Reporting Language created by Larry Wall}
\acro{PS}{short for \ac{PostScript}}
\acro{Python}{a programming language named in honor of Monty Python's Flying
    Circus}
% R---------------------------------------------------------------
\acro{RFI}{Radio Frequency Interference which is what you have when your
    receiver picks up signals you do not want}
% S---------------------------------------------------------------
\acro{SBD}{Single Band Delay, the delay parameter referring to the time
    offset between two signals being correlated}
\acro{search}{this is a tool that searches in delay/delay-rate space to
    allow visualization of a fringe peak and to aid in establishing the
    validity of more marginal-\acs{SNR} cases}
\acro{sed}{is a stream editor, that ingests line-oriented data and performs
    programmatic operations on it prior to output}
\acro{SFXC}{\acs{JIVE}'s software \ac{FX}-kind correlator}
\acro{SNR}{Signal to Noise Ratio}
\acro{SWIN}{the output format used by the \ac{DiFX} correlator}
\acro{MHO}{MIT Haystack Observatory Postprocessing System}
% T---------------------------------------------------------------
\acro{TEC}{Total Electron Content, refering to the column density of
    electrons in the line of sight through the ionosphere.  Conventionally
    one TEC Unit is \protect{$10^{16}$ electrons / m$^2$}}
% U---------------------------------------------------------------
\acro{unit}{a unit test is a short test used to validate a small part of
    some larger code module}
\acro{Unicode}{here, a general reference to a collection of methods for
    representing printable characters beyond ASCII.  The painful
    \ac{Python} 2 to 3 transition was driven by a need to more correctly
    handle strings of Unicode character representations.}
\acro{UNIX}{the name of a family of operating systems
    (born in the 70's at Bell Laboratories)}
% V---------------------------------------------------------------
\acro{VEX}{\acs{VLBI} EXperiment (file), a means of fully describing
    a planned \acs{VLBI} experiment or observation}
\acro{VEX2XML}{a program that converts \acs{VEX} files into an easily
    parsed \acs{XML} represention}
\acro{VGOS}{\acs{VLBI} Global Observing System;
    was called \acs{VLBI}2010 until the mid 2010's}
\acro{VLBI}{Very Long Baseline Interferometry}
% W---------------------------------------------------------------
\acro{Whitneys}{correlation amplitudes are normally expressed between 0 and 1,
    but in our work they are usually small and in \ac{HOPS} traditionally
    multiplied by ten thousand, in which case, the unit of correlation amplitude
    is ``Whitneys'' after Alan Whitney who may be commended or blamed for the
    usage.}
\acro{word}{an architecture-dependent unit of storage---these days, most of
    our processors use 8-\acs{byte} words}
% X---------------------------------------------------------------
\acro{XF}{a general term for correlation that does the cross-correlation
    first, and then transforms the result to frequency space}
\acro{XML}{eXtensible Markup Language}
% Z---------------------------------------------------------------
\acro{zero-pad}{the practice of extending time or frequency sequences with
    some number of zeroes which, for \ac{FFT}s has the effect of smoothing
    in the other domain}
\end{acronym} 
%
% eof
%

}

\label{page:LastPage}
\end{document}
%
% eof
%
