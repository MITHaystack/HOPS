\documentclass[notitlepage,letterpaper,pdftex,12pt,final]{article}
%\documentclass[notitlepage,letterpaper,pdftex,12pt,draft]{article}
% article < report < book ; preable material follows
%
% This file is mostly boilerplate that you can copy and tweak.
% The true material is in included files.  ../common should hold
% things that more than one document might need.  The Makefiles
% should likewise be mostly boilerplate with minor tweaks.
%

% (un)comment to (see)hide full paths to files actually used
% however, you'll need to override the TEXOPT= setting to batchmode
% made in the Makefile.
\listfiles

\addtolength\textwidth{1.7in}
\addtolength\oddsidemargin{-0.95in}
\addtolength\evensidemargin{-0.95in}
\addtolength\marginparwidth{-0.95in}
\typeout{textwidth is \the\textwidth}

% useful to block out portions of text
\usepackage{ifthen}
% to allow defining our own colors
\usepackage[dvipsnames]{xcolor}
% makes hyperlinks work
\usepackage{hyperref}
\hypersetup{colorlinks=true,linkcolor=darkblue,citecolor=darkgreen}
% for figures; insert [draft] before {...} not to see imgs
\usepackage{graphicx}
% for small figures surrounded by text
\usepackage{wrapfig}
% for control over headers and footers
\usepackage{fancyhdr}
\pagestyle{fancy}
\fancyhf{}
% for handling acronyms
\usepackage[printonlyused]{acronym}

% for bibliographies if needed
\usepackage[utf8]{inputenc}
\usepackage[english]{babel}
\usepackage[backend=biber,style=numeric,sorting=none]{biblatex}
%\usepackage[backend=bibtex,style=numeric,sorting=none]{biblatex}
\addbibresource{hops.bib}
% context-sensitive quotes
\usepackage{csquotes}
% underline for emphasis: \uline etc
\usepackage{ulem}
% for generating outlines
\usepackage{outlines}
\usepackage{enumitem}

% for bold math
\usepackage{bm}
% for serious math which we may have in a few places (eventually)
\usepackage{amsmath}
% equations include section number
\numberwithin{equation}{section}

% for marginal notations concerning geodesy
\usepackage{marginnote}

% to go back to default after \ragged commmands
\usepackage{ragged2e}

% see common/shortcuts.tex for what is defined in this file
%
% a file of commands to save typing
%

% commonly used things
\newcommand{\eg}{\textit{e.g.}}
\newcommand{\EG}{\textit{E.g.}}
\newcommand{\ie}{\textit{i.e.}}
\newcommand{\IE}{\textit{I.e.}}
\newcommand{\etc}{\textit{\&c.}}
\newcommand{\Sec}{Section}
\newcommand{\Fig}{Figure}
\newcommand{\Tab}{Table}
\newcommand{\App}{Appendix}

% for a work in progress:
\newcommand{\FIX}[1][fixme]{{\color{red}#1}}
\newcommand{\TBC}{{\color{red}TBC}}
\newcommand{\TBD}{{\color{red}TBD}}
\newcommand{\TBR}{{\color{red}TBR}}
\newcommand{\FIXME}[1][]{{\color{red}FIXME -- #1}}

% specific to this project
\newcommand{\MHO}{MIT-HOPS}
\newcommand{\HOPS}{HOPS}

% standard colors
\definecolor{darkblue}{rgb}{0,0,.5}
\definecolor{darkgreen}{rgb}{0,.5,0}
\definecolor{darkred}{rgb}{.5,0,0}
\definecolor{violet}{rgb}{.4,0.,.7}

% for coping with geodetic things
\newboolean{geos}
\setboolean{geos}{true}% or false to hide geodetic marginal notations
%
\definecolor{boxfrcolor}{rgb}{.50,.20,.00}
\definecolor{boxbgcolor}{rgb}{0.9,.98,.98}
\definecolor{boxfgcolor}{rgb}{0.4,0.0,0.0}
% \geomargin{footnote comment}
% \geomargin[{\color{..}box note}]{footnote comment}
\newcommand{\geobox}[1]%
{{\parbox{10mm}{\small\textit{\fcolorbox{boxfrcolor}{boxbgcolor}{#1}}}}}
\newcommand{\geomargin}[2][{\color{boxfgcolor}geodesy}]%
{\ifthenelse{\boolean{geos}}{% if:
\marginnote{\protect\geobox{#1}}\footnote{#2}}{% else: show nothing
}}

% for itemizing requirements
\newcounter{req}
\newcommand{\reqid}[1]{\item[\refstepcounter{req}#1-\thereq]}

%
% eof
%

\setboolean{geos}{true}% or false to hide geodetic marginal notations

% control options
\newboolean{skipappendix}
\setboolean{skipappendix}{false}

% at some point this gets frozen
\newcommand{\recdate}{\today}

\begin{document}
\DeclareGraphicsExtensions{.png, .jpg, .pdf}
% best to put figures in subdirs
%\graphicspath{{figs/}{scans/}}

% for subsequent pages
\setlength\headheight{15pt}
\fancyhead[L]{HOPS}
\fancyhead[C]{}
\fancyhead[R]{Requirements}
\fancyfoot[R]{Page \thepage\ of\ \pageref{page:LastPage}}
\fancyfoot[L]{\recdate}

\title{ngEHT Specifications for a new HOPS}

\author{%
\LARGE John Barrett, Geoff Crew, Dan Hoak and Violet Pfeiffer \\
\Large MIT Haystack Observatory}
\date{Version 0.1, \recdate}
\maketitle
\normalsize

\renewcommand\abstractname{Executive Summary}
\abstract{%
\large
This document lays out the general architecture adopted for
this rewrite of \ac{HOPS}.  The orginal version of \ac{HOPS}
was written with a mix of C and F77 with some dependencies
that are becoming a challenge to satisfy.  The new architecture
will involve a mix of C/C++ for the lower-layers, and add a
Python layer to better support human interactions.  The new
implementation will be more modular, and easier to support
for the coming decades.
}

% \begingroup . . . \endgroup can be used to keep things together
% and/or an explicit page break and/or adjust spacing as needed so
% it looks presentable.  Uncomment the sections you need.
%\begingroup
%\renewcommand\contentsname{Contents}
%\renewcommand\listfigurename{Figures}
%\renewcommand\listtablename{Tables}
%\vspace{24pt}\hrule
\pagebreak
\tableofcontents
%\vspace{24pt}\hrule
%\pagebreak
%\listoffigures
%\vspace{24pt}\hrule
%\listoftables
%\vspace{24pt}\hrule
%\endgroup

% it is sometimes cleaner to start sections on new pages in a longer
% document--then changes to each section don't repage everything.
\pagebreak

% break document into appropriate portions
%\input{tbd}

\pagebreak
%
% design details
%
\section{Commentary on Software Features and Design}
\label{sec:commentary}

The following sections provide some background and initial
design discussions for the topics listed in the outline.

\subsection{General Architecture}
\label{sec:genarch}

\subsubsection{Detail on language choice}
\label{sec:software-lang}
Several aspects need to be taken into account when deciding on a choice
of programming language for this project. Namely, some of these are:
\begin{enumerate}[itemsep=-1ex]
 \item Availability of software developer expertise.
 \item The inherent performance attainable with a specific language.
 \item Availability of high performance open source utility libraries
  for math, I/O, etc.
 \item The primary language of the existing code base (C).
 \item The accessibility and ease of extensibility of the project by
  users with varying levels of experience.
\end{enumerate}
Obtaining a reasonable balance between these considerations is difficult
with a single language. Therefore it may be desirable to consider a
multi-language project, wherein the base computation layer is handled
within C/C++, but additional data manipulation can be done via optional
Python plugins embedded within the appplication or independently by
external Python scripts which have access to some of the underlying
application libraries.  C++ is a reasonable choice given current
personnel, and the possibility of reuse of portions of the existing code
base in C. It also allows for the use of a wide variety of open source libraries, bot C
and C++ (not least of which is the built in standard library which provides
access to a wide collection of basic data types (strings, vectors, maps, etc) and
algorithms (searching, sorting, etc) which reduces the required amount
of maintenance of internal code and reliance on external libraries.
Further augmenting C/C++ libraries with inter-language
communication to Python can be done via a wide variety of mature tools
(ctypes, boost.Python, SWIG, etc.), and may increases the ease at which
outside users can augment the software.  Adopting C++ would allow a
easier path to memory management (currently handled rather painfully
in the existing HOPS code base).

Note that Python 2 is no longer supported, so to be clear, all new
development will be Python 3.

\subsubsection{Build system version control}
\label{sec:software-build}
We have a working SVN repository and the system is currently
built with autotools.
Do we want to continue using SVN or move to git?
Do we want to continue using autotools or migrate to cmake?
We can consider both, but there is no rush or a compelling
reason to do either, except perhaps for long term maintenance.

\subsubsection{Options for parallel processing}
\label{sec:software-parallel}

The existing fringe-fitting process is largely a data-parallel
process operating on individual baselines with no inter-process
communication. This lends itself easily to simple parallelism using
multiple independent processes (SPMD), which has been exploited
\cite{blackburn2019eht} to deal with the EHT data volume. However, this approach
eliminates the ability to simultaneously fit for global or station-based
parameters and requires multiple iterations in order to apply successive
calibration/corrections. Therefore, if some calibration tasks are to be
done simultaneously with fringe-fringe fitting, this will require both a
substantial architectural change from the current fitting algorithm, but
also necessarily reduce the degree of (simple) parallelism available. To
accommodate this, some parallel processing will have to be addressed
within the application. There several architectural options from which
to choose to provide support for differing levels of parallelism. A
limited table of these options is detailed in the following table:

\small
\begin{tabular}{|c|c|c|c|}
\hline
Name & Classification & Hardware Scalability & Effort \\ \hline
OpenCL/CUDA & SIMD & Single machine, CPU/GPU & Low-Medium \\ \hline
pthreads & MIMD & Single machine (CPU) & Medium \\ \hline
c++11 threads & MIMD & Single machine (CPU) & Medium \\ \hline
OpenMP & SIMD & Single machine (CPU) & Low-Medium \\ \hline
OpenMPI & MIMD & Multiple machines (CPU) & High \\ \hline
\end{tabular}
\normalsize

Note that a proper re-design of the data types and low-level algorithmic
codes should make it relatively straightforward to develop various
parallel versions of the fringe fitter or analysis tools, but this should probably be done
only after profiling a single-threaded version of the code, to determine where the most crucial
bottlenecks are.

\subsubsection{Interactivity vs. batch processing}
\label{sec:software-interaction}

The existing HOPS processing is batch-oriented for fringing, but
interactive at the \texttt{aedit} stage\dots.

Batch-oriented fringe production works well at the later stages (after
tuning) but at the initial analysis stages, some flexibility is needed.
This is particularly true as the amount of information contained in
broadband fringing greatly exceeds what will fit on one page.  And
expansion to multiple pages places more burden on the analyst---too
much to look at.

In the analysis stage (e.g. \texttt{aedit}) some degree of sophistication is
needed in the selection and display of data.  Again, the volumes of
data to be understood will require smarter software.

\subsection{External package dependencies}
\label{sec:software-externals}

In order to leverage existing open source software some decisions should be made relatively early on about which external (maintained by outside unrelated entities)
should be required. For example the current library used by HOPS for executing FFT's is FFTW3, which is required and not optional, likewise, for creating fringe
plots the package pgplot is required and not optional. Wherever possible requirements should be made optional in order to keep complexity low and reduce reliance on
external maintainers, but to make the best use of existing resources some minimal subset needs to be decided upon.

\subsection{Imports from Correlator Output}
\label{sec:corr-imports}

\subsubsection{DiFX Outputs}
\label{sec:difx-corr}

At present DiFX is the de-facto correlator for the EHT.  However, there
are other correlator codes (e.g. SFXC or CorrelX) that may be useful in
the future.  Also the so-called Swinbourne output format from DiFX may evolve.
Thus we need a define a suitably generic model for the correlation output
that will be adaptable to possible future development.

In general this framework includes inputs to the correlation (e.g.
VEX and V2D) as well as correlation setup files (e.g. calc).

\subsubsection{File conversions}
\label{sec:corr-import}

The current HOPS pathway from DiFX is difx2mark4 which is very specific
to DiFX and (the current) HOPS.  For analyis in AIPS or CASA, the difx2fits
program is currently used.

Aside from file input, there is generally a desire to migrate data
between the HOPS world and the AIPS/CASA world.  Thus it is sensible
to provide pathways from the new HOPS file structure to these other
tools.

\subsection{Exports to Subsequent Analyses}
\label{sec:exports}

\subsubsection{UVFITS}
\label{sec:uvfits}
The current C\&E-WG processing pipeline ends with the generation of
the UV flavor of FITS (UVFITS).  That tool should properly be updated
to work with the new fileset and added to the HOPS tool collection.

\subsubsection{CALCSOLVE}
\label{sec:calcsolve}
The geodetic analysts current take delivery of HOPS fringe results
for import into their CALCSOLVE program which solves for Earth Orientation
Parameters and other geodetic products.  The current delivery format is
native HOPS mk4 file format.  This capability must be preserved until
they can adopt their input library to use the new HOPS.

\subsection{HOPS file specifications}
\label{sec:hopsfiles}

\subsubsection{File Types}
\label{sec:ftypes}
HOPS makes use of a filesystem, and every file consists of representations
of objects.  A uniform plan for migrating these from what we currently have
to something more sensible is a project that decomposes into the various
types discussed in the next sections.  This design task consists of
all of the general considerations before the details of the subsequent tasks.

\subsubsection{Mark4 file types}
\label{sec:mk4types}
The existing ``Mark4'' filesystem was (believe it or not) an improvement
of the previous ``Mark3'' filesystem which was closely wedded to (a now
extremely outdated) filesystem for an HP filesystem.  The modularity of
the Mark4 data types is not particularly in question.  However the binary
format is bigendian and wedded to the C structures and other capabilities
of the UNIX system that was available at the time (SunOS, a precursor of
Solaris).  Since that time a number of viable archival data formats have
been created and are in general use.  A migration of the Mark4 types to
an HDF5-based fileformat is considered to be the desirable option at this
point.  The libraries from that package are well tested and essentially
ready to use.

\subsubsection{Python Wrappers}
\label{sec:pywrap}
One of the first steps of the C\&E-WG was to create Python wrappers for
the Mark4 types so that some manipulations could be done directly in
Python.  This is now considered an essential feature and it should be
natively supported by \nuHOPS.
\marginnote{\tiny{CASA uses numpy within its python layer, so that is
likely a component of our python user interface.  Since this was written,
I've discovered that SWIG may be used to automatically generate some of
the bindings, including for numpy, if one is judicious about what you ask
SWIG to do.}}

\subsubsection{Alist}
\label{sec:alist}
Following fringing, for the data analysis stage, HOPS provided a ``oneline''
fringe format (an aline) which could be manipulated by several tools
(e.g. \texttt{alist} and \texttt{aedit}).
The amount of information that is desirable to
be captured on the aline has grown to the point of incomprehensibility,
but the need for such a thing remains.  The precise solution is likely
a topic of study as there are a number of alternatives.

\subsubsection{Fourfit Control File}
\label{sec:control}
The fringing process may be controlled (almost) equivalently on the
command line or via a ``control'' file.  This file has a peculiar
format supporting a limited set of logic contructs for setting various
parameters.  In a package making specific use of Python, it is sensible
do discard the existing control file machinery in favor of a native
Python format and conventions for command line adjustments to it.

\subsubsection{VEX file parsing}
\label{sec:vex2xml}
The VLBI Experiments are specified in a rather arcane VEX file which
is currently stuck at verion 1.5.  The community has identified a version
2.0, but it has not actually been implemented.  For use at ALMA, we
developed and recently added to HOPS a VEX2XML parser that allows the
VEX file to be parsed to XML and then standard XML libraries may be
used to extract information.  We propose to adopt 2.0 features as they
become useful to the EHT, and work through the VEX2XML parser tool.

\subsection{New Objects}
\label{sec:newobjects}
The mark4 fileset stores data in number types:  the 100 series for
correlation products, the 200 series for fringes and the 300 series
for station calibrations.  The list grew in a somewhat ad hoc fashion
and was decidedly driving by the needs of analyzing data from the old
hardware correlator.  The list of types needs to be reviewed and
re-organized for the modern era.

\subsection{Algorithmic specifications}
\label{sec:algospecs}

\subsubsection{Baseline-based delay/delay-rate fitting}
\label{sec:fringing}
At the heart of fourfit is the baseline-based delay/delay-rate fitting.
Fourfit expresses as both a so-called ``single-band delay'' (SBD, an average
over the recorded channels) as well as the ``multi-band delay'' (MBD, over the
full band).  This algorithm must of course be preserved, but it should
be noted that it may be decomposed into the data calculations and the fitting
program.  The current algorithm also allows for an ionospheric fitter.
These parts must be decomposed and put in library functions for more flexible
use.

\subsubsection{Global Fringe Fitting}
\label{sec:globalfringe}
In particular, given a saner organization for the fitting, it should be
possible to provide the (more conventional, as provided in AIPS and CASA)
global fringe fitter which assigns results on a station basis.  This requires
a(t least one) least-squares fitting technique to be implemented.

\subsubsection{Spectral Line Fringing}
\label{sec:specline}
There are developments in mmVLBI that should allow work with spectral line
(i.e. narrow) sources; support for this capability should be in place in
the \nuHOPS.

\subsubsection{Ionospheric Fitting}
\label{sec:ionosphere}
VGOS currently fits for the total electron content (TEC) of the ionosphere
as part of the high-delay precision fits.  This is likely not needed at
the high frequencies the EHT uses, but it definitely must be preserved.

\subsubsection{Coherence Fitting}
\label{sec:cofit}
The current HOPS tool \texttt{cofit} allows one to identify the shortest
averaging interval with the maximum SNR.  The algorithm and plotting
artifacts should be continue to be supported as the tool is still useful.

\subsubsection{Weak Fringe Searching}
\label{sec:search}
The current HOPS tool \texttt{search} allows one to examine the two-dimensional
delay and delay-rate space to ascertain whether weak peaks are likely to be
real or not.  The tool is still of some use and should be retained.

\subsubsection{Pulsar Folding/Searching}
\label{sec:pulsar}
Support for folding data on a known pulsar period, or for searching
for pulsar periodicities in data was at one point contemplated for HOPS.
It may be useful to consider this in the new architecture.

\subsubsection{Space Based Support}
\label{sec:space}

If there is a future possibility of space based radio telescopes participating in the EHT network, this may require some additional features in the fringe fitting software, such
as compensating for higher order delay residual terms (delay-acceleration) in addition to the linear delay/delay-rate model used for ground based stations.

\subsubsection{PolConvert}
\label{sec:polconvert}
The ALMA observatory uses a linear, rather than a circular polarization
basis.  The current practice is to follow the correlation process (into
a mixed basis) by a polarization conversion step (using the tool
\texttt{PolConvert}).  In principle this step could be carried out in
the post-processing stage within \nuHOPS.  The architecture should
support this.

\subsection{Calibration Specification}
\label{sec:calspecs}

\subsubsection{Phase Calibration}
\label{sec:phasecal}
The EHT generally has used manual phase cals (no other alternative).
The geodetic sites generally have a pulsed tone phase cal system.
Depending on whether a pulsed system can be developed for the EHT,
it may in principal need to use both methods.

\subsubsection{Manual Phase Calibration}
\label{sec:manphasecal}
Lacking a pulse phase cal system at all of its observatories, the EHT
generally relies on a manual phase cal procedure.  There are several
scripts that have been written to estimate the manual phase cals (at
each station) on one bright scan and then place these phases into the
control file.  Likewise, delays may be estimated from a single scan
and placed in the control file.  Scripts to continue to do this must
be supported in the \nuHOPS~architecture.

\subsubsection{Pulse Phase Calibration}
\label{sec:pulsephasecal}

Pulse phase calibration is a method of compensating for the time varying phase and delay response of a telescope's receiving system. This is done
by injecting a train of sharp pulses (a Dirac comb) which are synchronized with the site's reference clock. This train of pulses is equivalent
to set of equally separated tones in frequency space. Since the tones are known to be in phase at the point of injection, it is then possible to monitor
the frequency dependent phase changes made to the incoming signal by the receiving system by tracking the accumulated phase of each tone.
This technique allows for the elimination of the (possibly time-varying) dispersion introduced by the receiving system.

Typically this calibration process is applied to the data in two steps after it is recorded. The first step is the extraction of the phase calibration tone phases. This is primarily done by
the correlator by calculating the in-phase and quadrature components of the stations' signals at each of the discrete tone frequencies. The second step is done during the fringe-fitting process,
where each stations tone-phase data is applied to correct the correlated signal. When there are many tone available across the correlated bandwidth, this process also allows for the correction of
instrumental delays (higher order terms are not currently considered). However, as is often the case, the pulse calibration data is not perfect and it can be significantly contaminated by RFI or weak tones.
The current implementation of (multiple tone) phase-calibration in fourfit is fairly robust, and admits some ability to mask bad tone data. However, there is ample room for improvement in the automatic
flagging of bad phase calibration data, which would be especially useful in the case where the data quality varies across both the time and frequency domain.


\subsubsection{Instrumental Bandpass Calibration}
\label{sec:bandpass}
It should be possible in \nuHOPS~to perform an instrumental bandpass
correction as is available in other packages.  Here one or more scans
on a bright calibrator may be used to establish the per-station deviation
of the bandpass from the optimal flat response.  These derived bandpass
solutions may then be applied to adjust the correlation amplitude as a
function of frequency.  Such a bandpass correction  may also be fully
complex (i.e. phase adjustments as well).

\subsubsection{Atmospheric Phase Calibration}
\label{sec:atmosphere}

The currently implmentation of fourfit allows from some limited ability to handle atmospheric phase calibration. Currently, this is done via the introduction of ad-hoc phases, which are applied independently from the formal fitting procedure for delay/delay-rate.
In the existing EHT calibration pipeline, these ad-hoc phases are estimated (after data-flagging and some bandpass correction have been applied) from data residuals after the fringe-fitting. The smoothed phase corrections estimated from the residuals
must then be exported to a ad-hoc phase file in order to be applied on the next pass. This process (while generic) is time consuming and requires the presence of a reference station with good SNR to the majority of stations in the network. Therefore, it
desirable to have a dedicated algorithm to estimate and apply atmospheric phase calibration within the fringe-fitting process itself.


\subsubsection{Polarization Calibration Corrections}
\label{sec:polarization}
In the current EHT analysis, network calibration techniques are used to self-calibrate the array for polarization work.  This is an area
to discuss with the other working groups what form of support in \nuHOPS~would be most effective. In addition, good estimates of station-based polarization calibration parameters
also enables the coherent summation (Stokes-I) of individual polarization-products resulting in higher SNR observations. This is also of great interest for geodetic observations, and important
for observations between stations with mixed polarization types (circular-linear).

\subsubsection{Ionopheric Calibration Corrections}
\label{sec:ionoscalcorr}

While the ionosphere does not typically play much of a roll at the high frequencies at which the EHT is observing (230, 345 GHz), knowledge of the ionosphere is crucial for obtaining
good results during geodetic observations that occur at lower frequencies (2-14 GHz) where the dispersion it causes is much stronger. Currently, fitting for the(line-of-site) differential total electron content ($\Delta$TEC) of the
ionosphere can currently be done from geodetic VLBI observations themselves during fringe-fitting on a single baseline. However, the current implementation could be strengthened if the fringe-fitting architecture were extended to allow for the possibility
of simultaneously fitting for the line-of-site TEC associated with station, as a station (rather than baseline) based quantity. This would help in the reduction of non-closing $\Delta$ TEC errors, and which can currently only be
done in an ad-hoc manner. Additionally, fitting for the $\Delta$TEC is sensitive to residual instrumental phase effects which can be difficult to detect unless examining data from multiple single baselines/scans.


\subsubsection{Source Structure Corrections}
\label{sec:sourcestructcorr}

One explicitly geodetic feature that may be desirable in the \nuHOPS is the possibility to introduce phase and delay corrections to compensate for source structure given an image model of a non-point like radio source. Having this ability would
eliminate a significant source of systematic error in geodetic delay observables, and allow for a larger catalog of usable (bright) sources for geodetic observations.



\subsection{Infrastructure}
\label{sec:infra}

\subsubsection{Messaging}
\label{sec:msg}
HOPS currently passes commentary to the user via a uniform messaging
service.  This allows the user to turn on verbosity or run completely
silently.  This should be preserved, but due to the potential volume
of such messages, the control should become finer grained---i.e. the
user should be able to specify what portions of code are making comments.

\subsubsection{Utilities}
\label{sec:utils}
HOPS has a few utilities for date or geometry calculations---these
should be reviewed and clearly supported.

\subsubsection{Performance}
\label{sec:perform}
HOPS has a performance monitoring system built into the existing code.
This should be retained and augmented by more sophisticated profiling tools.

\subsubsection{Averaging}
\label{sec:average}
Time averaging of data was added ad hoc to several of the tools; we
propose to build this capability in directly with native support.

\subsection{Data Inspection and Visualization}
\label{sec:inspect}

\subsubsection{To/From ASCII}
\label{sec:ascii}
Every bit of data that HOPS works with should be available in a human-readable
form.  So every object in the new design should support methods to export to
or import from some human readable format.

\subsubsection{The Fourfit Plot}
\label{sec:fplot}
The current fringe fitter, \texttt{fourfit} produces a single page summary
of the fringe result for every baseline.  In the high-bandwidth modern era,
one is hard pressed to capture everything in a readable format.  A single-page
format is still desirable, as is the ability to page through multiple plots
(i.e. as can be done with \texttt{fplot}), but some controls over the
information that is displayed is probably desirable.  Thus in \nuHOPS,
there will still be a standard page, but one may also create custom formats
which may be more useful for some experiment setups.

\subsubsection{Interactive Tools}
\label{sec:aedit}
The post-fringe processing in HOPS begins with \texttt{alist} which provides
one line summaries of every fringe.  These ``alist'' files may then be
used to make data-quality selections or support inspection (\texttt{fplot} of
errant fringes).  This capability must be retained, but it can almost certainly
be better implemented in Python, which would make it more extensible.

\subsubsection{Alternative Visualization}
\label{sec:alternatives}
Additional visualizations of data should be provided as these are often
useful for understanding subtle issues with data.  Again, a Python layer
providing access to the underlying \nuHOPS~data formats would probably be
the most elegant solution.

\subsection{New Libraries}
\label{sec:libes}
The existing algorithms should be encoded into new library methods
that have clear inputs, outputs and side-effects.  (In the current
HOPS, there are many side-effects to global variables, so the methods
cannot be directly reused as coded.)

\subsection{New Programs}
\label{sec:progs}
With the new architecture in place, versions of the existing programs
(e.g. \texttt{fourfit}, \texttt{alist} and \texttt{aedit}) should be
provided along with new, improved tools for the desired new capabilities.

%
% eof
%


\pagebreak
%
%  Annotated source tree; regenerate the list with
%
%   find . \( -name .git -o -name cmake -o -name m4 -o \
%             -name auto\* -o -name src -o -name lib -o \
%             -name include \) -prune -o -type d -print
%

\section{Annotated Source Tree}
\label{sec:srctree}

\FIXME[need a python script to do the find, or similar and re-generate
appropropriate section headings]

\tiny
\begin{verbatim}
./doc
./doc/covertest
./doc/common
./doc/developer
./doc/develplan
./doc/reference
./doc/userman
./doc/code_outline
./doc/geodetic
./doc/requirements
./doc/specification
./doc/specification/code
./doc/testreports
./source
./source/c_src
./source/c_src/mk4util
./source/c_src/afio
./source/c_src/applications
./source/c_src/applications/alist
./source/c_src/applications/aedit
./source/c_src/applications/fourfit
./source/c_src/dfio
./source/c_src/vex
./source/c_src/vex/text
./source/c_src/fourfit_libs
./source/c_src/fourfit_libs/ffplot
./source/c_src/fourfit_libs/ffsearch
./source/c_src/fourfit_libs/ffcontrol
./source/c_src/fourfit_libs/ffcore
./source/c_src/fourfit_libs/ffio
./source/c_src/fourfit_libs/ffmath
./source/python_src
./source/python_src/scripts
./source/python_src/hops
./source/bash_src
./source/cpp_src
./source/cpp_src/Messaging
./source/cpp_src/MK4Interface
./source/cpp_src/Bindings
./source/cpp_src/Utilities
./source/cpp_src/Test
./source/cpp_src/Containers
\end{verbatim}



\addtocounter{section}{1}
\renewcommand{\refname}{\thesection. References}
\addcontentsline{toc}{section}{\thesection. References}
\bibstyle{plainurl}
\printbibliography
\label{sec:references}

% if skipappendix-is-true then (nothing) else typeset Appendices
\ifthenelse{\boolean{skipappendix}}{}{%

\pagebreak
\appendix
\section{Acronyms, Commands and Glossary}
%
% \section{Acronyms, Commands and Glossary}
%
% \acro{acronym}[short name]{full name / description}
% \ac{acronym} is the usual usage in text that defines (and gives short name)
% \acs{acronym} gives just the short name
% \acf{acronym} gives just the full name
% \acsu{acronym} gives the short name and marks it used
% \a..p{acronum} makes it plural
%
% the optional short name can include math as the acronym key cannot
% there are a zillion other options, see https://ctan.org/pkg/acronym
%
\begin{acronym}
% A---------------------------------------------------------------
\acro{A-list}{a one line description of baseline fringes used by \ac{HOPS}}
\acro{adump}{a program that dumps columns from \ac{A-list} scan data}
\acro{alist}{a program for creating a file of \ac{A-list} scan data}
\acro{aedit}{a program for editing a file of \ac{A-list} scan data}
\acro{average}{a program that calculates averages on \ac{A-list} scan data}
\acro{AIPS}{Astronomical Image Processing System}
\acro{ALMA}{Atacama Large Millimeter/Submillimeter Array}
\acro{AMP}{short for ``amplitude'' the correlation coefficient}
\acro{AP}{Acquision Period which refers to a period of time over which the
    correlator integrates the input (noisy) data to produce a usable output.
    Terms such as \acs{dump} or ``integration'' are also sometimes used,
    but both can be ambiguous.}
\acro{awk}{A programmable language for parsing line and field oriented input.
    The program was part of the original \acs{UNIX} product, and is named for
    its three authors,  Alfred Aho, Peter Weinberger, and Brian Kernighan}
% B---------------------------------------------------------------
\acro{bigendian}{refers to a computer hardware architecture where the
    most significant
    bits of a larger storage object (bytes, words\ldots) are serialized first.}
\acro{bit}{a 0 or 1}
\acro{byte}{a unit of storage corresponding to 8 bits}
% C---------------------------------------------------------------
\acro{C}{The ``C'' programming language, created to make \ac{UNIX} portable}
\acro{C++}{The C++ programming language, an object-oriented
    successor to \ac{C}}
\acro{C/C++}{Refers to code that may be either \acs{C}, \acs{C++} or a mix of
    the two ``dialects''.  The two compliers currently in use in the project,
    \acs{GCC} and \acs{Clang} manage both dialects.}
\acro{CASA}{Common Astronomy Software Applications}
\acro{channel}{an ambigous term which refers either to a spectral channel,
    \ie~frequency point of an \acsu{FFT} or to a sub-band of a
    larger receiver band.}
\acro{cofit}{a \ac{HOPS} tool to assess atmospheric coherence in terms of
    \ac{SNR} and \ac{AMP} variation with integration interval}
\acro{CorAsc2}{Correlator to Ascii (2nd version)}
\acro{cover}{a coverage test exercises all logic branches of some code module}
% D---------------------------------------------------------------
\acro{DFT}{Discrete Fourier Transform}
\acro{DR}{Delay Rate, the fringe parameter concerning
    the change of delay with time}
\acro{DiFX}{the ``distributed'' \ac{FX} correlator}
\acro{difx2mark4}{a program (part of \ac{DiFX}) to convert \ac{SWIN}
    format correlation products into the ``Mark4'' (or \acs{Mk4})
    data files used by \ac{HOPS}}
\acro{dump}{a term used with hardware correlators to refer to a time
    integration performed by hardware/firmware circuitry.  The dumped
    data may then be further integrated in software.}
% E---------------------------------------------------------------
\acro{EHT}{the Event Horizon Telescope}
\acro{EHTC}{the Event Horizon Telescope Collaboration, which usually
refers to the organization that operates the \ac{EHT}}
% F---------------------------------------------------------------
\acro{FFT}{Fast Fourier Transform}
\acro{FFTW3}{Fastest Fourier Transform in the West, version 3}
\acro{Fortran}{a FORmula TRANslation language, in common use prior to \ac{C}}
\acro{FX}{a general term for correlation that does the cross-correlation
    after first transforming to frequency space}
\acro{FITS}{Flexible Image Transport System, now referring to a
    general digital data format}
\acro{FITS-IDI}{A dialect of \ac{FITS}
    designed for the interchange of data for interferometry}
\acro{flag}{A term commonly used in radio astronomy to mark bad data
    for exclusion from further analysis.}
\acro{fourfit}{the main fringe-finding command in \ac{HOPS}}
\acro{fringex}{an \ac{HOPS} tool to explore the fringe} 
\acro{fourmer}{a program that combines data from two sub-bands into
    a larger common band}
% G---------------------------------------------------------------
\acro{ghostscript}{Ghostscript, the GNU \acs{PostScript} emulator}
\acro{Gbps}{refers to data recording rate, usually.  8 Gbps is 1 GB/s
    or one billion characters (of ASCII) per second.  Usually there
    are (packet) overheads in the actual recording so the write or
    playback speed may be somethings slightly or grossly different.
    The \ac{HOPS} era started with kbps worked through Mbps and ended
    with Gbps.  Tbps will probably be with us in another decade.}
\acro{GHz}{one billion Hz}
\acro{GNU}{GNU is Not Unix (a software project launched by
    Richard Stallman in the 80's)}
\acro{GNU/Linux}{a family of operating systems using Linus' kernel and
    GNU's software packages}
\acro{grep}{global regular expression parser, a name for a collection of
    tools that perform regular expression parsing of input data strings.}
\acro{GS}{short for \ac{ghostscript}}
\acro{GSL}{\acs{GNU} Science Library, a library of functionality for
    science applications.}
\acro{GUI}{Graphical User Interface}
% H---------------------------------------------------------------
\acro{HDF5}{Hierarchical Data Format, version 5.\protect\footnote{Why would you
    \textit{want} to use anything that took 5 versions to get right?}}
\acro{HOPS}{Haystack Observatory Postprocessing System}
\acro{Hz}{A frequency unit named for Heinrich Hertz.
    A frequency of one Hz is one oscillation per second.}
% I---------------------------------------------------------------
\acro{i/o}{short for input/output referring to the fact that programs are
    written to act on something and provide something}
\acro{IPP}{Intel Performance Primitives is a library of functionality
    optimized for use with the Intel processor family}
% J---------------------------------------------------------------
\acro{JIVE}{now just a name for an organization, it is still an
    Institution for VLBI in Europe, just not a Joint one}
% L---------------------------------------------------------------
\acro{Linux}{a family of operating systems built around Linus Torval's version
    of the UNIX kernel}
\acro{littleendian}{refers to a computer hardware architecture where the
    least significant
    bits of a larger storage object (bytes, words\ldots) are serialized first.}
\acro{LSF}{Least Squares Fit}
% M---------------------------------------------------------------
\acro{MBD}{Multi-Band Delay, the delay parameter referring to the change of
    phase with frequency in a multi-channel (sub-band) system.}
\acro{Mk4}{The fourth in a series of \ac{VLBI} hardware correlators.  The
    Mark4 replaced the Mark3 near the beginning of the millenium, and was
    finally put to rest by \ac{DiFX} in the mid 2010's}
\acro{m4py}{a shallow \acs{Python} wrapper which provides access to
    \acs{Mk4} data files and types}
\acro{MS}{Measurement Set, a formal specification for data to be analyzed
    with reference to a Measurement Equation}
\acro{MSRI}{Mid-scale Research Initiative}
% N---------------------------------------------------------------
\acro{NSF}{National Science Foundation}
\acro{ngEHT}{next-generation \acs{EHT}}
% O---------------------------------------------------------------
\acro{ovex}{an ``observer'' dialect of \acs{VEX}}
\acro{OpenMPI}{Open MPI Project is an open source Message Passing Interface
    implementation}
% P---------------------------------------------------------------
\acro{PDF}{Portable Document Format (developed by Adobe) as a successor
    to \ac{PostScript}}
\acro{PGPLOT}{a ``pretty good'' plotting package developed and maintained
    by Tim Pearson at Caltech.  He's retired now, so it is stuck at verion
    5.2.2, (released Feb 2001)}
\acro{PostScript}{a printer page description language developed by Adobe.
    \ac{fourfit} plots are currently generated in \ac{PostScript} and
    often converted to \acs{PDF}}
\acro{PERL}{Practical Extraction and Reporting Language created by Larry Wall}
\acro{PS}{short for \ac{PostScript}}
\acro{Python}{a programming language named in honor of Monty Python's Flying
    Circus}
% R---------------------------------------------------------------
\acro{RFI}{Radio Frequency Interference which is what you have when your
    receiver picks up signals you do not want}
% S---------------------------------------------------------------
\acro{SBD}{Single Band Delay, the delay parameter referring to the time
    offset between two signals being correlated}
\acro{search}{this is a tool that searches in delay/delay-rate space to
    allow visualization of a fringe peak and to aid in establishing the
    validity of more marginal-\acs{SNR} cases}
\acro{sed}{is a stream editor, that ingests line-oriented data and performs
    programmatic operations on it prior to output}
\acro{SFXC}{\acs{JIVE}'s software \ac{FX}-kind correlator}
\acro{SNR}{Signal to Noise Ratio}
\acro{SWIN}{the output format used by the \ac{DiFX} correlator}
\acro{MHO}{MIT Haystack Observatory Postprocessing System}
% T---------------------------------------------------------------
\acro{TEC}{Total Electron Content, refering to the column density of
    electrons in the line of sight through the ionosphere.  Conventionally
    one TEC Unit is \protect{$10^{16}$ electrons / m$^2$}}
% U---------------------------------------------------------------
\acro{unit}{a unit test is a short test used to validate a small part of
    some larger code module}
\acro{Unicode}{here, a general reference to a collection of methods for
    representing printable characters beyond ASCII.  The painful
    \ac{Python} 2 to 3 transition was driven by a need to more correctly
    handle strings of Unicode character representations.}
\acro{UNIX}{the name of a family of operating systems
    (born in the 70's at Bell Laboratories)}
% V---------------------------------------------------------------
\acro{VEX}{\acs{VLBI} EXperiment (file), a means of fully describing
    a planned \acs{VLBI} experiment or observation}
\acro{VEX2XML}{a program that converts \acs{VEX} files into an easily
    parsed \acs{XML} represention}
\acro{VGOS}{\acs{VLBI} Global Observing System;
    was called \acs{VLBI}2010 until the mid 2010's}
\acro{VLBI}{Very Long Baseline Interferometry}
% W---------------------------------------------------------------
\acro{Whitneys}{correlation amplitudes are normally expressed between 0 and 1,
    but in our work they are usually small and in \ac{HOPS} traditionally
    multiplied by ten thousand, in which case, the unit of correlation amplitude
    is ``Whitneys'' after Alan Whitney who may be commended or blamed for the
    usage.}
\acro{word}{an architecture-dependent unit of storage---these days, most of
    our processors use 8-\acs{byte} words}
% X---------------------------------------------------------------
\acro{XF}{a general term for correlation that does the cross-correlation
    first, and then transforms the result to frequency space}
\acro{XML}{eXtensible Markup Language}
% Z---------------------------------------------------------------
\acro{zero-pad}{the practice of extending time or frequency sequences with
    some number of zeroes which, for \ac{FFT}s has the effect of smoothing
    in the other domain}
\end{acronym} 
%
% eof
%


}

\label{page:LastPage}
\end{document}
