

\section{Language, Build and Version Control System}
\label{sec:language}

Several aspects need to be taken into account when deciding on a choice of programming language for this project. Namely, some of these are:
\begin{enumerate}
 \item Availability of software developer expertise.
 \item The inherent performance attainable with a specific language.
 \item Availability of high performance open source utility libraries for math, I/O, etc.
 \item The primary language of the existing code base (C).
 \item The accessibility and ease of extensibility of the project by users with varying levels of experience
\end{enumerate}

Obtaining a reasonable balance between these considerations is difficult with a single language. Therefore we
plan to develop a multi-language project, wherein the base computation layer is handled within C/C++,
but additional data manipulation can be done via optional Python plugins embedded within the application
or independently by external Python scripts which have access to some of the underlying application libraries.
C++ is a good choice given current personnel and developer expertise, and being a super-set of the C language
it provides the ability to reuse of portions of the existing C code base with little to no change, while also adding modern language features
(templates, classes and inheritance, const. correctness, function overloading, etc.). A combined C/C++ approach
allows for much easier memory management (currently handled rather painfully in the existing HOPS code base) and also
enables the use of a wide variety of open source libraries, not least of which is the built in standard template library
which provides access to a wide collection of basic data types (strings, vectors, maps, etc) and algorithms (searching, sorting, etc.)
which will reduce the required amount of maintenance of internal code and reliance on external libraries.

Further augmenting C/C++ libraries with inter-language communication to Python can be done via a wide variety of mature tools
(ctypes, boost.Python, SWIG, pybind11, etc.), and will increase the ease at which outside users can augment the software without
needing to have expertise in C/C++/. Since Python 2 is no longer supported, all new development will be Python 3.

The build system for this project will be automake\urlfootnote{https://www.gnu.org/software/automake/} or CMake\urlfootnote{https://cmake.org}, and version control will proceed through a locally hosted (Haystack) git repository. HOPS3 used the automake build system and there is some
advantage in re-using that frame work, while the CMake build system is generally easier to maintain than the current automake system when faced with the complexities of a multi-language project. It also allows for a more user-friendly configuration at the time of compilation, as the user can be presented with a menu providing options which are dependent on the available set of tools/libraries that are currently installed and detected on the users system. Initialy
both build systems may be implemented to see which provides an optimal choice. While version control will be handled in a local \textit{git} repository during
the prime development phase, it is expected that eventually a public \textit{git} repository will be made available for releases which will also make it much easier to leverage community contributions if so desired at some point in the future.
