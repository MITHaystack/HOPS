%
% the meat of the thing
%
\section{What was promised at HILO}
\label{sec:hilo}

This appendix captures what was presented at the Hilo
EHT meeting (Dec. 2019) in a software session intended to
capture feedback on the plan.  The discussion
did not provide any clear guidance to adjust our plans.

\newcommand{\sbitem}{\hfill\break\hspace{2mm}$\diamond$\ }

\subsection{slide 2}
\begin{itemize}
\item HOPS performed adequately for 2017 data reduction
\sbitem consistency with AIPS and CASA was established
\sbitem the HOPS-based pipeline was adopted for production work\footnote{%
after validation against the alternatives using AIPS and CASA.
We consider it still highly valuable to update \ac{HOPS}, and indeed, it
makes sense to enable interoperability with CASA---however, that is outside
of our scope.}
\item The ngEHT funding via MSRI-I calls for updates to HOPS:
\sbitem 4 years of development to become “shovel ready”o
\sbitem N stations more than doubles (\eg to 20-30)
\sbitem Bandwidth quadruples at some sites (\eg 8 GHz)
\sbitem Simultaneous 230 / 345 GHz observing perhaps
\sbitem Anything else that gets decided in then next 4 years
\end{itemize}

\subsection{slide 3}
\begin{itemize}
\item We’ve (almost) come to the end of what can be fixed with “bandaids”
\item Maximum \# of channels is 64 and that’s hard to fix (a-zA-Z0-9\%\$)
\item No full complex bandpass corrections (only per-channel phase+delay)
\item single-baselines are both a virtue and a problem:
\sbitem HOPS finds more fringes than AIPS or CASA
\sbitem station-based phases and delays are not readily accessible
\item Fourfit is a one-shot process; multi-step processing not supported
\item User interface is a challenge:
\sbitem control file syntax is a bit arcane
\sbitem all you get is one fourfit plot that either works or is garbage
\end{itemize}

\subsection{slide 3}
\begin{itemize}
\item HOPS code has 30+ years of history in it
\sbitem was coded in C, but reads like the Fortran it was ported from
\sbitem not modular except for a few of the i/o libraries
\sbitem was written for hardware correlators
\sbitem was written for computers that no longer exist
\sbitem little endianism won out over big endianism (apparently)
\sbitem was (successfully) successfully adapted to DiFX (but not \eg SFXC)
\item Plotting and results are not independently generated
\sbitem amplitude and SNR come as side-effects of what you plotted
\sbitem PGPLOT is maybe ok today, but not really supported anymore
\end{itemize}

\subsection{slide 4}
\begin{itemize}
\item Global fringe solutions (and station based-quantities)
\item Complex bandpass
\item A more human-friendly interface (\eg Python as CASA does)
\item Allow distributed computing and/or parallelization (threads, OpenMPI)
\item Insert hooks to allow plug-in modules for customizations as needed
\item Allow a strategy for iterative calibration and fringing
\item Improved data formats (internal in-memory as well as disk storage)
\item Enable better exchange with other analysis packages:
\sbitem FITS-IDI? (or HDF5 or whatever else comes along?), CASA MS?, ...
\sbitem (either enables better use of HOPS with simulated data)
\item A more flexible/interactive plotting system
\sbitem single summary is fine when everything is working
\sbitem provide real support for investigation of problems
\end{itemize}

\subsection{slide 5}
\begin{itemize}
\item Maintain existing tools “as is” for serious regression (probably patched)
\item Arbitrary number of channels; eliminate internal magic sizing numbers
\item New control file format (\eg use python or some config module)
\item New internal data formats (rationalized, new structures or objects)
\item New disk data formats: “mk4” $\rightarrow$ “hops”
\sbitem machine/compiler independent little-endian (not big-endian)
\sbitem rationalized data types (as with internal formats, optimized for disk i/o)
\sbitem new root file format (ovex is ancient history, and current root is artificial)
\sbitem preserve the current m4py-type capability
\sbitem allow translator tools to exchange with “hops”, “mk4” and other formats
\end{itemize}

\subsection{slide 6}
\begin{itemize}
\item Basically: FIX what is broken
\item Not gratuitously break the current pipelines, but allow simplification
\item Most likely to be implemented in a mix of C, C++ or Python
\item Provide a more canonical adaptation to unix/linux environments
\item Implement what is most important to have available in 4 years
\end{itemize}
%
% eof
%
