%
% the meat of the thing
%
% this is a counter for consecutive numbering
\newcounter{req}
\newcommand{\reqid}[1]{\item[\refstepcounter{req}#1-\thereq]}

\section{Testable Requirements}
\label{sec:testable}

This section enumerates specific requirements that we plan to meet by test.
These are partitioned into subsections by automated test
(\Sec~\ref{sec:auto}, \ie~these that that may be executed
in a nightly build), software tests that may be run at specific
juctions (\Sec~\ref{sec:regress}, \ie~regression tests),
and tests that are procedural, requiring human interaction
(\Sec~\ref{sec:procedure}, \ie~\ac{GUI} tests).%
\geomargin{In addition to these requirements, there are a few additional
geodetic requirements which are captured outside the ngEHT effort.}

\ac{HOPS} is currently in the version 3.x series; the new code will
begin with the 4.x series, and to distinguish the two flavors, (in code
or otherwise) we shall use \ac{MHO} for the latter.

The requirements in this document are numbered with a prefix
according to the type of test that may be employed.

\subsection{Automated Testable Requirements}
\label{sec:auto}

It is anticipated that each of these tests will be captured by a
``check'' (shell or Python) script (such as currently supported in
the autotools or created to more fully ensure safe development).

\begin{description}
\reqid{A} Every data type written to disk in \ac{MHO} shall be 
amenable to human-comprehensible examination as is done by the
\texttt{\ac{CorAsc2}} program.
\reqid{A} Implement regression tests. \FIXME[enumerate]
\reqid{A} Use unit tests for all new code. \FIXME[enumerate]
\end{description}

\FIXME[Refactor time average of data]

\subsection{Software-Testable Requirements}
\label{sec:regress}

It is anticipated that each of these tests will be captured by a
script that may be executed on a regular basis, possibly requiring
user intervention.  For example, when run in an interactive mode,
\ac{fourfit} displays fringe plots which require a human to acknowledge
that they have in fact been displayed.  (One can test the generation of
the plot without a human, but the display mechanics require visual
inspection.)

\begin{description}
\reqid{S} \ac{ghostscript} copypage test
\reqid{S} \ac{aedit} experiment summary plot
\reqid{S} \ac{aedit} quantity with time display plot
\reqid{S} \FIXME[more]
\end{description}

It is anticipated that for these a mechanism shall be provided
so that they human need merely acknowledge success, failure or
make a note of issues.

\subsection{Procedure-Testable Requirements}
\label{sec:procedure}

This section includes items that may be directly verified in a quasi-automated
fashion which may involve a human.  That is, there will be a procedure for the
human to execute with a documented report to be generated.

\begin{description}
\reqid{P} It should be possible to make 2D plots of the type
currently possible within \texttt{\ac{aedit}} of \ac{HOPS}.
\reqid{P} \FIXME[more]


\reqid[X] mk4 data type should be changed to little endian or a more
    optimal algorithm should be used to convert it to little endian
    when processing it.
\reqid[X] Refactor the Single Band Delay \ac{SBD}
    and Multi Band Delay \ac{MBD} algorithm. Make it modular.
\reqid[X] Implement an algorithm using a \ac{LSF} % least squares fitting
    technique for the global fringe fitter.
\reqid[X] Support mmVLBI work with spectral line sources (narrow sources).
\reqid[X] Pulsar folding functionality.
\reqid[X] Refactor alist, aedit, and fourfit to be compatible with \ac{HOPS}.

\end{description}

\section{General (not directly testable) Desires}
\label{sec:desires}

This section covers items that are verifiable by inspection (or analysis
or discussion).  It is subdivided into a section on goals of the current
project which we expect to reach by the end of the project, and a section
on future-proofing.

\subsection{Current Desires}
\label{sec:currentdesires}

\begin{description}
\reqid{D} Everything currently possible in \ac{HOPS} should remain
possible in \ac{MHO}.
\reqid{D} \ac{MHO} should be easier to use than \ac{HOPS}.

\reqid{X} Python wrapper for the mk4 data type.
\reqid{X} Expand the functionality of alist to allow for more data
    analysis/visualization options.
    % those are both commercial(Like Looker or Tableau?)
\reqid{X} Replace the existing fourfit commandline control file with
    a Python commandline control file that has the same exact functionality.
\reqid{X} Support VEX 2.0 and make it compatible with the VEX2XML parser tool.
\reqid{X} Rename the mk4 data type number series.
\reqid{X} Benchmarking and performance analysis should be augmenteded
    with a more sophisticated tools.
\reqid{X} Data import/export to/from ASCII. (Export to or import data
    from a human readible format.)
\reqid{X} Refactor how alist files are handled to make one line
    summaries of every fringe plot but do it in Python.
\reqid{X} Data visualization tools in Python.
\reqid{X} Refactor all existing algorithms in to new Python libraries
    that run C or C++ code under the hood.
\end{description}

\subsection{Future-Proofing}
\label{sec:future}

For every required package (beyond the historically, generally
available \ac{GNU/Linux} packages) the code should be structured so
that it is relatively straightforwards to port to a new package with
similar functionality.

\begin{description}
\reqid{F} The \ac{FFT} calls should be wrapped so that packages other than
    \ac{FFTW3} can be used.
\end{description}

%
% eof
%
