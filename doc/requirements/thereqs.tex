%
% the meat of the thing
%
\newcounter{req}
\section{Testable Requirements}
\label{sec:testable}

This section enumerates specific requirements that we plan to meet by test.
These are partitioned into subsections by automated test
(\Sec~\ref{sec:auto}, \ie~these that that may be executed
in a nightly build), software tests that may be run at specific
juctions (\Sec~\ref{sec:regress}, \ie~regression tests),
and tests that are procedural, requiring human interaction
(\Sec~\ref{sec:procedure}, \ie~\ac{GUI} tests).%
\geomargin{In addition to these requirements, there are geodetic
requirements which are captured outside the ngEHT effort.}

\ac{HOPS} is currently in the version 3.x series; the new code will
begin with the 4.x series, and to distinguish the two flavors, we
shall use \ac{MHO} for the latter.

The requirements in this document are numbered with a prefix
according to the type of test that may be employed.

\subsection{Automated Testable Requirements}
\label{sec:auto}

It is anticipated that each of these tests will be captured by a
``check'' script (as currently supported in the autotools).

\begin{description}
\item[A.\thereq] Every data type written to disk in \ac{MHO} shall be 
amenable to human-comprehensible examination as with \texttt{\ac{CorAsc2}}
\end{description}

\subsection{Software-Testable Requirements}
\label{sec:regress}

It is anticipated that each of these tests will be captured by a
script that may be executed periodically.

\begin{description}
\item[S.\thereq] foo.
\end{description}

It is anticipated that each of these tests will have a procedure
for someone to follow to verify the capability.

\subsection{Procedure-Testable Requirements}
\label{sec:procedure}

\begin{description}
\item[P.\thereq] It should be possible to make 2D plots of the type
currently possible within \texttt{\ac{aedit}} of \ac{HOPS}.
\end{description}

\section{General (not directly testable) Desires}
\label{sec:desires}

\begin{description}
\item[D.\thereq] Everything currently possible in \ac{HOPS} should remain
possible in \ac{MHO}.
\item[D.\thereq] \ac{MHO} should be easier to use than \ac{HOPS}.
\end{description}

%
% eof
%
