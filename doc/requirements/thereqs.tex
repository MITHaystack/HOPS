%
% the meat of the thing
%
\newcounter{req}
\section{Testable Requirements}
\label{sec:testable}

This section enumerates specific requirements that we plan to meet by test.
These are partitioned into subsections by automated test
(\Sec~\ref{sec:auto}, \ie~these that that may be executed
in a nightly build), software tests that may be run at specific
juctions (\Sec~\ref{sec:regress}, \ie~regression tests),
and tests that are procedural, requiring human interaction
(\Sec~\ref{sec:procedure}, \ie~\ac{GUI} tests).%
\geomargin{In addition to these requirements, there are geodetic
requirements which are captured outside the ngEHT effort.}

\ac{HOPS} is currently in the version 3.x series; the new code will
begin with the 4.x series, and to distinguish the two flavors, we
shall use \ac{MHO} for the latter.

The requirements in this document are numbered with a prefix
according to the type of test that may be employed.

\subsection{Automated Testable Requirements}
\label{sec:auto}

It is anticipated that each of these tests will be captured by a
``check'' script (as currently supported in the autotools).

\begin{description}
\item[A.\thereq] Every data type written to disk in \ac{MHO} shall be 
amenable to human-comprehensible examination as with \texttt{\ac{CorAsc2}}
\end{description}
\begin{itemize}
\item Implement regression tests.
\item Use unit tests for all new code.
\end{itemize}

\subsection{Software-Testable Requirements}
\label{sec:regress}

It is anticipated that each of these tests will be captured by a
script that may be executed periodically.

\begin{description}
\item[S.\thereq] foo.
\end{description}

It is anticipated that each of these tests will have a procedure
for someone to follow to verify the capability.

\begin{itemize}
\item Refactor time average of data.
\end{itemize}

\subsection{Procedure-Testable Requirements}
\label{sec:procedure}

\begin{description}
\item[P.\thereq] It should be possible to make 2D plots of the type
currently possible within \texttt{\ac{aedit}} of \ac{HOPS}.
\end{description}

\begin{itemize}
\item mk4 data type should be changed to little endian or a more optimal algorithm should be used to convert it to little endian when processing it.
\item Refactor the Single Bandf Delay (SBD) and Multi Band Delay (MBD) algorithm. Make it modular.
\item Implement an algorithm using the least squares fitting technique for the global fringe fitter.
\item Support mmVLBI work with spectral line sources (narrow sources).
\item Pulsar folding functionality.
\item Refactor alist, aedit, and fourfit to be compatible with \ac{HOPS}.
\end{itemize}

\section{General (not directly testable) Desires}
\label{sec:desires}

\begin{description}
\item[D.\thereq] Everything currently possible in \ac{HOPS} should remain
possible in \ac{MHO}.
\item[D.\thereq] \ac{MHO} should be easier to use than \ac{HOPS}.

\begin{itemize}
\item Python wrapper for the mk4 data type.
\item Expand the functionality of alist to allow for more data analysis/visualization options. (Like Looker or Tableau?)
\item Replace the existing fourfit commandline control file with a Python commandline control file that has the same exact functionality.
\item Support VEX 2.0 and make it compatible with the VEX2XML parser tool.
\item Rename the mk4 data type number series.
\item Benchmarking and performance analysis should be augmenteded with a more sophisticated tools.
\item Data import/export to/from ASCII. (Export to or import data froma human readible format.)
\item Refactor how alist files are handled to make one line summaries of every fringe plot but do it in Python.
\item Data visualization tools in Python. %Duplicate?
\item Refactor all existing algorithms in to new Python libraries that run C or C++ code under the hood.
\end{itemize}

\end{description}

%
% eof
%
