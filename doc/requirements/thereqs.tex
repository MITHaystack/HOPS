%
% the meat of the thing
%
% a counter for consecutive numbering and \reqid are defined in shortcuts.
%

\section{Testable Requirements}
\label{sec:testable}

This section enumerates specific requirements that we plan to meet by test.
Note that there are many untestable ``desires'' or ``goals''; these are
discussed in then next section (\Sec~\ref{sec:desires}).  Consequently,
potential tests that address such things do not appear in this section.
These issues are more fully addressed in the design document (\cite{design}).

The itemization and description of thest tests are partitioned into
subsections according to how they are likely to be executed.
The first section deals with short automated tests
(\Sec~\ref{sec:auto}, \ie~these that that may be executed
in a nightly build), software tests that may be run at specific
juctions (\Sec~\ref{sec:regress}, \ie~regression tests),
and tests that are procedural, requiring human interaction
(\Sec~\ref{sec:procedure}, \ie~\ac{GUI} tests).%
\geomargin{In addition to these requirements, there are a few additional
geodetic requirements which are captured outside the ngEHT effort.}
Details of the testing process are to be found in the coverage
and testing document (\cite{cover}).

\acs{HOPS} is currently in the version 3.x series; the new code will
begin with the 4.x series, and to distinguish the two flavors, (in code
and otherwise) we shall use \acs{MHO} to specifically refer to the latter.

The requirements in this document are numbered with a prefix
according to the type of test that may be employed:
A for automated tests, S for software testable, P for procedural.
The desire and goals are prefixed with D for desire and
F for future-proofing.

\subsection{Automated Testable Requirements}
\label{sec:auto}

It is anticipated that each of these tests will be captured by a
``check'' (shell or Python) script (such as currently supported in
the autotools or created to more fully ensure safe development).
Existing unit tests within \ac{HOPS} will be ported into the
new \ac{MHO} code set.

The following are essential design requirements that should also
be validated with an automated test.

\begin{description}
\reqid{A} \ac{MHO} should support up to
    \TBD~30 stations and \TBD~435 baselines.
\reqid{A} the number of \acsp{channel} may exceed 64 and should
    have no practical limit.
\end{description}

The following existing tests currently exist in \ac{HOPS}
and can be ported to pass in \ac{MHO} as well.
\FIXME[This list is not currently well-ordered].
\FIXME[The tests need to be reworded into ``requirement'' form.]
% find . -name Makefile.am -print \
% -exec sed '/TESTS[^_].*=/,/^$/!d' {} \; -print | sed 's/^/% /' |
% grep -v PROGRAM

\begin{description}
% ./vex2xml/Makefile.am
\reqid{A} the \acsu{VEX2XML} tool which convert \acs{VEX} to \acs{XML}
    should continue to work for \acs{VEX} version 1.5.
    (\texttt{chk\_v2v.sh} is a test that verifies that this tool,
    developed for \acsu{ALMA}, performs against a dozen separate
    captured \acs{VEX} files.)
% ./data/ae_testdata/Makefile.am
\reqid{A} \ac{aedit} performance consistency on versions 5 and 6
    on captured data is needed.  (\texttt{chk\_fsumm.sh} currently
    such a test with captured \ac{A-list} data.)
\reqid{A} \ac{aedit} performance consistency on version 6 and
    any version \TBD~7 should be maintained.
    \FIXME[a discussion of a post-6 version should appear in the specification]
% ./postproc/fourmer/Makefile.am
\reqid{A} the \acs{fourmer} tool relabels channels when it combines
    two sub-bands.  A test to assure this continues to be done
    properly (such as is now done by \texttt{test\_new\_chan\_id})
    should be maintained.
% ./sub/dfio/Makefile.am
\reqid{A} if \TBD~the non-compressed form of the fringe file used in
    the \acs{Mk4} file format is to be supported, we require that the
    fringe plot be the same as that offered by the compressed version.
    (\texttt{test\_compress} a test of the \ac{PostScript}
    compression used in the \ac{Mk4} fringe file format.)
\reqid{A} \FIXME[what does this test test, actually]
    (\texttt{test\_mk4fringe} a test related to the uncompress
    or compressed storage of \ac{Mk4} fringe files)
\reqid{A} \FIXME[what does this test]
    (\texttt{chk\_baselines.sh})
\reqid{A} the \acs{alist} program should provide valid \acs{A-list}
    files in versions 5, 6 and any version \TBD~7.
    (\texttt{chk\_alist.sh} currently does this for versions 5 and 6.)
\reqid{A}
    \texttt{chk\_adump.sh}
\reqid{A}
    \texttt{chk\_aedit.sh}
\reqid{A}
    \texttt{chk\_fringex.sh}
\reqid{A}
    \texttt{chk\_average.sh}
\reqid{A}
    \texttt{chk\_avefix.sh}
\reqid{A} there should be a test to verify \acs{cofit} functionality
    (\texttt{chk\_cofit.sh} currently verifies this)
\reqid{A} there should be a test to verify \acs{search} functionality
    (\texttt{chk\_search.sh} currently does a search on a captive
    data set)
\reqid{A} there should continue to be a functional \acs{fourmer} tool
    to assemble separately correlated frequency sub-bands.
    (\texttt{chk\_fourmer.sh} currently does this for a
    captive two 512-MHz bands.  A test to cover an extension
    to current \acs{EHT} 2-GHz bands should be build and maintained.)
\reqid{A}
    \texttt{chk\_frmrsrch.sh}
\reqid{A}
    \texttt{chk\_hdlinks.sh}
\reqid{A}
    \texttt{chk\_min\_weight.sh}
\reqid{A}
    \texttt{chk\_notches.sh}
\reqid{A}
    \texttt{chk\_flagging.sh}
\reqid{A}
    \texttt{chk\_ff\_dump.sh}
\end{description}

\FIXME[should have tests for other common control constructs]

In addition to the above tests, data from a variety of experiments
has been captured and can be re-fringed on demand.  Data from two of
these are packaged with the \ac{HOPS} distribution:
% ./data/ff_testdata/Makefile.am
\begin{description}
\reqid{A} \texttt{chk\_ff\_2836.sh} a canonical S--X geodetic experiment
\reqid{A} \texttt{chk\_ff\_3365.sh} an early \ac{EHT} session suitable for
    testing \ac{fourmer}
\end{description}
Both of these tests exercise the basic fringe search and reduction
that \ac{fourfit} was built to do and have been the basis of standard
regression for \ac{HOPS} and distribution validation.   Additional tests
with other captured data sets are discussed in the next section.

Some additional tests extant in \ac{HOPS} should not be needed in
\ac{MHO}; these are listed for reference in \App~\ref{sec:obsolete}.

\subsection{Software-Testable Requirements}
\label{sec:regress}

It is anticipated that each of these tests will be captured by a
script that may be executed on a regular basis, possibly requiring
user intervention.  For example, when run in an interactive mode,
\ac{fourfit} displays fringe plots which require a human to acknowledge
that they have in fact been displayed.  (One can test the generation of
the plot without a human, but the display mechanics require visual
inspection.)

\begin{description}
% ./sub/dfio/copypage/Makefile.am
\reqid{S} \texttt{fplot\_test} a test that the \ac{ghostscript} copypage
    command works as expected.  This requires a human to verify that
    \ac{ghostcript} properly display a page.
\reqid{S} \ac{aedit} experiment summary plot test.  This requires a
    human to verify that the summary is displayed, and that clicking
    on a scan-baseline-pol point produces the appropriate fringe plot.
\reqid{S} \ac{aedit} quantity with time display plot.  This requires
    a human to verify that the desired data are shown, and that it
    is possible to flag (discard) points.
\reqid{S} \FIXME[more, most likely]
\end{description}

Additional scans have been captured with fringe tests captured
in the \ac{HOPS} development system.  (That is, the tests exist,
but the data is not included in the distribution tarball.)

% ./data/ff_testdata/Makefile.am
\begin{description}
\reqid{S} \texttt{chk\_ff\_2843.sh}
\reqid{S} \texttt{chk\_ff\_3372.sh}
\reqid{S} \texttt{chk\_ff\_3413.sh}
\end{description}

Additional test data has been captured from the (very) successful
2017 \ac{EHT} observations.  Fringing of these scans has not yet
been incorporated into \ac{HOPS} testing, but shall form the basis
of regression for \ac{MHO}.  \FIXME[elaborate]

It is anticipated that for these a mechanism shall be provided
so that they human need merely acknowledge success, failure or
make a note of issues.

\subsection{Procedure-Testable Requirements}
\label{sec:procedure}

This section includes items that may be directly verified in a quasi-automated
fashion which may involve a human.  That is, there will be a procedure for the
human to execute with a documented report to be generated.

\begin{description}
\reqid{P} It should be possible to make 2D plots of the type
    currently possible within \texttt{\ac{aedit}} of \ac{HOPS}.
\reqid{P} \FIXME[more]


\end{description}

\section{General (not directly testable) Desires}
\label{sec:desires}

This section covers items that are verifiable by inspection (or analysis
or discussion).  It is subdivided into a section on goals of the current
project which we expect to reach by the end of the project, and a section
on future-proofing.

\subsection{Current Desires}
\label{sec:currentdesires}

\begin{description}
\reqid{D} Implement regression tests. \FIXME[enumerate]

\reqid{D} Use \acsu{unit} tests for all code. \FIXME[enumerate]

\reqid{D} Unit tests should \acsu{cover} all code. \FIXME[enumerate]

\reqid{D} Everything currently possible in \ac{HOPS} should remain
    possible in \ac{MHO}.

\reqid{D} \ac{MHO} should be easier to use than \ac{HOPS}.

\reqid{D} Mk4 data type should be changed to little endian or a more
    optimal algorithm should be used to convert it to little endian
    when processing it.

\reqid{D} Refactor the Single Band Delay \ac{SBD},
    Multi Band Delay \ac{MBD} and Delay Rate \ac{DR} algorithm to
    make it more modular (and amenable to variations on the method).

\reqid{D} Python wrapper for the Mk4 data type.

\reqid{D} Expand the functionality of alist to allow for more data
    analysis/visualization options.
    % those are both commercial(Like Looker or Tableau?)
\reqid{D} Implement data visualization tools in Python.

\reqid{D} Replace the existing fourfit commandline control file with
    a Python commandline control file that has the same exact functionality.

\reqid{D} Support VEX 2.0 and make it compatible with the VEX2XML parser tool.

\reqid{D} Rename the Mk4 data type number series.

\reqid{D} Benchmarking and performance analysis should be augmenteded
    with a more sophisticated tools.

\reqid{D} Data import/export to/from ASCII. (Export to or import data
    from a human readible format.)

\reqid{D} Refactor how alist files are handled to make one line
    summaries of every fringe plot but do it in Python.

\reqid{D} Refactor alist, aedit, and fourfit to be compatible with \ac{HOPS}.

\reqid{D} Refactor all existing algorithms in to new Python libraries
    that run C or C++ code under the hood.

\reqid{D} Implement an algorithm using a \ac{LSF} % least squares fitting
    technique for the global fringe fitter.
\reqid{D} Support mmVLBI work with spectral line sources (narrow sources).
\reqid{D} Pulsar folding functionality.

\reqid{D} The \ac{average} command averages collections of \ac{A-list} data
    (in text or binary forms) generated with \ac{fringex} and output as
    new \ac{A-list} files.  This capability needs to be preserved, but
    also refactored to become less cumbersome to use.  (It is required
    by the current \ac{search} and \ac{cofit} commands.)

\reqid{D} Every data type written to disk in \ac{MHO} shall be 
    amenable to human-comprehensible examination as is done by the
    \texttt{\ac{CorAsc2}} program.

\end{description}

\subsection{Future-Proofing}
\label{sec:future}

For every required package (beyond the historically, generally
available \ac{GNU/Linux} packages) the code should be structured so
that it is relatively straightforwards to port to a new package with
similar functionality.

\begin{description}
\reqid{F} The \ac{FFT} calls should be wrapped so that packages other than
    \ac{FFTW3} can be used.
\reqid{F} The \ac{PGPLOT} package has long been obsolete, and any replacement
    package could follow the same fate.  Thus all plotting package calls
    need to be managed with an interface that allows porting to equivalent
    packages.
\end{description}

%
% eof
%
