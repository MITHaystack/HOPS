%
% introductory thoughts
%

\section{Introduction}
\label{sec:intro}

The \ac{EHT} has launched a \ac{MSRI} project which looks to develop
the technologies needed for a second-generation \acs{EHT}. 
A significant component of the telescope is the software needed to 
properly operate, reduce and analyze the data taken by the member telescopes.
This document defines the requirements for the refactoring of the \ac{HOPS} in 
concert with the \acs{MSRI} proposal. 

The next generation \acs{EHT} is planned to include up to 30 stations and 
435 baselines, with wider bandwidth (128 Gbps has been mentioned---meaning 
four dual-polarization, 4 GHz bands) \TBR{}.  Support for greater bit depth is also of 
interest should recording media be available.  The refactoring
of \ac{HOPS} shall support the full analysis of ngEHT data within these limits,
and will be made smarter, more automatic, more robust, and easier to use.

%Furthermore, the \acs{EHT} may elect to operate with a more frequent observing cadence. 

For the needs of the \acs{EHT} Campaigns of 2017 it was decided to augment
the existing HOPS package with some Python-based packages in order to
create a pipeline for the initial reduction of data. The new \ac{HOPS} shall
support independent analyses, using Python or other languages, through
accessible data formats (e.g. HDF5, JSON, XML, or similar), interactive tools, and 
conversion to human-readable ASCII files. The current plotting functionality,
provided by PGPLOT, shall be replaced by Python/Matplotlib.

%In this continued development of HOPS, we shall assume
%that HOPS alone must be capable of the full analysis; but we should also
%be mindful that options to move the data to \ac{AIPS} or \ac{CASA} must at some
%level exist.

The primary development language will be C/C++, with a Python scripting layer
to provide ease of use. Version control shall be provided by the MIT-hosted github.

Finally, all functionality of the current \acs{HOPS} shall be maintained, with
backward-compatibility for old data formats. There shall be extensive use of regression
tests and demonstrated reproducibility of prior results.

This document is intended to be one in a series, with other documents describing:
\begin{itemize}
\item Specifications and architectural design
\item Coverage and test plan
\item The development plan
\item The user manual
\item Requirements specific to geodesy
\end{itemize}




%This section enumerates specific requirements that we plan to meet by test.
%Note that there are many untestable ``desires'' or ``goals''; these are
%discussed in then next section (\Sec~\ref{sec:desires}).  Consequently,
%potential tests that address such things do not appear in this section.
%These issues are more fully addressed in the design document (\cite{design}).
%
%The itemization and description of thest tests are partitioned into
%subsections according to how they are likely to be executed.
%The first section deals with short automated tests
%(\Sec~\ref{sec:auto}, \ie~these that that may be executed
%in a nightly build), software tests that may be run at specific
%juctions (\Sec~\ref{sec:regress}, \ie~regression tests),
%and tests that are procedural, requiring human interaction
%(\Sec~\ref{sec:procedure}, \ie~\ac{GUI} tests).%
%\geomargin{In addition to these requirements, there are a few additional
%geodetic requirements which are captured outside the ngEHT effort.}
%Details of the testing process are to be found in the coverage
%and testing document (\cite{cover}).
%
%\acs{HOPS} is currently in the version 3.x series; the new code will
%begin with the 4.x series, and to distinguish the two flavors, (in code
%and otherwise) we shall use \acs{MHO} to specifically refer to the latter.
%
%The requirements in this document are numbered with a prefix
%according to the type of test that may be employed:
%A for automated tests, S for software testable, P for procedural.
%The desire and goals are prefixed with D for desire and
%F for future-proofing.
%
%\subsection{General Requirements}
%\label{sec:generalreq}
%
%It is anticipated that each of these tests will be captured by a
%``check'' (shell or Python) script (such as currently supported in
%the autotools or created to more fully ensure safe development).
%Existing unit tests within \ac{HOPS} will be ported into the
%new \ac{MHO} code set.



%
% eof
%
