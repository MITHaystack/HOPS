\documentclass[notitlepage,letterpaper,pdftex,12pt,final]{article}
%\documentclass[notitlepage,letterpaper,pdftex,12pt,draft]{article}
% article < report < book ; preable material follows
%
% This file is mostly boilerplate that you can copy and tweak.
% The true material is in included files.  ../common should hold
% things that more than one document might need.  The Makefiles
% should likewise be mostly boilerplate with minor tweaks.
%

% (un)comment to (see)hide full paths to files actually used
% however, you'll need to override the TEXOPT= setting to batchmode
% made in the Makefile.
\listfiles

\addtolength\textwidth{1.7in}
\addtolength\oddsidemargin{-0.95in}
\addtolength\evensidemargin{-0.95in}
\addtolength\marginparwidth{-0.95in}
\typeout{textwidth is \the\textwidth}

% useful to block out portions of text
\usepackage{ifthen}
% to allow defining our own colors
\usepackage[dvipsnames]{xcolor}
% makes hyperlinks work
\usepackage{hyperref}
\hypersetup{colorlinks=true,linkcolor=darkblue,citecolor=darkgreen}
% for figures; insert [draft] before {...} not to see imgs
\usepackage{graphicx}
\graphicspath{ {../common/uml-diagrams} }
% for small figures surrounded by text
\usepackage{wrapfig}
% for control over headers and footers
\usepackage{fancyhdr}
\pagestyle{fancy}
\fancyhf{}
% for handling acronyms
\usepackage[printonlyused]{acronym}

% for bibliographies if needed
\usepackage[utf8]{inputenc}
\usepackage[english]{babel}
\usepackage[backend=bibtex,style=numeric,sorting=none]{biblatex}
\addbibresource{hops.bib}
%\usepackage{natbib}
% Use the appendix package.
\usepackage{appendix}
% context-sensitive quotes
\usepackage{csquotes}
% underline for emphasis: \uline etc
\usepackage{ulem}
% for generating outlines
\usepackage{outlines}
\usepackage{enumitem}

% for bold math
\usepackage{bm}
% for serious math which we may have in a few places (eventually)
\usepackage{amsmath}
% equations include section number
\numberwithin{equation}{section}

% for marginal notations concerning geodesy
\usepackage{marginnote}

% to go back to default after \ragged commmands
\usepackage{ragged2e}

% see common/shortcuts.tex for what is defined in this file
%
% a file of commands to save typing
%

\newcommand{\eg}{\textit{e.g.}}
\newcommand{\EG}{\textit{E.g.}}
\newcommand{\ie}{\textit{i.e.}}
\newcommand{\IE}{\textit{I.e.}}
\newcommand{\etc}{\textit{\&c.}}
\newcommand{\Sec}{Section}
\newcommand{\Fig}{Figure}
\newcommand{\Tab}{Table}
\newcommand{\App}{Appendix}
\newcommand{\FIX}[1][fixme]{{\color{red}#1}}
\newcommand{\TBC}{{\color{red}TBC}}
\newcommand{\TBD}{{\color{red}TBD}}
\newcommand{\TBR}{{\color{red}TBR}}
\newcommand{\FIXME}[1][]{{\color{red}FIXME -- #1}}

\newcommand{\MHO}{MIT-HOPS}
\newcommand{\HOPS}{HOPS}

%
% eof
%

\setboolean{geos}{false}% or false to hide geodetic marginal notations

% control options
\newboolean{skipappendix}
\setboolean{skipappendix}{false}

% at some point this gets frozen
\newcommand{\recdate}{\today}

\begin{document}
\DeclareGraphicsExtensions{.png, .jpg, .pdf}
% best to put figures in subdirs
%\graphicspath{{figs/}{scans/}}

% for subsequent pages
\setlength\headheight{15pt}
\fancyhead[L]{HOPS}
\fancyhead[C]{}
\fancyhead[R]{Requirements}
\fancyfoot[R]{Page \thepage\ of\ \pageref{page:LastPage}}
\fancyfoot[L]{\recdate}

\title{ngEHT Requirements for a new HOPS}

\author{%
\LARGE John Barrett, Geoff Crew, Dan Hoak and Violet Pfeiffer \\
\Large MIT Haystack Observatory}
\date{Version 1.0, \recdate}
\maketitle
\normalsize

\renewcommand\abstractname{\large Executive Summary}
\abstract{
This document spells out the requirements for the planned re-work of HOPS.
With a defined set of requirements (\Sec~\ref{sec:testable}),
we can ensure that the new architecture
and implementation supports the needs.  And by the conclusion of the project
a collection of test reports matching results to requirements may be presented.

Note that there is an implicit desire to end up with a suite of software that
not only meets the stated needs, but which also is easier to maintain, and more
extensible for the future.  While we might state these desires as requirements,
it is not possible to test against them, and thus these do not formally appear 
in this document, and are discussed elsewhere.
}

% N.B.: NASA document describing preferred language for enumerating requirements:
% https://www.nasa.gov/seh/appendix-c-how-to-write-a-good-requirement


% \begingroup . . . \endgroup can be used to keep things together
% and/or an explicit page break and/or adjust spacing as needed so
% it looks presentable.  Uncomment the sections you need.
%\begingroup
%\renewcommand\contentsname{Contents}
%\renewcommand\listfigurename{Figures}
%\renewcommand\listtablename{Tables}
%\vspace{24pt}\hrule

\pagebreak
\tableofcontents

%\vspace{24pt}\hrule
%\pagebreak
%\listoffigures
%\vspace{24pt}\hrule
%\listoftables
%\vspace{24pt}\hrule
%\endgroup

% it is sometimes cleaner to start sections on new pages in a longer
% document--then changes to each section don't repage everything.

% break document into appropriate portions

\pagebreak
%
% introductory thoughts
%

\section{Introduction}
\label{sec:intro}

The \ac{EHT} has launched a \ac{MSRI} project which looks to develop
the technologies needed for a second-generation \acs{EHT}. 
A significant component of the telescope is the software needed to 
properly operate, reduce and analyze the data taken by the member telescopes.
This document defines the requirements for the refactoring of the \ac{HOPS} in 
concert with the \acs{MSRI} proposal. 

The next generation \acs{EHT} is planned to include up to 30 stations and 
435 baselines, with wider bandwidth (128 Gbps has been mentioned---meaning 
four dual-polarization, 4 GHz bands) \TBR{}.  Support for greater bit depth is also of 
interest should recording media be available.  The refactoring
of \ac{HOPS} shall support the full analysis of ngEHT data within these limits,
and will be made smarter, more automatic, more robust, and easier to use.

%Furthermore, the \acs{EHT} may elect to operate with a more frequent observing cadence. 

For the needs of the \acs{EHT} Campaigns of 2017 it was decided to augment
the existing HOPS package with some Python-based packages in order to
create a pipeline for the initial reduction of data. The new \ac{HOPS} shall
support independent analyses, using Python or other languages, through
accessible data formats (e.g. HDF5, JSON, XML, or similar), interactive tools, and 
conversion to human-readable ASCII files. The current plotting functionality,
provided by PGPLOT, shall be replaced by Python/Matplotlib.

%In this continued development of HOPS, we shall assume
%that HOPS alone must be capable of the full analysis; but we should also
%be mindful that options to move the data to \ac{AIPS} or \ac{CASA} must at some
%level exist.

The primary development language will be C/C++, with a Python scripting layer
to provide ease of use. Version control shall be provided by the MIT-hosted github.

Finally, all functionality of the current \acs{HOPS} shall be maintained, with
backward-compatibility for old data formats. There shall be extensive use of regression
tests and demonstrated reproducibility of prior results.

This document is intended to be one in a series, with other documents describing:
\begin{itemize}
\item Specifications and architectural design
\item Coverage and test plan
\item The development plan
\item The user manual
\item Requirements specific to geodesy
\end{itemize}




%This section enumerates specific requirements that we plan to meet by test.
%Note that there are many untestable ``desires'' or ``goals''; these are
%discussed in then next section (\Sec~\ref{sec:desires}).  Consequently,
%potential tests that address such things do not appear in this section.
%These issues are more fully addressed in the design document (\cite{design}).
%
%The itemization and description of thest tests are partitioned into
%subsections according to how they are likely to be executed.
%The first section deals with short automated tests
%(\Sec~\ref{sec:auto}, \ie~these that that may be executed
%in a nightly build), software tests that may be run at specific
%juctions (\Sec~\ref{sec:regress}, \ie~regression tests),
%and tests that are procedural, requiring human interaction
%(\Sec~\ref{sec:procedure}, \ie~\ac{GUI} tests).%
%\geomargin{In addition to these requirements, there are a few additional
%geodetic requirements which are captured outside the ngEHT effort.}
%Details of the testing process are to be found in the coverage
%and testing document (\cite{cover}).
%
%\acs{HOPS} is currently in the version 3.x series; the new code will
%begin with the 4.x series, and to distinguish the two flavors, (in code
%and otherwise) we shall use \acs{MHO} to specifically refer to the latter.
%
%The requirements in this document are numbered with a prefix
%according to the type of test that may be employed:
%A for automated tests, S for software testable, P for procedural.
%The desire and goals are prefixed with D for desire and
%F for future-proofing.
%
%\subsection{General Requirements}
%\label{sec:generalreq}
%
%It is anticipated that each of these tests will be captured by a
%``check'' (shell or Python) script (such as currently supported in
%the autotools or created to more fully ensure safe development).
%Existing unit tests within \ac{HOPS} will be ported into the
%new \ac{MHO} code set.



%
% eof
%


\pagebreak
%
% the meat of the thing
%
% this is a counter for consecutive numbering
\newcounter{req}
\newcommand{\reqid}[1]{\item[\refstepcounter{req}#1-\thereq]}

\section{Testable Requirements}
\label{sec:testable}

This section enumerates specific requirements that we plan to meet by test.
These are partitioned into subsections by automated test
(\Sec~\ref{sec:auto}, \ie~these that that may be executed
in a nightly build), software tests that may be run at specific
juctions (\Sec~\ref{sec:regress}, \ie~regression tests),
and tests that are procedural, requiring human interaction
(\Sec~\ref{sec:procedure}, \ie~\ac{GUI} tests).%
\geomargin{In addition to these requirements, there are a few additional
geodetic requirements which are captured outside the ngEHT effort.}

\ac{HOPS} is currently in the version 3.x series; the new code will
begin with the 4.x series, and to distinguish the two flavors, (in code
or otherwise) we shall use \ac{MHO} for the latter.

The requirements in this document are numbered with a prefix
according to the type of test that may be employed.

\subsection{Automated Testable Requirements}
\label{sec:auto}

It is anticipated that each of these tests will be captured by a
``check'' (shell or Python) script (such as currently supported in
the autotools or created to more fully ensure safe development).

\begin{description}
\reqid{A} Every data type written to disk in \ac{MHO} shall be 
amenable to human-comprehensible examination as is done by the
\texttt{\ac{CorAsc2}} program.
\reqid{A} Implement regression tests. \FIXME[enumerate]
\reqid{A} Use unit tests for all new code. \FIXME[enumerate]
\end{description}

\FIXME[Refactor time average of data]

\subsection{Software-Testable Requirements}
\label{sec:regress}

It is anticipated that each of these tests will be captured by a
script that may be executed on a regular basis, possibly requiring
user intervention.  For example, when run in an interactive mode,
\ac{fourfit} displays fringe plots which require a human to acknowledge
that they have in fact been displayed.  (One can test the generation of
the plot without a human, but the display mechanics require visual
inspection.)

\begin{description}
\reqid{S} \ac{ghostscript} copypage test
\reqid{S} \ac{aedit} experiment summary plot
\reqid{S} \ac{aedit} quantity with time display plot
\reqid{S} \FIXME[more]
\end{description}

It is anticipated that for these a mechanism shall be provided
so that they human need merely acknowledge success, failure or
make a note of issues.

\subsection{Procedure-Testable Requirements}
\label{sec:procedure}

This section includes items that may be directly verified in a quasi-automated
fashion which may involve a human.  That is, there will be a procedure for the
human to execute with a documented report to be generated.

\begin{description}
\reqid{P} It should be possible to make 2D plots of the type
currently possible within \texttt{\ac{aedit}} of \ac{HOPS}.
\reqid{P} \FIXME[more]


\reqid[X] mk4 data type should be changed to little endian or a more
    optimal algorithm should be used to convert it to little endian
    when processing it.
\reqid[X] Refactor the Single Band Delay \ac{SBD}
    and Multi Band Delay \ac{MBD} algorithm. Make it modular.
\reqid[X] Implement an algorithm using a \ac{LSF} % least squares fitting
    technique for the global fringe fitter.
\reqid[X] Support mmVLBI work with spectral line sources (narrow sources).
\reqid[X] Pulsar folding functionality.
\reqid[X] Refactor alist, aedit, and fourfit to be compatible with \ac{HOPS}.

\end{description}

\section{General (not directly testable) Desires}
\label{sec:desires}

This section covers items that are verifiable by inspection (or analysis
or discussion).  It is subdivided into a section on goals of the current
project which we expect to reach by the end of the project, and a section
on future-proofing.

\subsection{Current Desires}
\label{sec:currentdesires}

\begin{description}
\reqid{D} Everything currently possible in \ac{HOPS} should remain
possible in \ac{MHO}.
\reqid{D} \ac{MHO} should be easier to use than \ac{HOPS}.

\reqid{X} Python wrapper for the mk4 data type.
\reqid{X} Expand the functionality of alist to allow for more data
    analysis/visualization options.
    % those are both commercial(Like Looker or Tableau?)
\reqid{X} Replace the existing fourfit commandline control file with
    a Python commandline control file that has the same exact functionality.
\reqid{X} Support VEX 2.0 and make it compatible with the VEX2XML parser tool.
\reqid{X} Rename the mk4 data type number series.
\reqid{X} Benchmarking and performance analysis should be augmenteded
    with a more sophisticated tools.
\reqid{X} Data import/export to/from ASCII. (Export to or import data
    from a human readible format.)
\reqid{X} Refactor how alist files are handled to make one line
    summaries of every fringe plot but do it in Python.
\reqid{X} Data visualization tools in Python.
\reqid{X} Refactor all existing algorithms in to new Python libraries
    that run C or C++ code under the hood.
\end{description}

\subsection{Future-Proofing}
\label{sec:future}

For every required package (beyond the historically, generally
available \ac{GNU/Linux} packages) the code should be structured so
that it is relatively straightforwards to port to a new package with
similar functionality.

\begin{description}
\reqid{F} The \ac{FFT} calls should be wrapped so that packages other than
    \ac{FFTW3} can be used.
\end{description}

%
% eof
%


\pagebreak
\addtocounter{section}{1}
\renewcommand{\refname}{\thesection. References}
\addcontentsline{toc}{section}{\thesection. References}
\bibstyle{plainurl}
\printbibliography
\label{sec:references}

% if skipappendix-is-true then (nothing) else typeset Appendices
\ifthenelse{\boolean{skipappendix}}{}{%
\appendix

%\pagebreak
%%
% hilo presentation
%
\section{What was promised at HILO}
\label{sec:hilo}

This appendix captures what was presented at the Hilo
\acs{EHT} meeting (Dec. 2019) in a software session intended to
capture feedback on the plan.  The discussion
did not provide any clear guidance to significantly adjust our plans.

\newcommand{\sbitem}{\hfill\break\hspace{2mm}$\diamond$\ }

\subsection{slide 2}
\begin{itemize}
\item \ac{HOPS} performed adequately for 2017 data reduction
\sbitem consistency with \acsu{AIPS} and \acsu{CASA} was established
\sbitem the \ac{HOPS}-based pipeline was adopted for production
work\footnote{%
after validation against the alternatives using \ac{AIPS} and \ac{CASA}.
We consider it still highly valuable to update \ac{HOPS}, and indeed, it
makes sense to enable interoperability with \ac{CASA}---however, that is
outside of our scope.}
\item The \acsu{ngEHT} funding via \acsu{MSRI}-I calls for updates to HOPS:
\sbitem 4 years of development to become “shovel ready”o
\sbitem N stations more than doubles (\eg~to 20--30)
\sbitem Bandwidth quadruples at some sites (\eg~8 GHz)
\sbitem Simultaneous 230 / 345 \acsu{GHz} observing perhaps
\sbitem Anything else that gets decided in then next 4 years
\end{itemize}

\subsection{slide 3}
\begin{itemize}
\item We’ve (almost) come to the end of what can be fixed with “bandaids”
\item Maximum \# of channels is 64 and that’s hard to fix (a-zA-Z0-9\%\$)
\item No full complex bandpass corrections (only per-channel phase+delay)
\item single-baselines are both a virtue and a problem:
\sbitem \ac{HOPS} finds more fringes than \ac{AIPS} or \ac{CASA}.
\sbitem station-based phases and delays are not readily accessible
\item \ac{fourfit} is a one-shot process; multi-step processing not supported
\item User interface is a challenge:
\sbitem control file syntax is a bit arcane
\sbitem all you get is one \ac{fourfit} plot that either works or is garbage
\end{itemize}

\subsection{slide 3}
\begin{itemize}
\item \ac{HOPS} code has 30+ years of history in it
\sbitem was coded in \ac{C}, but reads like the \ac{Fortran} it was ported from
\sbitem not modular except for a few of the \ac{i/o} libraries
\sbitem was written for hardware correlators
\sbitem was written for computers that no longer exist
\sbitem little endianism won out over big endianism (apparently)
\sbitem was successfully adapted to \ac{DiFX}
(but not \eg~\ac{SFXC})
\item Plotting and results are not independently generated
\sbitem amplitude and \ac{SNR} come as side-effects of what you plotted
\sbitem \ac{PGPLOT} is maybe ok today, but not really supported anymore
\end{itemize}

\subsection{slide 4}
\begin{itemize}
\item Global fringe solutions (and station based-quantities)
\item Complex bandpass
\item A more human-friendly interface (\eg~\acs{Python} as \ac{CASA} does)
\item Allow distributed computing and/or parallelization (threads, OpenMPI)
\item Insert hooks to allow plug-in modules for customizations as needed
\item Allow a strategy for iterative calibration and fringing
\item Improved data formats (internal in-memory as well as disk storage)
\item Enable better exchange with other analysis packages:
\sbitem \ac{FITS-IDI}? (or \ac{HDF5} or whatever else comes along?),
\ac{CASA},
\ac{MS}?,\ldots
\sbitem (either enables better use of \ac{HOPS} with simulated data)
\item A more flexible/interactive plotting system
\sbitem single summary is fine when everything is working
\sbitem provide real support for investigation of problems
\end{itemize}

\subsection{slide 5}
\begin{itemize}
\item Maintain existing tools “as is” for serious regression (probably patched)
\item Arbitrary number of channels; eliminate internal magic sizing numbers
\item New control file format (\eg~use \acs{Python} or some config module)
\item New internal data formats (rationalized, new structures or objects)
\item New disk data formats: “mk4” $\rightarrow$ “hops”
\sbitem machine/compiler independent little-endian (not big-endian)
\sbitem rationalized data types (as with internal formats,
optimized for disk \ac{i/o})
\sbitem new root file format (\ac{ovex}
is ancient history, and current root is artificial)
\sbitem preserve the current \ac{m4py}-type capability
\sbitem allow translator tools to exchange with “hops”, “mk4” and other formats
\end{itemize}

\subsection{slide 6}
\begin{itemize}
\item Basically: FIX what is broken
\item Not gratuitously break the current pipelines, but allow simplification
\item Most likely to be implemented in a mix of \ac{C}, \ac{C++} or
\acs{Python}
\item Provide a more canonical adaptation to \ac{GNU/Linux} environments
\item Implement what is most important to have available in 4 years
\end{itemize}
%
% eof
%

%
%\pagebreak
%%
% an appendix for some existing, but obsolete tests.
%
\section{Obsolete HOPS tests}
\label{sec:obsolete}

This section lists some tests present in the existing HOPS which likely
do not require support going forwards.

%
% eof
%

%
%\pagebreak
%%
% a section on UML diagrams
% 
% \appendix
\section{Sample UML Diagrams}
\begin{description}
% A---------------------------------------------------------------
\item[Activity Diagram]  A UML diagram that demonstrates how users
interact with a given application, data flows, and branching options
for workflows and processes.

% C---------------------------------------------------------------
\item[Class Diagram] A UML diagram that demonstrates the various classes
in software, the relationships between those classes and how they depend
on each other.
\begin{figure}[h!]
\center{\fbox{%
\includegraphics[width=0.70\textwidth]{class-diagram-example}}}
\caption[Class Diagram Example]{
This is a sample class diagram, see text for explanation.}
\end{figure}

\item[Component Diagram] A UML diagram that demonstrates how the modular
parts of software interact with each other using APIsi, interfaces
or other methods. The modular parts or components include databases,
web services, thin clients, thick clients, functions, libraries, and
other dependencies.

% S---------------------------------------------------------------
\item[State Diagram] A UML diagram that describes the behavior and state
of the software.
\end{description}

%
% eof
%


\pagebreak
\section{Acronyms, Commands, and Glossary}
%
% \section{Acronyms, Commands and Glossary}
%
% \acro{acronym}[short name]{full name / description}
% \ac{acronym} is the usual usage in text that defines (and gives short name)
% \acs{acronym} gives just the short name
% \acf{acronym} gives just the full name
% \acsu{acronym} gives the short name and marks it used
%
% the optional short name can include math as the acronym key cannot
%
% see https://ctan.org/pkg/acronym
%
\begin{acronym}
% A---------------------------------------------------------------
\acro{A-list}{a one line description of baseline fringes used by \ac{HOPS}}
\acro{alist}{a program for creating a file of \ac{A-list} scans}
\acro{aedit}{a program for editing a file of \ac{A-list} scans}
\acro{AIPS}{Astronomical Image Processing System}
\acro{AMP}{short for ``amplitude'' the correlation coefficient}
% C---------------------------------------------------------------
\acro{C}{The ``C'' programming language, created to make \ac{UNIX} portable}
\acro{C++}{The C++ programming language, an object-oriented
    successor to \ac{C}}
\acro{CASA}{Common Astronomy Software Applications package}
\acro{channel}{an ambigous term which refers either to a spectral channel,
\ie~frequency point of an \acsu{FFT} or to a sub-band of a larger receiver
band.}
\acro{cofit}{a \ac{HOPS} tool to assess atmospheric coherence in terms of
    \ac{SNR} and \ac{AMP} variation with integration interval}
\acro{CorAsc2}{Correlator to Ascii (2nd version)}
\acro{cover}{a coverage test exercises all logic branches of some code module}
% D---------------------------------------------------------------
\acro{DFT}{Discrete Fourier Transform}
\acro{DR}{Delay Rate, the fringe parameter concerning the change of delay with time}
\acro{DiFX}{the ``distributed'' \ac{FX} correlator}
\acro{difx2mark4}{a program (part of \ac{DiFX}) to convert \ac{SWIN}
format correlation products into the ``Mark4'' data files used by \ac{HOPS}}
% E---------------------------------------------------------------
\acro{EHT}{the Event Horizon Telescope, which usually refers to the
    observing array}
\acro{EHTC}{the Event Horizon Telescope Collaboration, which usually
refers to the organization that operates the \ac{EHT}}
% F---------------------------------------------------------------
\acro{FFT}{Fast Fourier Transform}
\acro{FFTW3}{Fastest Fourier Transform in the West, version 3}
\acro{Fortran}{a FORmula TRANslation language, in common use prior to \ac{C}}
\acro{FX}{a general term for correlation that does the cross-correlation
    after first transforming to frequency space}
\acro{FITS}{Flexible Image Transport System, now referring to a
    general digital data format}
\acro{FITS-IDI}{A dialect of \ac{FITS}
    designed for the interchange of data for interferometry}
\acro{fourfit}{the main fringe-finding command in \ac{HOPS}}
\acro{fringex}{an \ac{HOPS} tool to explore the fringe} 
\acro{fourmer}{a program that combines data from two sub-bands into
    a larger common band}
% G---------------------------------------------------------------
\acro{ghostscript}{Ghostscript, the GNU \acs{PostScript} emulator}
\acro{GHz}{one billion Hz}
\acro{GNU}{GNU is Not Unix (a software project launched by
    Richard Stallman in the 80's)}
\acro{GNU/Linux}{a family of operating systems using Linus' kernel and
    GNU's software packages}
\acro{GS}{short for \ac{ghostscript}}
\acro{GUI}{Graphical User Interface}
% H---------------------------------------------------------------
\acro{HDF5}{Hierarchical Data Format, version 5.}\footnote{Why would you
\emph{want} to use anything that took 5 versions to get right?}
\acro{HOPS}{Haystack Observatory Postprocessing System}
\acro{Hz}{A frequency unit named for Heinrich Hertz.
    A frequency of one Hz is one oscillation per second.}
% I---------------------------------------------------------------
\acro{i/o}{short for input/output referring to the fact that programs are
    written to act on something and provide something}
\acro{IPP}{Intel Performance Primitives is a library of functionality
optimized for use with the Intel processor family}
% J---------------------------------------------------------------
\acro{JIVE}{now just a name for an organization, it is still an
    Institution for VLBI in Europe, just not a Joint one}
% L---------------------------------------------------------------
\acro{Linux}{a family of operating systems built around Linus Torval's version
    of the UNIX kernel}
\acro{LSF}{Least Squares Fit}
% M---------------------------------------------------------------
\acro{MBD}{Multi-Band Delay, the delay parameter referring to the change of
    phase with frequency in a multi-channel (sub-band) system.}
\acro{Mk4}{The fourth in a series of \ac{VLBI} hardware correlators.  The
    Mark4 replaced the Mark3 near the beginning of the millenium, and was
    finally put to rest by \ac{DiFX} in the mid 2010's}
\acro{m4py}{a shallow \acs{Python} wrapper which provides access to
    \acs{mk4} data files and types}
\acro{MS}{Measurement Set, a formal specification for data to be analyzed
    with reference to a Measurement Equation}
\acro{MSRI}{Mid-scale Research Initiatives program at the \ac{NSF}}
% N---------------------------------------------------------------
\acro{NSF}{National Science Foundation}
\acro{ngEHT}{next-generation \ac{EHT}}
% O---------------------------------------------------------------
\acro{ovex}{an ``observer'' dialect of \acs{VEX}}
\acro{OpenMPI}{Open MPI Project is an open source Message Passing Interface implementation}
% P---------------------------------------------------------------
\acro{PDF}{Portable Document Format (developed by Adobe) as a successor
    to \ac{PostScript}}
\acro{PGPLOT}{a ``pretty good'' plotting package developed and maintained
    by Tim Pearson at Caltech.  He's retired now, so it is stuck at verion
    5.2.2, (released Feb 2001)}
\acro{PostScript}{a printer page description language developed by Adobe.
    \ac{fourfit} plots are currently generated in \ac{PostScript} and
    often converted to \acs{PDF}}
\acro{PS}{short for \ac{PostScript}}
\acro{Python}{a programming language named in honor of Monty Python's Flying
    Circus}
% S---------------------------------------------------------------
\acro{SBD}{Single Band Delay, the delay parameter referring to the time
    offset between two signals being correlated}
\acro{SFXC}{\acs{JIVE}'s software \ac{FX}-kind correlator}
\acro{SNR}{Signal to Noise Ratio}
\acro{SWIN}{the output format used by the \ac{DiFX} correlator}
\acro{MHO}{MIT Haystack Observatory Postprocessing System}
% U---------------------------------------------------------------
\acro{unit}{a unit test is a short test used to validate a small part of
    some larger code module}
\acro{UNIX}{the name of a family of operating systems
    (born in the 70's at Bell Laboratories)}
% V---------------------------------------------------------------
\acro{VEX}{\acs{VLBI} EXperiment (file), a means of fully describing
    a planned \acs{VLBI} experiment or observation}
\acro{VEX2XML}{a program that converts \acs{VEX} files into an easily
    parsed \acs{XML} represention}
\acro{VGOS}{\acs{VLBI} Global Observing System;
    was called \acs{VLBI}2010 until the mid 2010's}
\acro{VLBI}{Very Long Baseline Interferometry}
% W---------------------------------------------------------------
\acro{Whitneys}{correlation amplitudes are normally expressed between 0 and 1,
    but in our work they are usually small and in \ac{HOPS} traditionally
    multiplied by ten thousand, in which case, the unit of correlation amplitude
    is ``Whitneys'' after Alan Whitney who may be commended or blamed for the
    usage.}
% X---------------------------------------------------------------
\acro{XF}{a general term for correlation that does the cross-correlation
    first, and then transforms the result to frequency space}
\acro{XML}{eXtensible Markup Language}
% Z---------------------------------------------------------------
\acro{zero-pad}{the practice of extending time or frequency sequences with
    some number of zeroes which, for \ac{FFT}s has the effect of smoothing
    in the other domain}
\end{acronym} 
%
% eof
%

}

\label{page:LastPage}
\end{document}
%
% eof
%
