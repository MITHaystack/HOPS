\documentclass[notitlepage,letterpaper,pdftex,12pt,final]{article}
%\documentclass[notitlepage,letterpaper,pdftex,12pt,draft]{article}
% article < report < book ; preable material follows
%
% This file is mostly boilerplate that you can copy and tweak.
% The true material is in included files.  ../common should hold
% things that more than one document might need.  The Makefiles
% should likewise be mostly boilerplate with minor tweaks.
%

% (un)comment to (see)hide full paths to files actually used
% however, you'll need to override the TEXOPT= setting to batchmode
% made in the Makefile.
\listfiles

\addtolength\textwidth{1.7in}
\addtolength\oddsidemargin{-0.95in}
\addtolength\evensidemargin{-0.95in}
\addtolength\marginparwidth{-0.95in}
\typeout{textwidth is \the\textwidth}

% useful to block out portions of text
\usepackage{ifthen}
% to allow defining our own colors
\usepackage[dvipsnames]{xcolor}
% makes hyperlinks work
\usepackage{hyperref}
\hypersetup{colorlinks=true,linkcolor=darkblue,citecolor=darkgreen}
% for figures; insert [draft] before {...} not to see imgs
\usepackage{graphicx}
% for small figures surrounded by text
\usepackage{wrapfig}
% for control over headers and footers
\usepackage{fancyhdr}
\pagestyle{fancy}
\fancyhf{}
% for handling acronyms
\usepackage[printonlyused]{acronym}

% for bibliographies if needed
\usepackage[utf8]{inputenc}
\usepackage[english]{babel}
\usepackage[backend=biber,style=numeric,sorting=none]{biblatex}
\addbibresource{hops.bib}
% context-sensitive quotes
\usepackage{csquotes}
% underline for emphasis: \uline etc
\usepackage{ulem}
% for generating outlines
\usepackage{outlines}
\usepackage{enumitem}

% for bold math
\usepackage{bm}
% for serious math which we may have in a few places (eventually)
\usepackage{amsmath}
% equations include section number
\numberwithin{equation}{section}

% for marginal notations concerning geodesy
\usepackage{marginnote}

% to go back to default after \ragged commmands
\usepackage{ragged2e}

% see common/shortcuts.tex for what is defined in this file
%
% a file of commands to save typing
%

% commonly used things
\newcommand{\eg}{\textit{e.g.}}
\newcommand{\EG}{\textit{E.g.}}
\newcommand{\ie}{\textit{i.e.}}
\newcommand{\IE}{\textit{I.e.}}
\newcommand{\etc}{\textit{\&c.}}
\newcommand{\Sec}{Section}
\newcommand{\Fig}{Figure}
\newcommand{\Tab}{Table}
\newcommand{\App}{Appendix}

% for a work in progress:
\newcommand{\FIX}[1][fixme]{{\color{red}#1}}
\newcommand{\TBC}{{\color{red}TBC}}
\newcommand{\TBD}{{\color{red}TBD}}
\newcommand{\TBR}{{\color{red}TBR}}
\newcommand{\FIXME}[1][]{{\color{red}FIXME -- #1}}

% specific to this project
\newcommand{\MHO}{MIT-HOPS}
\newcommand{\HOPS}{HOPS}

% standard colors
\definecolor{darkblue}{rgb}{0,0,.5}
\definecolor{darkgreen}{rgb}{0,.5,0}
\definecolor{darkred}{rgb}{.5,0,0}
\definecolor{violet}{rgb}{.4,0.,.7}

% for coping with geodetic things
\newboolean{geos}
\setboolean{geos}{true}% or false to hide geodetic marginal notations
%
\definecolor{boxfrcolor}{rgb}{.50,.20,.00}
\definecolor{boxbgcolor}{rgb}{0.9,.98,.98}
\definecolor{boxfgcolor}{rgb}{0.4,0.0,0.0}
% \geomargin{footnote comment}
% \geomargin[{\color{..}box note}]{footnote comment}
\newcommand{\geobox}[1]%
{{\parbox{10mm}{\small\textit{\fcolorbox{boxfrcolor}{boxbgcolor}{#1}}}}}
\newcommand{\geomargin}[2][{\color{boxfgcolor}geodesy}]%
{\ifthenelse{\boolean{geos}}{% if:
\marginnote{\protect\geobox{#1}}\footnote{#2}}{% else: show nothing
}}

% for itemizing requirements
\newcounter{req}
\newcommand{\reqid}[1]{\item[\refstepcounter{req}#1-\thereq]}

%
% eof
%

\setboolean{geos}{true}% or false to hide geodetic marginal notations

% control options
\newboolean{skipappendix}
\setboolean{skipappendix}{false}

% at some point this gets frozen
\newcommand{\recdate}{\today}

\begin{document}
\DeclareGraphicsExtensions{.png, .jpg, .pdf}
% best to put figures in subdirs
%\graphicspath{{figs/}{scans/}}

% for subsequent pages
\setlength\headheight{15pt}
\fancyhead[L]{HOPS}
\fancyhead[C]{}
\fancyhead[R]{Requirements}
\fancyfoot[R]{Page \thepage\ of\ \pageref{page:LastPage}}
\fancyfoot[L]{\recdate}

\title{ngEHT Coverage and Testing for a new HOPS}

\author{%
\LARGE John Barrett, Geoff Crew, Dan Hoak and Violet Pfeiffer \\
\Large MIT Haystack Observatory}
\date{Version 0.1, \recdate}
\maketitle
\normalsize

\renewcommand\abstractname{Executive Summary}
\abstract{%
\large
This document includes the general plan and some specifics
on our testing of the new \ac{HOPS}.  \FIXME[more here]
}

% \begingroup . . . \endgroup can be used to keep things together
% and/or an explicit page break and/or adjust spacing as needed so
% it looks presentable.  Uncomment the sections you need.
%\begingroup
%\renewcommand\contentsname{Contents}
%\renewcommand\listfigurename{Figures}
%\renewcommand\listtablename{Tables}
%\vspace{24pt}\hrule
\pagebreak
\tableofcontents
%\vspace{24pt}\hrule
%\pagebreak
%\listoffigures
%\vspace{24pt}\hrule
%\listoftables
%\vspace{24pt}\hrule
%\endgroup

% it is sometimes cleaner to start sections on new pages in a longer
% document--then changes to each section don't repage everything.

% break document into appropriate portions
\pagebreak
%
% A starter on coverate and unit testing
%
\section{Coverage and Unit Testing}
\label{sec:coverunit}

this can be words about coverage and unit testing using autotools
(which we can do now...)

\reqid{A} The \acs{alist} program shall generate valid \acs{A-list} files in 
versions 5, 6. 
\reqex{\texttt{chk\_alist.sh} currently does this for versions 5 and 6.}

\reqid{A} The \acs{adump} program shall provide valid ASCII text representations 
of columns of data from an \acs{A-list} of version 5 \& 6.
\reqex{\texttt{chk\_adump.sh} currently does this for versions 5 and 6.}

\reqid{A} The \acs{aedit} program shall process versions 5 \& 6 \acs{A-list} 
files with respect to flagging, selecting, summarizing and generating a new 
output \acs{A-list}.
\reqex{the \texttt{chk\_aedit.sh} currently makes a pass through many of these}

\reqid{A} HOPS4 shall maintain the performance of \acs{aedit} version 5 (for 
geodesy) and 6 (for EHT).
\reqex{Currently, \texttt{chk\_fsumm.sh} performs such a test with captured 
\acs{A-list} data.}

\reqid{A} HOPS4 shall maintain the \acs{fourmer} tool, which shall properly 
relabel channels when it combines two sub-bands.
\reqex{This is currently performed by \texttt{test\_new\_chan\_id}.}

\reqid{A} The \acs{fringex} program shall retain existing functionality to rotate
fringes with respect to fourfit fringe solutions.
\reqex{The \texttt{chk\_fringex.sh} exists to verify this in HOPS3.}

\reqid{A} The ``average'' capability, implemented by \acs{average} used in 
concert with \acs{fringex} to subdivide explore and average fringes, shall be 
preserved with equivalent functionality.
\reqex{The \texttt{chk\_average.sh} program tests this for the existing 
\acs{average} application, but the piping mechanism is cumbersome and should be
re-implemented in a more easily used (and likely more efficient) C/C++ or 
\acs{Python} application.}

\reqid{A} The functionality of the program \acs{cofit} shall be preserved.
\reqex{\texttt{chk\_cofit.sh} currently verifies this.}

\reqid{A} There current functionality of the program \acs{search} shall be 
preserved.
\reqex{\texttt{chk\_search.sh} currently verifies this by performing a search on
a captive data set.}

\reqid{A} There shall continue to be a functional \acs{fourmer} tool to assemble
separately correlated frequency sub-bands.
\reqex{\texttt{chk\_fourmer.sh} currently does this for a captive two 512-MHz 
bands. We will need to build and maintain a test to cover an extension to 
current \acs{EHT} 2-GHz bands.}

\reqid{A} The ability to explore fringes (as is currently done with the 
combination of \acs{fringex}, \acs{average} and search must be preserved. 
HOPS4 shall support \acs{Python} scripting to aid the user in searching 
fringe space.
\reqex{The \texttt{chk\_frmrsrch.sh} script executes such a case.}

\reqid{A} The HOPS4 suite must provide mechanisms to preserve the correlator 
output data.
\reqex{A \acs{PERL} script, \texttt{hops\_data\_links.pl}, exists to manage 
symbolic links toanalysis files in a working directory (separate) from the original 
correlator output directory.  A script, \texttt{chk\_hdlinks.sh}, verified this capability 
in HOPS3, and a similar mechanism should be provided in HOPS4.}
    
\reqid{A} It shall be possible to automatically discard correlator Acquisition 
Periods with small weights.
\reqex{When data are poorly recorded, the correlation product is the result 
of less data than it should be, leading to incorrect results. The ability to flag 
such input should not be arduous. The HOPS3 script 
\texttt{chk\_min\_weight.sh} currently tests this capability. In addition, 
user-supplied ad-hoc flagging capability is tested by \texttt{chk\_flagging.sh}.}

\reqid{A} HOPS4 shall have the ability to flag data based on a user-supplied 
list (e.g. of frequency intervals) or a flag file.
\reqex{This is needed especially when \acs{RFI} or calibration tones signals 
are present.  In HOPS3 \texttt{chk\_notches.sh} verifies this.}

\reqid{A} Every data type written to disk in HOPS4 shall be convertable to 
form that is amenable to human examination, as is done in HOPS3 by the 
\texttt{\ac{CorAsc2}} program.
\reqex{For the HOPS3 \acs{fourfit} program, this is verified by 
\texttt{chk\_ff\_dump.sh}; for \acs{A-list} data, this is provided by \acs{adump},
or some combination of awk, sed and grep.}



%
% eof
%


\pagebreak
%
% A starter on coverage and unit testing
%
\section{Regression Testing}
\label{sec:regress}

this can be words about internal regression testing with autotools
(which we can do now...)


\subsection{Regression Tests with Captured Data}
\label{sec:captdatareq}

% ./data/ff_testdata/Makefile.am
In addition to the above tests, data from a variety of experiments
has been captured and are routinely refringed as part of the
HOPS build ``check'' process.  Data from two of
these are packaged with the HOPS distribution:

\begin{description}
\reqid{A} \texttt{chk\_ff\_2836.sh} a canonical S--X geodetic experiment
\reqid{A} \texttt{chk\_ff\_3365.sh} an early \ac{EHT} session suitable for
    testing \ac{fourmer}
\end{description}
Both of these tests exercise the basic fringe search and reduction
that \acs{fourfit} was built to do and have been the basis of standard
regression for HOPS and distribution validation.   Additional tests
with other captured data sets are discussed in the next section.



%Some additional tests extant in \ac{HOPS} should not be needed in
%\ac{MHO}; these are listed for reference in \App~\ref{sec:obsolete}.

%\subsection{Software-Testable Requirements}
%\label{sec:regress}

%It is anticipated that each of these tests will be captured by a
%script that may be executed on a regular basis, possibly requiring
%user intervention.  For example, when run in an interactive mode,
%\ac{fourfit} displays fringe plots which require a human to acknowledge
%that they have in fact been displayed.  (One can test the generation of
%the plot without a human, but the display mechanics require visual
%inspection.)

% ./sub/dfio/copypage/Makefile.am

Additional scans have been captured with fringe tests captured
in the \ac{HOPS} development system.  (That is, the tests exist,
but the data is not included in the distribution tarball.)


% ./data/ff_testdata/Makefile.am
\begin{description}
\reqid{S} verify that experiment 2843 fringes consistently.
  \reqex{currently done with \texttt{chk\_ff\_2843.sh} in HOPS}
\reqid{S} verify that experiment 3372 fringes consistently.
  \reqex{currently done with \texttt{chk\_ff\_3372.sh} in HOPS}
\reqid{S} verify that experiment 3413 fringes consistently.
  \reqex{currently done with \texttt{chk\_ff\_3413.sh} in HOPS}

\reqid{P} Regression tests of 2017 or 2018 data samples
  \reqex{Among the so-called ``golden'' scan data which has been
    captured for \acs{DiFX} regression testing, there are scans
    which can be refringed and \acs{A-list} manipulated and compared
    with previous results.  As set that is sufficient to guarantee
    consistency with published results is required.  Reprocessing
    the entire 2017 and 2018 data is not required.}

\end{description}

Additional test data has been captured from the (very) successful
2017 \ac{EHT} observations.  Fringing of these scans has not yet
been incorporated into HOPS testing, but shall form the basis
of regression for HOPS4.

It is anticipated that for these a mechanism shall be provided
so that they human need merely acknowledge success, failure or
make a note of issues.

Currently captured for \acs{DiFX} regression testing are the first 20 seconds
on the following experiment-release-subband-doy-hhmm samples:
\begin{verbatim}
e17d05-7-hi-095-0839 e17d05-7-lo-095-0839 e17d05-7-pc-095-0839
e17b06-7-hi-096-0750 e17b06-7-lo-096-0750 e17b06-7-pc-096-0750
e17c07-7-hi-097-1303 e17c07-7-lo-097-1303 e17c07-7-pc-097-1303
e17a10-7-hi-100-0725 e17a10-7-lo-100-0725 e17a10-7-pc-100-0725
e17e11-7-hi-101-0322 e17e11-7-hi-101-0803 e17e11-7-lo-101-0322
e17e11-7-lo-101-0803 e17e11-7-pc-101-0322 e17e11-7-pc-101-0803
\end{verbatim}
``lo'' and ``hi'' refer to the ``b3'' and ``b4'' sub-bands of the
64 \acs{Gbps} (2018 and onwards) frequency setup; ``pc'' refers to polconverted,rather than the mixed polarization correlation products.

\TBC\FIX[---is this a sufficient set?]

These or other scans could also be captured for full scan duration and
segmented at some interval, but it is not clear that this is necessary.

\subsection{Regression testing to be applied to following requirements}
\label{sec:regressreqs}

\begin{description}
% Post Processing and Fringe Plots
\reqid{A} The ability to explore fringes (as is currently done with the
  combination of \acs{fringex}, \acs{average} and search must be preserved.
  HOPS4 shall support \acs{Python} scripting to aid the user in searching
  fringe space.
\reqid{P} It should be possible to make 3D visualization plots.
\reqex{\acs{search} makes contour plots, but a 3D visualization of amplitude
  with delay and delay-rate would be useful.}
\reqid{P} HOPS4 shall implement interactive visualization tools in \acs{fourfit}.
\reqex{The HOPS3 version of \acs{fourfit} provides a fixed fringe summary
  plot. HOPS4 should have an interactive plotting capability to enable zooming
  and viewing fourfit results with an expanded scale.}
\reqid{D} Replace the existing \acs{fourfit} control file with a
  \acs{Python}-based control file. Currently existing functionality shall be
  preserved.
\reqex{The existing control file should continue to be supported. The new
  control file functionality will not likely be back-ported to HOPS3.}

% Miscellaneous Requirements
\reqid{A} It shall be possible to automatically discard correlator
  \acs{AP} (integrations) with small weights.
\reqid{A} HOPS4 shall have the ability to flag data based on a user-supplied
  list (\eg~of frequency intervals with time in some flag file).
\reqid{A} Every data type written to disk in HOPS4 shall be convertable to
  form that is amenable to human examination (e.g. ASCII or CSV), as is done in HOPS3 by the
  \texttt{\acs{CorAsc2}} program.
\reqid{A} HOPS4 shall implement an algorithm for solving for station-based
  quantities from baseline-based quantities as a global fringe fitter.
\reqid{A} The HOPS4 \acs{fourfit} shall have the ability to apply complex
  bandpass correction.
\reqid{A} HOPS4 shall provide a method to solve for complex bandpass corrections.
\reqid{A} Python wrappers for the new HOPS4 data objects shall be provided.
\reqex{The existing wrappers for the \acs{Mk4} data types in HOPS3 do not
  need to be preserved, but it is a capability worth preserving.}
\reqid{D} The HOPS4 suite must provide mechanisms to preserve the correlator
  output data.
\reqid{D} HOPS4 should preserve the the Single Band Delay, Multi Band Delay
  and Delay Rate search algorithm in its current form.
\reqex{Our desire is to make the algorithm modular, such that the user may (de)select search
  dimensions.
\reqid{D} Benchmarking and performance analysis should be augmented with
  more sophisticated tools.
\reqid{D} HOPS4 should support spectral-line VLBI.
\reqid{D} HOPS4 should support pulsar folding with a user specified period or
  ephemeris file with blanking.
\reqid{D} HOPS4 should enable multi-threading or multi-processing for batch jobs.
\end{description}

%
% eof
%


\pagebreak
%
% A starter on coverate and unit testing
%
\section{Integration Testing}
\label{sec:integration}

this can be words about integration regression testing with scripts

%
% eof
%


\pagebreak
%
% A starter on coverate and unit testing
%
\section{Human GUI Testing}
\label{sec:humanity}

this can be words about procedural testing of the command line
and GUI interfaces

%
% eof
%


\addtocounter{section}{1}
\renewcommand{\refname}{\thesection. References}
\addcontentsline{toc}{section}{\thesection. References}
\bibstyle{plainurl}
\printbibliography
\label{sec:references}

% if skipappendix-is-true then (nothing) else typeset Appendices
\ifthenelse{\boolean{skipappendix}}{}{%

\pagebreak
\appendix
\section{Acronyms, Commands and Glossary}
%
% \section{Acronyms, Commands and Glossary}
%
% \acro{acronym}[short name]{full name / description}
% \ac{acronym} is the usual usage in text that defines (and gives short name)
% \acs{acronym} gives just the short name
% \acf{acronym} gives just the full name
% \acsu{acronym} gives the short name and marks it used
% \a..p{acronum} makes it plural
%
% the optional short name can include math as the acronym key cannot
% there are a zillion other options, see https://ctan.org/pkg/acronym
%
\begin{acronym}
% A---------------------------------------------------------------
\acro{A-list}{a one line description of baseline fringes used by \ac{HOPS}}
\acro{adump}{a program that dumps columns from \ac{A-list} scan data}
\acro{alist}{a program for creating a file of \ac{A-list} scan data}
\acro{aedit}{a program for editing a file of \ac{A-list} scan data}
\acro{average}{a program that calculates averages on \ac{A-list} scan data}
\acro{AIPS}{Astronomical Image Processing System}
\acro{ALMA}{Atacama Large Millimeter/Submillimeter Array}
\acro{AMP}{short for ``amplitude'' the correlation coefficient}
\acro{AP}{Acquision Period which refers to a period of time over which the
    correlator integrates the input (noisy) data to produce a usable output.
    Terms such as \acs{dump} or ``integration'' are also sometimes used,
    but both can be ambiguous.}
\acro{awk}{A programmable language for parsing line and field oriented input.
    The program was part of the original \acs{UNIX} product, and is named for
    its three authors,  Alfred Aho, Peter Weinberger, and Brian Kernighan}
% B---------------------------------------------------------------
\acro{bigendian}{refers to a computer hardware architecture where the
    most significant
    bits of a larger storage object (bytes, words\ldots) are serialized first.}
\acro{bit}{a 0 or 1}
\acro{byte}{a unit of storage corresponding to 8 bits}
% C---------------------------------------------------------------
\acro{C}{The ``C'' programming language, created to make \ac{UNIX} portable}
\acro{C++}{The C++ programming language, an object-oriented
    successor to \ac{C}}
\acro{C/C++}{Refers to code that may be either \acs{C}, \acs{C++} or a mix of
    the two ``dialects''.  The two compliers currently in use in the project,
    \acs{GCC} and \acs{Clang} manage both dialects.}
\acro{CASA}{Common Astronomy Software Applications}
\acro{channel}{an ambigous term which refers either to a spectral channel,
    \ie~frequency point of an \acsu{FFT} or to a sub-band of a
    larger receiver band.}
\acro{cofit}{a \ac{HOPS} tool to assess atmospheric coherence in terms of
    \ac{SNR} and \ac{AMP} variation with integration interval}
\acro{CorAsc2}{Correlator to Ascii (2nd version)}
\acro{cover}{a coverage test exercises all logic branches of some code module}
% D---------------------------------------------------------------
\acro{DFT}{Discrete Fourier Transform}
\acro{DR}{Delay Rate, the fringe parameter concerning
    the change of delay with time}
\acro{DiFX}{the ``distributed'' \ac{FX} correlator}
\acro{difx2mark4}{a program (part of \ac{DiFX}) to convert \ac{SWIN}
    format correlation products into the ``Mark4'' (or \acs{Mk4})
    data files used by \ac{HOPS}}
\acro{dump}{a term used with hardware correlators to refer to a time
    integration performed by hardware/firmware circuitry.  The dumped
    data may then be further integrated in software.}
% E---------------------------------------------------------------
\acro{EHT}{the Event Horizon Telescope}
\acro{EHTC}{the Event Horizon Telescope Collaboration, which usually
refers to the organization that operates the \ac{EHT}}
% F---------------------------------------------------------------
\acro{FFT}{Fast Fourier Transform}
\acro{FFTW3}{Fastest Fourier Transform in the West, version 3}
\acro{Fortran}{a FORmula TRANslation language, in common use prior to \ac{C}}
\acro{FX}{a general term for correlation that does the cross-correlation
    after first transforming to frequency space}
\acro{FITS}{Flexible Image Transport System, now referring to a
    general digital data format}
\acro{FITS-IDI}{A dialect of \ac{FITS}
    designed for the interchange of data for interferometry}
\acro{flag}{A term commonly used in radio astronomy to mark bad data
    for exclusion from further analysis.}
\acro{fourfit}{the main fringe-finding command in \ac{HOPS}}
\acro{fringex}{an \ac{HOPS} tool to explore the fringe} 
\acro{fourmer}{a program that combines data from two sub-bands into
    a larger common band}
% G---------------------------------------------------------------
\acro{ghostscript}{Ghostscript, the GNU \acs{PostScript} emulator}
\acro{Gbps}{refers to data recording rate, usually.  8 Gbps is 1 GB/s
    or one billion characters (of ASCII) per second.  Usually there
    are (packet) overheads in the actual recording so the write or
    playback speed may be somethings slightly or grossly different.
    The \ac{HOPS} era started with kbps worked through Mbps and ended
    with Gbps.  Tbps will probably be with us in another decade.}
\acro{GHz}{one billion Hz}
\acro{GNU}{GNU is Not Unix (a software project launched by
    Richard Stallman in the 80's)}
\acro{GNU/Linux}{a family of operating systems using Linus' kernel and
    GNU's software packages}
\acro{grep}{global regular expression parser, a name for a collection of
    tools that perform regular expression parsing of input data strings.}
\acro{GS}{short for \ac{ghostscript}}
\acro{GSL}{\acs{GNU} Science Library, a library of functionality for
    science applications.}
\acro{GUI}{Graphical User Interface}
% H---------------------------------------------------------------
\acro{HDF5}{Hierarchical Data Format, version 5.\protect\footnote{Why would you
    \textit{want} to use anything that took 5 versions to get right?}}
\acro{HOPS}{Haystack Observatory Postprocessing System}
\acro{Hz}{A frequency unit named for Heinrich Hertz.
    A frequency of one Hz is one oscillation per second.}
% I---------------------------------------------------------------
\acro{i/o}{short for input/output referring to the fact that programs are
    written to act on something and provide something}
\acro{IPP}{Intel Performance Primitives is a library of functionality
    optimized for use with the Intel processor family}
% J---------------------------------------------------------------
\acro{JIVE}{now just a name for an organization, it is still an
    Institution for VLBI in Europe, just not a Joint one}
% L---------------------------------------------------------------
\acro{Linux}{a family of operating systems built around Linus Torval's version
    of the UNIX kernel}
\acro{littleendian}{refers to a computer hardware architecture where the
    least significant
    bits of a larger storage object (bytes, words\ldots) are serialized first.}
\acro{LSF}{Least Squares Fit}
% M---------------------------------------------------------------
\acro{MBD}{Multi-Band Delay, the delay parameter referring to the change of
    phase with frequency in a multi-channel (sub-band) system.}
\acro{Mk4}{The fourth in a series of \ac{VLBI} hardware correlators.  The
    Mark4 replaced the Mark3 near the beginning of the millenium, and was
    finally put to rest by \ac{DiFX} in the mid 2010's}
\acro{m4py}{a shallow \acs{Python} wrapper which provides access to
    \acs{Mk4} data files and types}
\acro{MS}{Measurement Set, a formal specification for data to be analyzed
    with reference to a Measurement Equation}
\acro{MSRI}{Mid-scale Research Initiative}
% N---------------------------------------------------------------
\acro{NSF}{National Science Foundation}
\acro{ngEHT}{next-generation \acs{EHT}}
% O---------------------------------------------------------------
\acro{ovex}{an ``observer'' dialect of \acs{VEX}}
\acro{OpenMPI}{Open MPI Project is an open source Message Passing Interface
    implementation}
% P---------------------------------------------------------------
\acro{PDF}{Portable Document Format (developed by Adobe) as a successor
    to \ac{PostScript}}
\acro{PGPLOT}{a ``pretty good'' plotting package developed and maintained
    by Tim Pearson at Caltech.  He's retired now, so it is stuck at verion
    5.2.2, (released Feb 2001)}
\acro{PostScript}{a printer page description language developed by Adobe.
    \ac{fourfit} plots are currently generated in \ac{PostScript} and
    often converted to \acs{PDF}}
\acro{PERL}{Practical Extraction and Reporting Language created by Larry Wall}
\acro{PS}{short for \ac{PostScript}}
\acro{Python}{a programming language named in honor of Monty Python's Flying
    Circus}
% R---------------------------------------------------------------
\acro{RFI}{Radio Frequency Interference which is what you have when your
    receiver picks up signals you do not want}
% S---------------------------------------------------------------
\acro{SBD}{Single Band Delay, the delay parameter referring to the time
    offset between two signals being correlated}
\acro{search}{this is a tool that searches in delay/delay-rate space to
    allow visualization of a fringe peak and to aid in establishing the
    validity of more marginal-\acs{SNR} cases}
\acro{sed}{is a stream editor, that ingests line-oriented data and performs
    programmatic operations on it prior to output}
\acro{SFXC}{\acs{JIVE}'s software \ac{FX}-kind correlator}
\acro{SNR}{Signal to Noise Ratio}
\acro{SWIN}{the output format used by the \ac{DiFX} correlator}
\acro{MHO}{MIT Haystack Observatory Postprocessing System}
% T---------------------------------------------------------------
\acro{TEC}{Total Electron Content, refering to the column density of
    electrons in the line of sight through the ionosphere.  Conventionally
    one TEC Unit is \protect{$10^{16}$ electrons / m$^2$}}
% U---------------------------------------------------------------
\acro{unit}{a unit test is a short test used to validate a small part of
    some larger code module}
\acro{Unicode}{here, a general reference to a collection of methods for
    representing printable characters beyond ASCII.  The painful
    \ac{Python} 2 to 3 transition was driven by a need to more correctly
    handle strings of Unicode character representations.}
\acro{UNIX}{the name of a family of operating systems
    (born in the 70's at Bell Laboratories)}
% V---------------------------------------------------------------
\acro{VEX}{\acs{VLBI} EXperiment (file), a means of fully describing
    a planned \acs{VLBI} experiment or observation}
\acro{VEX2XML}{a program that converts \acs{VEX} files into an easily
    parsed \acs{XML} represention}
\acro{VGOS}{\acs{VLBI} Global Observing System;
    was called \acs{VLBI}2010 until the mid 2010's}
\acro{VLBI}{Very Long Baseline Interferometry}
% W---------------------------------------------------------------
\acro{Whitneys}{correlation amplitudes are normally expressed between 0 and 1,
    but in our work they are usually small and in \ac{HOPS} traditionally
    multiplied by ten thousand, in which case, the unit of correlation amplitude
    is ``Whitneys'' after Alan Whitney who may be commended or blamed for the
    usage.}
\acro{word}{an architecture-dependent unit of storage---these days, most of
    our processors use 8-\acs{byte} words}
% X---------------------------------------------------------------
\acro{XF}{a general term for correlation that does the cross-correlation
    first, and then transforms the result to frequency space}
\acro{XML}{eXtensible Markup Language}
% Z---------------------------------------------------------------
\acro{zero-pad}{the practice of extending time or frequency sequences with
    some number of zeroes which, for \ac{FFT}s has the effect of smoothing
    in the other domain}
\end{acronym} 
%
% eof
%


}

\label{page:LastPage}
\end{document}
