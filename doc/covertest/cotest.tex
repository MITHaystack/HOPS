\documentclass[notitlepage,letterpaper,pdftex,12pt,final]{article}
%\documentclass[notitlepage,letterpaper,pdftex,12pt,draft]{article}
% article < report < book ; preable material follows
%
% This file is mostly boilerplate that you can copy and tweak.
% The true material is in included files.  ../common should hold
% things that more than one document might need.  The Makefiles
% should likewise be mostly boilerplate with minor tweaks.
%

% (un)comment to (see)hide full paths to files actually used
% however, you'll need to override the TEXOPT= setting to batchmode
% made in the Makefile.
\listfiles

\addtolength\textwidth{1.7in}
\addtolength\oddsidemargin{-0.95in}
\addtolength\evensidemargin{-0.95in}
\addtolength\marginparwidth{-0.95in}
\typeout{textwidth is \the\textwidth}

% useful to block out portions of text
\usepackage{ifthen}
% to allow defining our own colors
\usepackage[dvipsnames]{xcolor}
% makes hyperlinks work
\usepackage{hyperref}
\hypersetup{colorlinks=true,linkcolor=darkblue,citecolor=darkgreen}
% for figures; insert [draft] before {...} not to see imgs
\usepackage{graphicx}
% for small figures surrounded by text
\usepackage{wrapfig}
% for control over headers and footers
\usepackage{fancyhdr}
\pagestyle{fancy}
\fancyhf{}
% for handling acronyms
\usepackage[printonlyused]{acronym}

% for bibliographies if needed
\usepackage[utf8]{inputenc}
\usepackage[english]{babel}
\usepackage[backend=biber,style=numeric,sorting=none]{biblatex}
\addbibresource{hops.bib}
% context-sensitive quotes
\usepackage{csquotes}
% underline for emphasis: \uline etc
\usepackage{ulem}
% for generating outlines
\usepackage{outlines}
\usepackage{enumitem}

\usepackage[printwatermark]{xwatermark}
\newwatermark[pages=1-12,color=gray!25,angle=45,scale=5,xpos=0,ypos=0]{DRAFT}

% for bold math
\usepackage{bm}
% for serious math which we may have in a few places (eventually)
\usepackage{amsmath}
% equations include section number
\numberwithin{equation}{section}

% for marginal notations concerning geodesy
\usepackage{marginnote}

% to go back to default after \ragged commmands
\usepackage{ragged2e}

% see common/shortcuts.tex for what is defined in this file
%
% a file of commands to save typing
%

\newcommand{\eg}{\textit{e.g.}}
\newcommand{\EG}{\textit{E.g.}}
\newcommand{\ie}{\textit{i.e.}}
\newcommand{\IE}{\textit{I.e.}}
\newcommand{\etc}{\textit{\&c.}}
\newcommand{\Sec}{Section}
\newcommand{\Fig}{Figure}
\newcommand{\Tab}{Table}
\newcommand{\App}{Appendix}
\newcommand{\FIX}[1][fixme]{{\color{red}#1}}
\newcommand{\TBC}{{\color{red}TBC}}
\newcommand{\TBD}{{\color{red}TBD}}
\newcommand{\TBR}{{\color{red}TBR}}
\newcommand{\FIXME}[1][]{{\color{red}FIXME -- #1}}

\newcommand{\MHO}{MIT-HOPS}
\newcommand{\HOPS}{HOPS}

%
% eof
%

\setboolean{geos}{true}% or false to hide geodetic marginal notations

% control options
\newboolean{skipappendix}
\setboolean{skipappendix}{false}

% at some point this gets frozen
\newcommand{\recdate}{\today}

\begin{document}
\DeclareGraphicsExtensions{.png, .jpg, .pdf}
% best to put figures in subdirs
%\graphicspath{{figs/}{scans/}}

% for subsequent pages
\setlength\headheight{15pt}
\fancyhead[L]{HOPS}
\fancyhead[C]{}
\fancyhead[R]{Requirements}
\fancyfoot[R]{Page \thepage\ of\ \pageref{page:LastPage}}
\fancyfoot[L]{\recdate}

\title{ngEHT Coverage and Testing for a new HOPS}

\author{%
\LARGE John Barrett, Geoff Crew, Dan Hoak and Violet Pfeiffer \\
\Large MIT Haystack Observatory}
\date{Draft - Version 1.0, \recdate}
\maketitle
\normalsize

\renewcommand\abstractname{Summary}
\abstract{%
\large
This document describes the coverage and test plan for the HOPS4
redevelopment. The suite of tests is intended to provide developers
with up-to-date verification of the required HOPS functionality
across the supported platforms and distributions. It also provides
end users with build-time checks that the installation was successful.
Additionally, we use coverage tools to demonstrate that the test plan
executes a satisfactory fraction of the code, and that all required
functions and cases are exercised by the tests.

}

% \begingroup . . . \endgroup can be used to keep things together
% and/or an explicit page break and/or adjust spacing as needed so
% it looks presentable.  Uncomment the sections you need.
%\begingroup
%\renewcommand\contentsname{Contents}
%\renewcommand\listfigurename{Figures}
%\renewcommand\listtablename{Tables}
%\vspace{24pt}\hrule
\pagebreak
\tableofcontents
%\vspace{24pt}\hrule
%\pagebreak
%\listoffigures
%\vspace{24pt}\hrule
%\listoftables
%\vspace{24pt}\hrule
%\endgroup

% it is sometimes cleaner to start sections on new pages in a longer
% document--then changes to each section don't repage everything.

% break document into appropriate portions
%
% CONTENTS OF THE DOCUMENT SHOULD BE THE FOLLOWING:
% Mention goals, methods, scope, schedule, and expected results.
% What we are testing.
% Why we are testing that.
% Risk assesment and mitigations?
% Resources (i.e. hardware, software, dev tools, personnel)
% Documents that will be developed during testing (coverage report)
% 
% To do:
%   Add numbering to sections. 
%   Correctly reference requirements document.
%   Replace the data in the tables with real data.
%

% introductory thoughts %
\section{Purpose}
\label{sec:purpose}
The \ac{EHT} has launched a \ac{MSRI} project with the goal of developing
the technologies needed for a second-generation \acs{EHT}. 
This document describes the test plan for the refactoring of the \ac{HOPS}
as part of the work package of the \acs{MSRI} proposal. 

\section{Applicability}
\label{sec:applicability}
This document is applicable to the HOPS4 redevelopment effort only.

\section{Referenced and Applicable Documents}
% Mention the requirements document if necessary.
\label{sec:referenced-docs}
\begin{table}[h!]
\centering
 \begin{tabular}{c | c | c } 
 %\hline
 Title & Vers. & Date \\ [0.5ex]
 \hline%\hline
 HOPS4 Requirements & 1.0 & January 8, 2021 \\ [1ex]
 HOPS4 Specifications & 0.1 & \recdate \\
 %\hline
 \end{tabular}
 \caption{HOPS4 documents referenced by this document}
 \label{table:1}
\end{table}

\section{Resources}
\label{sec:resources}
% Personnel, hardware, software, dev tools.

%\section{Risk Assesment}
%\label{sec:risk-assessment}
% Risks to the project and how we will mitigate them.
% Hopefully we don't have to do this.

\section{Test Types and Schedule}
\label{sec:schedule}

The tests described in this document are grouped into the following categories.

\begin{itemize}
\item[] \textbf{Unit tests} are defined for discrete functions or sections of code.
\item[] \textbf{Regression tests} are automated and generate quantitative results for specific
  functions using captured data. They have pre-defined criteria for success (for example, an SNR range for a fringe fit). Many of these tests are direct comparisons to results from HOPS3,
  and are largely focused on detecting if changes break pre-existing functionality.
\item[] \textbf{Continuous integration tests} check that the latest HOPS4 codebase successfully compiles in a variety of environments and distributions.
\item[] \textbf{Benchmarking tests} are used to verify the performance of HOPS4 functions.
\item[] \textbf{Oracle tests} are direct comparisons between HOPS3 and HOPS4. They are implemented using an ``oracle'', a software package that runs HOPS3 to determine a result of interest. Then the necessary sub-set of HOPS4 code is executed on the same dataset to ensure the result is within tolerance of the oracle's result.
\item[] \textbf{Component tests} are tests of individual components (libraries) and their dependencies. These are somewhat like regression tests, but are more narrowly focused on a sub-set of code. They are designed such that they do not rely on any dependencies beyond what is needed by the component under test. 
\item[] \textbf{Manual tests} are tests that require human interpretation of results, for example to verify that a plot is generated with the proper formatting, or that a GUI is functioning as expected.

\end{itemize}

These tests are executed on different schedules.

\begin{table}[h!]
\centering
 \begin{tabular}{c | c} 
 %\hline
 Test type & Frequency \\ [0.5ex] 
 \hline%\hline
 Regression &  Per commit \\ 
 Continuous Integration & Daily \\
 Unit testing & Per Class/Function \\ 
 Benchmarking & As needed \\
 Oracle  & As needed \\
 Component & As needed \\ 
 Manual & As needed \\ [1ex]
 %\hline
 \end{tabular}
 \caption{Types of testing planned for HOPS4}
 \label{table:2}
\end{table}


\section{Reports}
\label{sec:reports}
Tests will generate reports, which are human-readable documents that 
summarize the success and failure of the tests at various stages
(commit, build, weekly, etc). 


\section{Scope}
\label{sec:scope}
% What will and will not be tested.
The scope of this test plan does not include \ac{HOPS} 3. This test plan will test
all new functionality, libraries, other dependencies, features, and refactoring involved in
\ac{HOPS}. We will also use specific algorithm verification tests, using \ac{HOPS} 3 as a test
oracle, and regression tests with captive 2017/2018 data scans.

%\pagebreak
%\section{Testing Methods}
%\label{sec:testing-methods}
% Mention that we use Agile so this is our schedule and they need to deal with it.

%
% A starter on coverate and unit testing
%
\section{Regression Testing}
\label{sec:regress}

this can be words about internal regression testing with autotools
(which we can do now...)

%
% eof
%

% Test that data with both HOPS4 and HOPS3 return the same results.

%\pagebreak
%
% A starter on coverate and unit testing
%
\section{Continuous Integration Testing}
\label{sec:integration}

%this can be words about integration regression testing with scripts
Continuous Integration is the process of merging all the working copies of the code
base in to master and running automated tests on the result of that merge, often 
several times a day.

Currently HOPS3 supports continuous integration by way of nightly automated scripts 
that report back the results of those tests to the development team.
The HOPS development team plans to continue to utilize this automated testing
and make it compatible with HOPS4. The HOPS development team plans to explore the posibility
of replacing the current nightly automated script with a freely available and widely
accepted CI tool utilized by the open source development community such as Github Actions. 

% This section should prbably include all of the same requirements mentioned in the integration section.
\subsection{Regression testing to be applied to following requirements}
\label{sec:integrationreqs}

\begin{description}
% Post Processing and Fringe Plots
\reqid{A} The ability to explore fringes (as is currently done with the
  combination of \acs{fringex}, \acs{average} and search must be preserved.
  HOPS4 shall support \acs{Python} scripting to aid the user in searching
  fringe space.
\reqid{P} It should be possible to make 3D visualization plots.
\reqex{\acs{search} makes contour plots, but a 3D visualization of amplitude
  with delay and delay-rate would be useful.}
\reqid{P} HOPS4 shall implement interactive visualization tools in \acs{fourfit}.
\reqex{The HOPS3 version of \acs{fourfit} provides a fixed fringe summary
  plot. HOPS4 should have an interactive plotting capability to enable zooming
  and viewing fourfit results with an expanded scale.}
\reqid{D} Replace the existing \acs{fourfit} control file with a
  \acs{Python}-based control file. Currently existing functionality shall be
  preserved.
\reqex{The existing control file should continue to be supported. The new
  control file functionality will not likely be back-ported to HOPS3.}

% Miscellaneous Requirements
\reqid{A} It shall be possible to automatically discard correlator
  \acs{AP} (integrations) with small weights.
\reqid{A} HOPS4 shall have the ability to flag data based on a user-supplied
  list (\eg~of frequency intervals with time in some flag file).
\reqid{A} Every data type written to disk in HOPS4 shall be convertable to
  form that is amenable to human examination (e.g. ASCII or CSV), as is done in HOPS3 by the
  \texttt{\acs{CorAsc2}} program.
\reqid{A} HOPS4 shall implement an algorithm for solving for station-based
  quantities from baseline-based quantities as a global fringe fitter.
\reqid{A} The HOPS4 \acs{fourfit} shall have the ability to apply complex
  bandpass correction.
\reqid{A} HOPS4 shall provide a method to solve for complex bandpass corrections.
\reqid{A} Python wrappers for the new HOPS4 data objects shall be provided.
\reqex{The existing wrappers for the \acs{Mk4} data types in HOPS3 do not
  need to be preserved, but it is a capability worth preserving.}
\reqid{D} The HOPS4 suite must provide mechanisms to preserve the correlator
  output data.
\reqid{D} HOPS4 should preserve the the Single Band Delay, Multi Band Delay
  and Delay Rate search algorithm in its current form.
\reqex{Our desire is to make the algorithm modular, such that the user may (de)select search
  dimensions.
\reqid{D} Benchmarking and performance analysis should be augmented with
  more sophisticated tools.
\reqid{D} HOPS4 should support spectral-line VLBI.
\reqid{D} HOPS4 should support pulsar folding with a user specified period or
  ephemeris file with blanking.
\reqid{D} HOPS4 should enable multi-threading or multi-processing for batch jobs.
\end{description}


%
% eof
%

% Handled by Travis CI or something similar. 

%\pagebreak
%
% A starter on unit testing
%
\section{Unit Testing and Coverage}
\label{sec:unit}

The HOPS4 requirements specify that ``every library or module shall have a corresponding test suite to verify functionality'' (requirement \textbf{A-6}). Unit tests are test code written at the time of development that exercise basic functions of function, classes, libraries, etc., and verify the code is functioning as expected.

Unit tests are designed to catch errors from changes to input data, updates to dependencies, unexpected behavior due to changes in the environment or operating system, incorrect error handling, and so on. Integration testing is intended to check if the code compiles and runs as a complete module; unit testing checks that the basic functions operate as expected. Unit tests are isolated tests on specific functions, and provide granular information about functions that are behaving differently than intended.

We intend to use a coverage analysis tool like \texttt{gcov} to check that we cover a sufficient fraction of the code when testing.


%
% eof
%

% Mention testing suite(s) we are using.

%\pagebreak
%
% A starter on benchmark testing
%
\section{Benchmark Testing}
\label{sec:benchmark}

%this can be words about benchmark testing with scripts

%
% eof
%

% Mention benchmark tool.

%\pagebreak
%
% A starter on coverate and unit testing
%
\section{Test Oracle}
\label{sec:oracle}
HOPS 3 being the previous release will be used as a test Oracle for HOPS 4. This means that
results from HOPS 4 will be compared with the results and expected behavior of HOPS 3 using
the same test data. The interface and otherwise expected functionality of the commandline UI
and its output will be tested in the same manner. The following requirements will all be tested
in this manner.

%this can be words about the test oracle with scripts

%\subsection{Regression testing to be applied to following requirements}
%\label{sec:oraclereqs}


%
% eof
%


%\pagebreak
%
% A starter on component testing
%
\section{Component Testing}
\label{sec:component}

\FIX[TBD - what do we mean by component testing?]

%Component testing is intended to verify functionality in the presence of different version of dependencies.

%this can be words about component testing with scripts

%\subsection{Component testing to be applied to following requirements}
%\label{sec:componentreqs}



%
% eof
%

% Test libraries and such.

%\pagebreak
%
% A starter on coverate and unit testing
%
\section{Human GUI Testing}
\label{sec:manual}

%this can be words about procedural testing of the command line
%and GUI interfaces
% Not sure if they need as much detail as this. This table would consist of the actual tests that would be run.
Manual testing involves a developer running the latest build and testing the user interface      
\begin{table}[h!]
\centering
 \begin{tabular}{||c c c c||} 
 \hline
 Test ID & Description & Steps & Pass/Fail Criteria \\ [0.5ex] 
 \hline\hline
 01 & Test if HOPS builds & 1. Run lorem ipsum. 2. ? 3. Profit  & Pass if able to build the project without any errors. 
 02 & Test if HOPS documentation builds & 1...inf & Pass if able to build the project \LaTeX files without any errors. \\ [1ex]
 \hline
 \end{tabular}
 \caption{HOPS4 documents referenced by this document}
 \label{table:1}
\end{table}



%
% eof
%

%i.e. Geoff tests HOPS 4 by running it as a user would.

%\pagebreak
%
% A starter on coverage testing
%
\section{Reports}
\label{sec:reports}

The goal of the testing plan is to generate reports that document the results of the tests, the progress towards fulfilling the requirements, and code coverage. Reports are human-readable documents that summarize the success and failure of the tests at various stages (commit, build, weekly, etc).

The coverage and testing reports may be directly used by project management to monitor progress. The exact content and format of these reports is to be determined.

The nightly build process generates a summary report on progress
against requirements.

%
% eof
%



\addtocounter{section}{1}
\renewcommand{\refname}{\thesection. References}
\addcontentsline{toc}{section}{\thesection. References}
\bibstyle{plainurl}
\printbibliography
\label{sec:references}

% if skipappendix-is-true then (nothing) else typeset Appendices
\ifthenelse{\boolean{skipappendix}}{}{%

\pagebreak
\appendix
\section{Acronyms, Commands and Glossary}
%
% \section{Acronyms, Commands and Glossary}
%
% \acro{acronym}[short name]{full name / description}
% \ac{acronym} is the usual usage in text that defines (and gives short name)
% \acs{acronym} gives just the short name
% \acf{acronym} gives just the full name
% \acsu{acronym} gives the short name and marks it used
%
% the optional short name can include math as the acronym key cannot
%
% see https://ctan.org/pkg/acronym
%
\begin{acronym}
% A---------------------------------------------------------------
\acro{A-list}{a one line description of baseline fringes used by \ac{HOPS}}
\acro{alist}{a program for creating a file of \ac{A-list} scans}
\acro{aedit}{a program for editing a file of \ac{A-list} scans}
\acro{AIPS}{Astronomical Image Processing System}
\acro{AMP}{short for ``amplitude'' the correlation coefficient}
% C---------------------------------------------------------------
\acro{C}{The ``C'' programming language, created to make \ac{UNIX} portable}
\acro{C++}{The C++ programming language, an object-oriented
    successor to \ac{C}}
\acro{CASA}{Common Astronomy Software Applications package}
\acro{channel}{an ambigous term which refers either to a spectral channel,
\ie~frequency point of an \acsu{FFT} or to a sub-band of a larger receiver
band.}
\acro{cofit}{a \ac{HOPS} tool to assess atmospheric coherence in terms of
    \ac{SNR} and \ac{AMP} variation with integration interval}
\acro{CorAsc2}{Correlator to Ascii (2nd version)}
\acro{cover}{a coverage test exercises all logic branches of some code module}
% D---------------------------------------------------------------
\acro{DFT}{Discrete Fourier Transform}
\acro{DR}{Delay Rate, the fringe parameter concerning the change of delay with time}
\acro{DiFX}{the ``distributed'' \ac{FX} correlator}
\acro{difx2mark4}{a program (part of \ac{DiFX}) to convert \ac{SWIN}
format correlation products into the ``Mark4'' data files used by \ac{HOPS}}
% E---------------------------------------------------------------
\acro{EHT}{the Event Horizon Telescope, which usually refers to the
    observing array}
\acro{EHTC}{the Event Horizon Telescope Collaboration, which usually
refers to the organization that operates the \ac{EHT}}
% F---------------------------------------------------------------
\acro{FFT}{Fast Fourier Transform}
\acro{FFTW3}{Fastest Fourier Transform in the West, version 3}
\acro{Fortran}{a FORmula TRANslation language, in common use prior to \ac{C}}
\acro{FX}{a general term for correlation that does the cross-correlation
    after first transforming to frequency space}
\acro{FITS}{Flexible Image Transport System, now referring to a
    general digital data format}
\acro{FITS-IDI}{A dialect of \ac{FITS}
    designed for the interchange of data for interferometry}
\acro{fourfit}{the main fringe-finding command in \ac{HOPS}}
\acro{fringex}{an \ac{HOPS} tool to explore the fringe} 
\acro{fourmer}{a program that combines data from two sub-bands into
    a larger common band}
% G---------------------------------------------------------------
\acro{ghostscript}{Ghostscript, the GNU \acs{PostScript} emulator}
\acro{GHz}{one billion Hz}
\acro{GNU}{GNU is Not Unix (a software project launched by
    Richard Stallman in the 80's)}
\acro{GNU/Linux}{a family of operating systems using Linus' kernel and
    GNU's software packages}
\acro{GS}{short for \ac{ghostscript}}
\acro{GUI}{Graphical User Interface}
% H---------------------------------------------------------------
\acro{HDF5}{Hierarchical Data Format, version 5.}\footnote{Why would you
\emph{want} to use anything that took 5 versions to get right?}
\acro{HOPS}{Haystack Observatory Postprocessing System}
\acro{Hz}{A frequency unit named for Heinrich Hertz.
    A frequency of one Hz is one oscillation per second.}
% I---------------------------------------------------------------
\acro{i/o}{short for input/output referring to the fact that programs are
    written to act on something and provide something}
\acro{IPP}{Intel Performance Primitives is a library of functionality
optimized for use with the Intel processor family}
% J---------------------------------------------------------------
\acro{JIVE}{now just a name for an organization, it is still an
    Institution for VLBI in Europe, just not a Joint one}
% L---------------------------------------------------------------
\acro{Linux}{a family of operating systems built around Linus Torval's version
    of the UNIX kernel}
\acro{LSF}{Least Squares Fit}
% M---------------------------------------------------------------
\acro{MBD}{Multi-Band Delay, the delay parameter referring to the change of
    phase with frequency in a multi-channel (sub-band) system.}
\acro{Mk4}{The fourth in a series of \ac{VLBI} hardware correlators.  The
    Mark4 replaced the Mark3 near the beginning of the millenium, and was
    finally put to rest by \ac{DiFX} in the mid 2010's}
\acro{m4py}{a shallow \acs{Python} wrapper which provides access to
    \acs{mk4} data files and types}
\acro{MS}{Measurement Set, a formal specification for data to be analyzed
    with reference to a Measurement Equation}
\acro{MSRI}{Mid-scale Research Initiatives program at the \ac{NSF}}
% N---------------------------------------------------------------
\acro{NSF}{National Science Foundation}
\acro{ngEHT}{next-generation \ac{EHT}}
% O---------------------------------------------------------------
\acro{ovex}{an ``observer'' dialect of \acs{VEX}}
\acro{OpenMPI}{Open MPI Project is an open source Message Passing Interface implementation}
% P---------------------------------------------------------------
\acro{PDF}{Portable Document Format (developed by Adobe) as a successor
    to \ac{PostScript}}
\acro{PGPLOT}{a ``pretty good'' plotting package developed and maintained
    by Tim Pearson at Caltech.  He's retired now, so it is stuck at verion
    5.2.2, (released Feb 2001)}
\acro{PostScript}{a printer page description language developed by Adobe.
    \ac{fourfit} plots are currently generated in \ac{PostScript} and
    often converted to \acs{PDF}}
\acro{PS}{short for \ac{PostScript}}
\acro{Python}{a programming language named in honor of Monty Python's Flying
    Circus}
% S---------------------------------------------------------------
\acro{SBD}{Single Band Delay, the delay parameter referring to the time
    offset between two signals being correlated}
\acro{SFXC}{\acs{JIVE}'s software \ac{FX}-kind correlator}
\acro{SNR}{Signal to Noise Ratio}
\acro{SWIN}{the output format used by the \ac{DiFX} correlator}
\acro{MHO}{MIT Haystack Observatory Postprocessing System}
% U---------------------------------------------------------------
\acro{unit}{a unit test is a short test used to validate a small part of
    some larger code module}
\acro{UNIX}{the name of a family of operating systems
    (born in the 70's at Bell Laboratories)}
% V---------------------------------------------------------------
\acro{VEX}{\acs{VLBI} EXperiment (file), a means of fully describing
    a planned \acs{VLBI} experiment or observation}
\acro{VEX2XML}{a program that converts \acs{VEX} files into an easily
    parsed \acs{XML} represention}
\acro{VGOS}{\acs{VLBI} Global Observing System;
    was called \acs{VLBI}2010 until the mid 2010's}
\acro{VLBI}{Very Long Baseline Interferometry}
% W---------------------------------------------------------------
\acro{Whitneys}{correlation amplitudes are normally expressed between 0 and 1,
    but in our work they are usually small and in \ac{HOPS} traditionally
    multiplied by ten thousand, in which case, the unit of correlation amplitude
    is ``Whitneys'' after Alan Whitney who may be commended or blamed for the
    usage.}
% X---------------------------------------------------------------
\acro{XF}{a general term for correlation that does the cross-correlation
    first, and then transforms the result to frequency space}
\acro{XML}{eXtensible Markup Language}
% Z---------------------------------------------------------------
\acro{zero-pad}{the practice of extending time or frequency sequences with
    some number of zeroes which, for \ac{FFT}s has the effect of smoothing
    in the other domain}
\end{acronym} 
%
% eof
%


}

\label{page:LastPage}
\end{document}
