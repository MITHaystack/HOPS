%
\section{Regression Testing}
\label{sec:regress}

Regression tests are tests of exisiting HOPS3 functions on captured datasets and they will be used to verify that the HOPS4 performance and results are equivalent to HOPS3.  Due to the large size of some datasets, only a limited number of regression tests will be run as part of the build and installation process on captured data that is packaged with the HOPS repository.  The autotools/automake framework for build and installation currently supports a number of ``check'' scripts.

More intensive regression tests on datasets with a greater numbers of baselines or frequency bands will  be run with automated nightly scripts, or at regular intervals around significant version changes. These datasets are captured in the \ac{HOPS} development system but not the code repository.  That is, the test scripts will be maintained along with the code base, but the data will be stored spearately and not included in the distribution tarball or the project github.

These regression tests are typically be limited to an examination of a small set of results and as such can be automated to inssert a pass/fail summary in a report.
The current set of regression tests in the \textbf{HOPS3} code base will be migrated and applied to \textbf{HOPS4} as the feature which individual check are enabled. 

%\FIX[Regression tests are not tied to particular requirements, since the functionality they check already exists in HOPS3.]


\subsection{Captured Datasets}
\label{sec:captdatareq}

The current set of regression tests use data from a variety of experiments have been captured and are routinely refringed as part of the HOPS build ``check'' process.  
These are packaged with the current \textbf{HOPS3} distribution, and consist of the follow:

% commented experiments are included in the tarball but don't appear to be used by the check scripts?

\begin{description}
%\item{\textbf{2491}} 
%\item{\textbf{2611}} 
\item{\textbf{2836}} is a canonical two-station S--X geodetic experiment. The script \texttt{chk\_ff\_2836.sh} verifies that the SNR (parsed from the postscript file) is within expected bounds.
\item{\textbf{2843}} is a scan with three stations.
%\item{\textbf{2849}} 
%\item{\textbf{2912}} 
%\item{\textbf{3064}} 
%\item{\textbf{3262}} 
\item{\textbf{3365}} is an early \ac{EHT} session. The script \texttt{chk\_fourmer.sh} uses the \texttt{fourmer} command to join two scans, runs \texttt{fourfit}, and checks that the output file size is correct. Many of the check scripts use data from 3365.
\item{\textbf{3372}} is a scan with the legacy geodetic X-band setup.
\item{\textbf{3413}} is a single-baseline broadband experiment.
\item{\textbf{3562}} is a three-baseline broadband experiment.
\item{\textbf{3571}} is a single-baseline broadband experiment.
\item{\textbf{2017 and 2018 data samples}} Among the so-called ``golden'' scan data which have been captured for \acs{DiFX} regression testing, there are scans which can be refringed and \acs{A-list} manipulated and compared with previous results.  A full set that is sufficient to guarantee consistency with published results is required and must be defined.
\end{description}
Some datasets overlap with captured data selected for DiFX regression testing, which is important to alert us to upstream changes which may affect this project.

The first 20 seconds of the following experiment-release-subband-doy-hhmm samples are currently captured for \acs{DiFX} regression testing:
\begin{verbatim}
e17d05-7-hi-095-0839 e17d05-7-lo-095-0839 e17d05-7-pc-095-0839
e17b06-7-hi-096-0750 e17b06-7-lo-096-0750 e17b06-7-pc-096-0750
e17c07-7-hi-097-1303 e17c07-7-lo-097-1303 e17c07-7-pc-097-1303
e17a10-7-hi-100-0725 e17a10-7-lo-100-0725 e17a10-7-pc-100-0725
e17e11-7-hi-101-0322 e17e11-7-hi-101-0803 e17e11-7-lo-101-0322
e17e11-7-lo-101-0803 e17e11-7-pc-101-0322 e17e11-7-pc-101-0803
\end{verbatim}
``lo'' and ``hi'' refer to the ``b3'' and ``b4'' sub-bands of the 64 \acs{Gbps} (2018 and onwards) frequency setup; ``pc'' refers to polconverted,rather than the mixed polarization correlation products.


Additional test data has been captured from the 2017 \ac{EHT} observations.  Fringing of these scans has not yet been incorporated into HOPS testing, but shall form the basis of regression for HOPS4. These or other scans could also be captured (full scan duration or segmented at some interval) if do desired. On going re-evaluation of the sufficiency and coverage of these regression tests will be part of monthly risk-assesment and augmented as needed.

\subsection{Check Scripts from HOPS3}
\label{sec:regresscheck}

% existing HOPS3 scripts

There are a number of existing ``check'' scripts that run during build/install of HOPS3 to check for software regressions. These check scripts validate several requirements. Here, we describe the check scripts and describe the requirements that will be validated with regression tests.

%Both of these tests exercise the basic fringe search and reduction
%that \acs{fourfit} was built to do and have been the basis of standard
%regression for HOPS and distribution validation.

\begin{description}

\item{\textbf{chk\_adump.sh}} Test of adump, using experiment 2843. Writes SNR and phase to an output file and checks for the expected number of lines. Applicable to requirement \textbf{A-17}.
\item{\textbf{chk\_aedit.sh}} Test of aedit, using experiment 2843. Writes a simple input file, runs aedit, checks for the expected number of lines. Applicable to requirement \textbf{A-18}, but may need to be expanded.
\item{\textbf{chk\_alist.sh}} Test of alist, using 2843. Checks for the expected number of lines. Applicable to requirement \textbf{A-16}.
\item{\textbf{chk\_avefix.sh}} Uses 2843. Checks for correct number of lines in output.
\item{\textbf{chk\_average.sh}} Uses the 'testdata/average' dataset. Checks for backwards compatibility and then a bug fix. %\FIX[Need to check if this script runs.]
\item{\textbf{chk\_baselines.sh}} Uses 2843. Runs fourfit for each baseline, checks that output is the correct size.
\item{\textbf{chk\_cofit.sh}} Uses 2843 with the cofit function. Checks the number of lines in the output. Applicable to requirement \textbf{A-22}.
\item{\textbf{chk\_env.sh}} Checks that the environmental variables needed to use HOPS are set correctly.
\item{\textbf{chk\_ff\_2836.sh}} Exercises the basic fringe search and reduction, and verifies that the SNR (parsed from the postscript file) is within expected bounds.
\item{\textbf{chk\_ff\_2843.sh}} (same as above)
\item{\textbf{chk\_ff\_3372.sh}} (same as above)
\item{\textbf{chk\_ff\_3413.sh}} (same as above)
\item{\textbf{chk\_ff\_3571.sh}} subsumed into chk\_min\_weight
\item{\textbf{chk\_ff\_display.sh}} not used? gs issue?
\item{\textbf{chk\_ff\_dump.sh}} Uses 2843. Runs fourfit with a filter to discard low-amplitude segments and checks that the SNR is within expected bounds. Applicable to requirement \textbf{A-31}.
\item{\textbf{chk\_flagging.sh}} Uses 3372. Flags data segments and checks that the SNR is within expected bounds. Applicable to requirement \textbf{A-30}, particularly the case of user-supplied ad-hoc flags.
\item{\textbf{chk\_fourmer.sh}} Uses 3365. Joins two scans, runs \texttt{fourfit}, and checks that the output file size is correct. Applicable to requirement \textbf{A-19}.
\item{\textbf{chk\_fringex.sh}} Uses 2843. Runs fringex and checks the number of lines in the output. Applicable to requirement \textbf{A-20}.
\item{\textbf{chk\_frmrsrch.sh}} Uses 3365 and the output of chk\_fourmer to run search and verify the output has the correct number of lines. Applicable to requirement \textbf{A-24}.
\item{\textbf{chk\_fsumm.sh}} Applicable to requirement \textbf{A-18}.
\item{\textbf{chk\_hdlinks.sh}} Uses 2843. Applicable to requirement \textbf{D-36}.
\item{\textbf{chk\_min\_weight.sh}} Applicable to requirement \textbf{A-29}.
\item{\textbf{chk\_notches.sh}} Applicable to requirement \textbf{A-30}.
\item{\textbf{chk\_ps2pdf.sh}}
\item{\textbf{chk\_search.sh}} Applicable to requirement \textbf{A-23}.
%\item{\textbf{tst\_fourfit.sh}}
  
\end{description}


The following are requirements that we plan to validate using regression tests. 
%\FIX[Need to add details about the requirement and the test; are the check scripts, where they exist, sufficient? What functionality needs to be verified?]

\begin{description}

\item{\textbf{A-16}} chk\_alist - Checks that alist produces the appropriately sized summary file form captured data (2843).
\item{\textbf{A-17}} chk\_adump - Verifies extraction of specified quantities (snr, phase) from captured alist file works
\item{\textbf{A-18}} chk\_aedit - Verifies that aedit batch mode runs with captured script and alist file and produces postscript file.
\item{\textbf{A-19}} chk\_fourmer - Verifies that fourmer can join two (high/low) frequency bands from captured data (3365) and fourfit can run on the result.
\item{\textbf{A-20}} chk\_fringex -  Verifies refringing calculation works.
\item{\textbf{A-21}} chk\_average - Verifies fringe segmenting manipulation works.
\item{\textbf{A-22}} chk\_cofit - Verifies that cofit tool can estimate coherence.
\item{\textbf{A-23}} chk\_search - Verifies that the search tool works for a marginal fringe.
\item{\textbf{A-24}} chk\_frmrsrch -  Verifies that the search tool works on fourmer'd data.
\item{\textbf{A-29}} chk\_min\_weight - Tests that fourfit correctly applies min\_weight flagging of data.
\item{\textbf{A-30}} chk\_flagging - Tests functionality of applying ad hoc flagging file in fourfit.
\item{\textbf{D-36}} chk\_hdlinks - Test that the hops\_data\_links.pl script correctly mirrors data to another directory.
\end{description}
  


\subsection{Automated Tests that are TBD}
\label{sec:regress3}

These requirements are largely new for HOPS4, and require development. Once the functionality is available, there will be automated scripts that verify the requirement is met.
There will be a need to identify appropriate captured datasets to test each of these requirements.

Some of the tests may be computationally intensive or require large datasets, so some tests will be run infrequently or only as part of significant version rollovers 
as opposed to continuous integration tools.

\begin{description}

%\item{\textbf{A-2}} Test with large number of stations, baselines, APs.
\item{\textbf{A-3}} Test 30 stations and 435 baselines.
\item{\textbf{A-4}} Test 128 channels.
\item{\textbf{A-5}} Support bit depth of 2.

\item{\textbf{A-9}} Support Swinburne format for DiFX correlator files.
\item{\textbf{A-10}} Applications shall behave as conventional GNU/Linux applications.
\item{\textbf{A-11}} Support VEX version 1.5.
\item{\textbf{A-12}} Plotting tools support standard formats and user-provided functions.

\item{\textbf{A-31}} Data types shall be convertible to human-readable format (ASCII, CSV, etc).
\item{\textbf{A-32}} Solve for station-based quantities from baseline-based quantities.
\item{\textbf{A-33}} Apply complex bandpass corrections.
\item{\textbf{A-34}} Solve for complex bandpass corrections.
\item{\textbf{A-35}} Indteract with data produces with Python wrappers.

\item{\textbf{D-39}} Spectral-line VLBI.
\item{\textbf{A-40}} Pulsar folding.
\item{\textbf{A-41}} Multi-threading and/or multi-processing.

\end{description}


% Post Processing and Fringe Plots
%\reqid{A} The ability to explore fringes (as is currently done with the
%  combination of \acs{fringex}, \acs{average} and search must be preserved.
%  HOPS4 shall support \acs{Python} scripting to aid the user in searching
%  fringe space.
%    \reqex{Currently \texttt{chk\_fringex.sh} and \texttt{chk\_average.sh} make checks for this in HOPS 3.}
%\reqid{P} It should be possible to make 3D visualization plots.
%\reqex{\acs{search} makes contour plots, but a 3D visualization of amplitude
%  with delay and delay-rate would be useful.}
%\reqid{P} HOPS4 shall implement interactive visualization tools in \acs{fourfit}.
%\reqex{The HOPS3 version of \acs{fourfit} provides a fixed fringe summary
%  plot. HOPS4 should have an interactive plotting capability to enable zooming
%  and viewing fourfit results with an expanded scale.}
%\reqid{D} Replace the existing \acs{fourfit} control file with a
%  \acs{Python}-based control file. Currently existing functionality shall be
%  preserved.
%\reqex{This is tested with \texttt{testfourfit.sh}.The existing control file should continue to be su%pported. The new
%  control file functionality will not likely be back-ported to HOPS3.}

% Miscellaneous Requirements
%\reqid{A} It shall be possible to automatically discard correlator
%  \acs{AP} (integrations) with small weights.
%\reqid{A} HOPS4 shall have the ability to flag data based on a user-supplied
%  list (\eg~of frequency intervals with time in some flag file).
%\reqid{A} Every data type written to disk in HOPS4 shall be convertable to
%  form that is amenable to human examination (e.g. ASCII or CSV), as is done in HOPS3 by the
%  \texttt{\acs{CorAsc2}} program.
%\reqid{A} HOPS4 shall implement an algorithm for solving for station-based
%  quantities from baseline-based quantities as a global fringe fitter.
%\reqid{A} The HOPS4 \acs{fourfit} shall have the ability to apply complex
%  bandpass correction.
%\reqid{A} HOPS4 shall provide a method to solve for complex bandpass corrections.
%\reqid{A} Python wrappers for the new HOPS4 data objects shall be provided.
%\reqex{The existing wrappers for the \acs{Mk4} data types in HOPS3 do not
%  need to be preserved, but it is a capability worth preserving.}
%\reqid{D} The HOPS4 suite must provide mechanisms to preserve the correlator
%  output data.
%\reqid{D} HOPS4 should preserve the the Single Band Delay, Multi Band Delay
%  and Delay Rate search algorithm in its current form.
%\reqex{Currently \texttt{chk\_search.sh} tests this functionality in HOPS 3. Our desire is to make th%e algorithm modular, such that the user may (de)select search
%  dimensions.
%\reqid{D} Benchmarking and performance analysis should be augmented with
%  more sophisticated tools.
%\reqid{D} HOPS4 should support spectral-line VLBI.
%\reqid{D} HOPS4 should support pulsar folding with a user specified period or
%  ephemeris file with blanking.
%\reqid{D} HOPS4 should enable multi-threading or multi-processing for batch jobs.
%\end{description}

%
% eof
%
