%
% A starter on coverate and unit testing
%
\section{Regression Testing}
\label{sec:regress}

this can be words about internal regression testing with autotools
(which we can do now...)


\subsection{Regression Tests with Captured Data}
\label{sec:captdatareq}

% ./data/ff_testdata/Makefile.am
In addition to the above tests, data from a variety of experiments
has been captured and are routinely refringed as part of the
HOPS build ``check'' process.  Data from two of
these are packaged with the HOPS distribution:

\begin{description}
\reqid{A} \texttt{chk\_ff\_2836.sh} a canonical S--X geodetic experiment
\reqid{A} \texttt{chk\_ff\_3365.sh} an early \ac{EHT} session suitable for
    testing \ac{fourmer}
\end{description}
Both of these tests exercise the basic fringe search and reduction
that \acs{fourfit} was built to do and have been the basis of standard
regression for HOPS and distribution validation.   Additional tests
with other captured data sets are discussed in the next section.



%Some additional tests extant in \ac{HOPS} should not be needed in
%\ac{MHO}; these are listed for reference in \App~\ref{sec:obsolete}.

%\subsection{Software-Testable Requirements}
%\label{sec:regress}

%It is anticipated that each of these tests will be captured by a
%script that may be executed on a regular basis, possibly requiring
%user intervention.  For example, when run in an interactive mode,
%\ac{fourfit} displays fringe plots which require a human to acknowledge
%that they have in fact been displayed.  (One can test the generation of
%the plot without a human, but the display mechanics require visual
%inspection.)

% ./sub/dfio/copypage/Makefile.am

Additional scans have been captured with fringe tests captured
in the \ac{HOPS} development system.  (That is, the tests exist,
but the data is not included in the distribution tarball.)

% ./data/ff_testdata/Makefile.am
\begin{description}
\reqid{S} verify that experiment 2843 fringes consistently.
  \reqex{currently done with \texttt{chk\_ff\_2843.sh} in HOPS}
\reqid{S} verify that experiment 3372 fringes consistently.
  \reqex{currently done with \texttt{chk\_ff\_3372.sh} in HOPS}
\reqid{S} verify that experiment 3413 fringes consistently.
  \reqex{currently done with \texttt{chk\_ff\_3413.sh} in HOPS}

\reqid{P} Regression tests of 2017 or 2018 data samples
  \reqex{Among the so-called ``golden'' scan data which has been
    captured for \acs{DiFX} regression testing, there are scans
    which can be refringed and \acs{A-list} manipulated and compared
    with previous results.  As set that is sufficient to guarantee
    consistency with published results is required.  Reprocessing
    the entire 2017 and 2018 data is not required.}

\end{description}

Additional test data has been captured from the (very) successful
2017 \ac{EHT} observations.  Fringing of these scans has not yet
been incorporated into HOPS testing, but shall form the basis
of regression for HOPS4.

It is anticipated that for these a mechanism shall be provided
so that they human need merely acknowledge success, failure or
make a note of issues.

Currently captured for \acs{DiFX} regression testing are the first 20 seconds
on the following experiment-release-subband-doy-hhmm samples:
\begin{verbatim}
e17d05-7-hi-095-0839 e17d05-7-lo-095-0839 e17d05-7-pc-095-0839
e17b06-7-hi-096-0750 e17b06-7-lo-096-0750 e17b06-7-pc-096-0750
e17c07-7-hi-097-1303 e17c07-7-lo-097-1303 e17c07-7-pc-097-1303
e17a10-7-hi-100-0725 e17a10-7-lo-100-0725 e17a10-7-pc-100-0725
e17e11-7-hi-101-0322 e17e11-7-hi-101-0803 e17e11-7-lo-101-0322
e17e11-7-lo-101-0803 e17e11-7-pc-101-0322 e17e11-7-pc-101-0803
\end{verbatim}
``lo'' and ``hi'' refer to the ``b3'' and ``b4'' sub-bands of the
64 \acs{Gbps} (2018 and onwards) frequency setup; ``pc'' refers to polconverted,rather than the mixed polarization correlation products.

\TBC\FIX[---is this a sufficient set?]

These or other scans could also be captured for full scan duration and
segmented at some interval, but it is not clear that this is necessary.


%
% eof
%
