%
% A starter on coverate and unit testing
%
\section{Manual Testing}
\label{sec:manual}

Manual tests are intended to verify requirements that demand human interpretation, for example a plotting GUI.

%this can be words about procedural testing of the command line
%and GUI interfaces
% Not sure if they need as much detail as this. This table would consist of the actual tests that would be run.
%Manual testing involves a developer running the latest build and testing the user interface      
%\begin{table}[h!]
%\centering
% \begin{tabular}{|c c c c|} 
% \hline
% Test ID & Description & Steps & Pass/Fail Criteria \\ [0.5ex] 
% \hline
% 01 & Test if HOPS builds & 1. Run lorem ipsum. 2. ? 3. Profit  & Pass if able to build the project without any errors. 
% 02 & Test if HOPS documentation builds & 1...inf & Pass if able to build the project \LaTeX files without any errors. \\ [1ex]
% \hline
% \end{tabular}
 %\caption{HOPS4 documents referenced by this document}
% \label{table:1}
%\end{table}

\subsection{Manual testing to be applied to following requirements}
\label{sec:manreqs}

\begin{description}
  
\item{\textbf{P-7}} HOPS shall not require FFTW3 and/or GSL and shall operate without loss of functionality in the absence of these packages.
\item{\textbf{P-8}} HOPS shall not require PGPLOT for basic functionality.
\item{\textbf{P-25}} Make 2D plots of the type currently possible within aedit. Currently partially automated with \textbf{chk\_aedit.sh}.
\item{\textbf{P-26}} Make 3D visualization plots. At first, some human reviewer will need to validate the plots; once the plotting routine is functional there can be an automated script that checks that the plots can be generated from captured data.
\item{\textbf{P-27}} Implement interactive visualization tools.

\end{description}
  



%\begin{description}
% File I/O and Supported File Types
%\reqid{A} HOPS4 shall support input from the DiFX \cite{deller2007difx,deller2011difx} correlator (in Swinburne format).
%\reqid{A} every application should provide
%  \verb+--help+ and \verb+--version+ responses (with zero exit status)
%  to behave as conventional \acs{GNU/Linux} applications.
%\reqex{Command-line arguments should be implemented with a
%  standard argument parsing library, and should respond helpfully
%  on bad command-line syntax.}
%\reqid{A} HOPS4 shall support input from \acs{VEX} version 1.5.
%\reqid{A} All plotting tools provided in HOPS4 shall have the ability to save the image to
%  a standard graphics format.
%\reqex{HOPS4 has the desire to support as much plotting flexibility as possible, including plot formats (fonts, panels, etc).}
%\reqid{D} HOPS4 shall provide an application to support export of native
%  data types to a \TBD~archival format (\eg~HDF5).

% Post Processing and Fringe Plots
%\reqid{A} The functionality of the program \acs{cofit} shall be preserved.
%  \reqex{\texttt{chk\_cofit.sh} currently verifies this.}
%\reqid{P} It should be possible to make 2D plots of the type currently possible
%  within \texttt{\acs{aedit}} of HOPS3.
%\reqex{These are plots such as \acs{SNR} with time broken out by baseline with
%  separate symbols per target. It is not currently possible, but should be possible
%  to combine several baselines into a composite plot.}
%\reqid{P} HOPS4 shall implement interactive visualization tools in \acs{fourfit}.
%and viewing fourfit results with an expanded scale.}
%\reqid{D} Replace the existing \acs{fourfit} control file with a
%  \acs{Python}-based control file. Currently existing functionality shall be
%  preserved.
%\reqex{This is tested with \texttt{testfourfit.sh}.The existing control file should continue to be supported. The new
%  control file functionality will not likely be back-ported to HOPS3.}
%\reqid{A} Every data type written to disk in HOPS4 shall be convertable to
%  form that is amenable to human examination (e.g. ASCII or CSV), as is done in HOPS3 by the
%  \texttt{\acs{CorAsc2}} program.
%\end{description}
%
% eof
%
