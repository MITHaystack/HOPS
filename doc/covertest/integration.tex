%
% A starter on coverate and unit testing
%
\section{Continuous Integration Testing}
\label{sec:integration}

%this can be words about integration regression testing with scripts
Continuous Integration is the process of merging all the working copies of the code
base in to master and running automated tests on the result of that merge, often 
several times a day.

Currently HOPS3 supports continuous integration by way of nightly automated scripts 
that report back the results of those tests to the development team.
The HOPS development team plans to continue to utilize this automated testing
and make it compatible with HOPS4. The HOPS development team plans to explore the posibility
of replacing the current nightly automated script with a freely available and widely
accepted CI tool utilized by the open source development community such as Github Actions. 

% This section should prbably include all of the same requirements mentioned in the integration section.
\subsection{Regression testing to be applied to following requirements}
\label{sec:integrationreqs}

\begin{description}
% Post Processing and Fringe Plots
\reqid{A} The ability to explore fringes (as is currently done with the
  combination of \acs{fringex}, \acs{average} and search must be preserved.
  HOPS4 shall support \acs{Python} scripting to aid the user in searching
  fringe space.
\reqid{P} It should be possible to make 3D visualization plots.
\reqex{\acs{search} makes contour plots, but a 3D visualization of amplitude
  with delay and delay-rate would be useful.}
\reqid{P} HOPS4 shall implement interactive visualization tools in \acs{fourfit}.
\reqex{The HOPS3 version of \acs{fourfit} provides a fixed fringe summary
  plot. HOPS4 should have an interactive plotting capability to enable zooming
  and viewing fourfit results with an expanded scale.}
\reqid{D} Replace the existing \acs{fourfit} control file with a
  \acs{Python}-based control file. Currently existing functionality shall be
  preserved.
\reqex{The existing control file should continue to be supported. The new
  control file functionality will not likely be back-ported to HOPS3.}

% Miscellaneous Requirements
\reqid{A} It shall be possible to automatically discard correlator
  \acs{AP} (integrations) with small weights.
\reqid{A} HOPS4 shall have the ability to flag data based on a user-supplied
  list (\eg~of frequency intervals with time in some flag file).
\reqid{A} Every data type written to disk in HOPS4 shall be convertable to
  form that is amenable to human examination (e.g. ASCII or CSV), as is done in HOPS3 by the
  \texttt{\acs{CorAsc2}} program.
\reqid{A} HOPS4 shall implement an algorithm for solving for station-based
  quantities from baseline-based quantities as a global fringe fitter.
\reqid{A} The HOPS4 \acs{fourfit} shall have the ability to apply complex
  bandpass correction.
\reqid{A} HOPS4 shall provide a method to solve for complex bandpass corrections.
\reqid{A} Python wrappers for the new HOPS4 data objects shall be provided.
\reqex{The existing wrappers for the \acs{Mk4} data types in HOPS3 do not
  need to be preserved, but it is a capability worth preserving.}
\reqid{D} The HOPS4 suite must provide mechanisms to preserve the correlator
  output data.
\reqid{D} HOPS4 should preserve the the Single Band Delay, Multi Band Delay
  and Delay Rate search algorithm in its current form.
\reqex{Our desire is to make the algorithm modular, such that the user may (de)select search
  dimensions.
\reqid{D} Benchmarking and performance analysis should be augmented with
  more sophisticated tools.
\reqid{D} HOPS4 should support spectral-line VLBI.
\reqid{D} HOPS4 should support pulsar folding with a user specified period or
  ephemeris file with blanking.
\reqid{D} HOPS4 should enable multi-threading or multi-processing for batch jobs.
\end{description}


%
% eof
%
